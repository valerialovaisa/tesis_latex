\chapter{Guía Básica de Implementación del RTOS ecOS} \label{app:apendice4}

\section{Descarga del código fuente del proyecto}

Inicialmente se deben descargar los fuentes de la sistema operativo desde el repositorio SVN mediante:

\begin{lstlisting}[breaklines]
 usuario@usuario-desktop:~/$ svn co http://opencores.org/ocsvn/openrisc/openrisc/trunk/rtos/ecos-3.0
\end{lstlisting}

\section{Descarga de la aplicación de configuración de ecOS}

La implementación de ecOS requiere de una aplicación de configuración que permite realizar la configuración básica del sistema. El código fuente de
esta herramienta se encuentra disponible para su descarga directa mediante el comando:

\begin{lstlisting}[breaklines]
 usuario@usuario-desktop:~/$ wget http://www.ecoscentric.com/snapshots/configtool-100305.bz2
\end{lstlisting}

Para descomprimir el paquete descargado se utiliza:

\begin{lstlisting}[breaklines]
usuario@usuario-desktop:~/$ bunzip2 configtool-100305.bz2
usuario@usuario-desktop:~/$ chmod 755 configtool-100305
usuario@usuario-desktop:~/$ sudo cp configtool-100305 /usr/local/bin/.
usuario@usuario-desktop:~/$ sudo ln -s /usr/local/bin/configtool-100305 /usr/local/bin/configtool
\end{lstlisting}

\section{Instalación de dependencias de compilación}

Para la compilación son necesarias algunas dependencias que deben instalarse mediante la siguiente secuencia de comandos:

\begin{lstlisting}[breaklines]
 usuario@usuario-desktop:~/$ sudo apt-get install tcl8.4-dev
 usuario@usuario-desktop:~/$ sudo apt-get install tk8.4-dev
\end{lstlisting}

A continuación deben copiarse los scripts de configuración a la ruta /usr/local/lib

\begin{lstlisting}[breaklines]
usuario@usuario-desktop:~/$ sudo cp /usr/lib64/tcl8.4/tclConfig.sh /usr/local/lib/.
usuario@usuario-desktop:~/$ sudo cp /usr/lib64/tk8.4/tkConfig.sh /usr/local/lib/.
\end{lstlisting}


\section{Compilación de la herramienta de configuración}

Una vez realiza instaladas las depencias se debe compilar la herramienta de configuración de ecOS como se detalla a continuación: 

\begin{lstlisting}[breaklines]
 usuario@usuario-desktop:~/<aplicacion de configuración>/$ make
 usuario@usuario-desktop:~/<aplicacion de configuración>/$ sudo make install
\end{lstlisting}

Debe agregarse la siguiente línea al archivo .bashrc.

\begin{lstlisting}[breaklines]
export ECOS_REPOSITORY=/opt/home/svan/OpenRISC/ecos/ecos-3.0/packages
\end{lstlisting}

Es necesario que se reconfiguren las variables mediante:
\begin{lstlisting}[breaklines]
 usuario@usuario-desktop:~/$ source ~/.bashrc
\end{lstlisting}

\section{Configuración del sistema}

\begin{lstlisting}[breaklines]
 usuario@usuario-desktop:~/<directorio de ecos>/$ ecosconfig new orpsoc 
\end{lstlisting}

\begin{lstlisting}[breaklines]
usuario@usuario-desktop:~/<directorio de ecos>/$ configtool ecos.ecc
\end{lstlisting}

\begin{lstlisting}[breaklines]
usuario@usuario-desktop:~/<directorio de ecos>/$ ecosconfig check
\end{lstlisting}


\section{Contrucción del sistema }


\begin{lstlisting}[breaklines]
usuario@usuario-desktop:~/<directorio de ecos>/$ ecosconfig tree
usuario@usuario-desktop:~/<directorio de ecos>/$ make
usuario@usuario-desktop:~/<directorio de ecos>/$ make tests
\end{lstlisting}

\section{Escribir un programa de prueba}

\begin{lstlisting}[language=C,frame=single]
#include <stdio.h>
int main(void) {
 printf("Hello, eCos world!\n");
return 0;
 }
\end{lstlisting}

\section{Compilación de programas de prueba}

\begin{lstlisting}[breaklines]
 usuario@usuario-desktop:~/$ or32-elf-gcc -g -O2 -nostdlib -install/include -Linstall/lib -Tinstall/lib/target.ld examples/hello.c -o hello.elf  
\end{lstlisting}



\section{Ejecución de aplicaciones bajo ecOS}

Para conectar con éxito el TAP del SoC que contiene el micro OpenRISC , se debe copiar los archivos BSDL de todos los dispositivos conectados a la cadena de JTAG a un directorio conocido (por ejemplo /home/user/bsdl).Luego debe colocarse, el directorio como argumento al llamar adv\_jtag\_bridge, adv\_jtag\_bridge-b /home/user/bsdl.

\begin{enumerate}

\item Conectar el cable JTAG al TAP
\item Iniciar or\_debug\_proxy

\begin{lstlisting}[breaklines]
usuario@usuario-desktop:~/$ sudo or_debug_proxy -c ftdi -r 50000 
\end{lstlisting}

\item Una vez que el programa se encuentre en funcionamiento se debe abrir otro terminal (por ejemplo gtkterm).
\item Configuramos el puerto a un puerto serie conectado a la placa.
\item Configuramos el bitrate a 115200.

\item Se inicia gdb y se carga el firmware de ejemplo compilado previamente para OpenRisc.

\item 
\begin{lstlisting}[breaklines]
usuario@usuario-desktop:~/$ or32-elf-gdb hello.elf
target remote :50000
load
set $pc=0x100
c
\end{lstlisting}

\item Dentro de gtkterm debería haber aparecido "Hello, eCos World!" 

\end{enumerate}
