\chapter{Presentación de Resultados}
	\section{Introducción} 
	
	Este capítulo presenta los resultados obtenidos durante la implementación de cada uno de los prototipos planteados en el Capítulo
	~\ref{chap:disenoeimpl} y durante la ejección de los casos de prueba correspondientes. 
	
	Para la ejecución de todas las pruebas se utilizó la placa desarrollo S3ADSP1800A del fabricante Xilinx que cumple con los requerimientos detallados
	en la Tabla ~\ref{tab:requsr1} y se encontraba dentro de las alternativas disponibles al momento del desarrollo de este trabajo. Aún cumpliendo con
	los requerimientos especificados, la placa de desarrollo no cuenta con un completo soporte de periféricos on board ni con amplia documentación de
	apoyo respecto de la materia. Se presentaron grandes dificultades en el acceso a la memoria SPI FLASH S33 de Intel la cual se encuentra soportada por
	herramientas \textit{oficiales} que únicamente corren bajo Windows. Alternativamente existe una versión de la placa de desarrollo S3ADSP1800A,
	disponible también en el laboratorio del CUDAR, que se encuentra equipada con una memoria FLASH SPI Numonyx M25P64 que puede ser accedida mediante
	herramientas de programación como XC3SPROG y UrJTAG alojando finalmente los programas necesarios para el arranque del sistema.
	
	Las pruebas realizadas con el sistema operativo de tiempo real ecOS proveyeron información útil para el desarrollo de sistemas embebidos de tiempo
	real. Se analizaron inicialmente las capacidades y limitaciones en la ejecución de hilos. Aunque estas pruebas tan solo verifican la utilización de
	una parte las capacidades, el sistema operativo ecOS posee mayor funcionalidad que no fue probada en este trabajo y presenta capacidades comparables
	a implementaciones como lo son FreeRTOS y su implentación comercial eCosPro.
	
	La capacidad, por defecto, del Kernel de Linux de ser compilado para arquitecturas OpenRISC posibilitó tener un entorno de ejecución de amplia
	funcionalidad y gran utilización en el ámbito de desarrollo de Sistemas Embebidos.  
	
	\section{Estudio de capacidades del proyecto MinSoC}
	El proyecto MinSoC se encuentra enfocado a su utilización en sistemas embebidos de capacidades ajustadas sintetizables en una gran cantidad de FPGA
	de diversos desarrolladores. La facilidad de adaptación del proyecto para ser portado a otras arquitecturas reconfigurables le confiere gran
	versatilidad ampliando notablemente su gama de aplicación.
	Durante el desarrollo de las pruebas se prentendió esteblecer los límites de aplicación del proyecto que establezcan referencias sólidas para una
	futura elección del proyecto en aplicaciones reales.

		\subsection{Resultados de la ejecución del multiplicador de matrices binarias}
			

	

	\section{Estudio de capacidades del proyecto ORPSoC}
    

		\subsection{Resultados de la ejecución del Benchmark Drystone}

		
		\subsection{Resultados de la ejecución del Benchmark CoreMark}
		
	
	\section{Estudio de factibilidad para la implentación de RTOS Embebidos}
	
		\subsection{Resultados del programa de prueba de hilos}
		
			
		
	\section{Estudio de factibilidad para la implentación Linux}
	
	
	 