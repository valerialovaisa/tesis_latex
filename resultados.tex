\chapter{Presentación de Resultados}
	\section{Introducción} 
	
	Este capítulo presenta los resultados obtenidos durante la implementación de cada uno de los prototipos planteados en el Capítulo
	~\ref{chap:disenoeimpl} y durante la ejección de los casos de prueba correspondientes. 
	
	Para la ejecución de todas las pruebas se utilizó la placa desarrollo S3ADSP1800A del fabricante Xilinx que cumple con los requerimientos detallados
	en la Tabla ~\ref{tab:requsr1} y se encontraba dentro de las alternativas disponibles al momento del desarrollo de este trabajo. Aún cumpliendo con
	los requerimientos especificados, la placa de desarrollo no cuenta con un completo soporte de periféricos on board ni con amplia documentación de
	apoyo respecto de la materia. Se presentaron grandes dificultades en el acceso a la memoria SPI FLASH S33 de Intel la cual se encuentra soportada por
	herramientas \textit{oficiales} que únicamente corren bajo Windows. Alternativamente existe una versión de la placa de desarrollo S3ADSP1800A,
	disponible también en el laboratorio del CUDAR, que se encuentra equipada con una memoria FLASH SPI Numonyx M25P64 que puede ser accedida mediante
	herramientas de programación como XC3SPROG y UrJTAG alojando finalmente los programas necesarios para el arranque del sistema.
	
	Las pruebas realizadas con el sistema operativo de tiempo real ecOS proveyeron información útil para el desarrollo de sistemas embebidos de tiempo
	real. Se analizaron inicialmente las capacidades y limitaciones en la ejecución de hilos. Aunque estas pruebas tan solo verifican la utilización de
	una parte las capacidades, el sistema operativo ecOS posee mayor funcionalidad que no fue probada en este trabajo y presenta capacidades comparables
	a implementaciones como lo son FreeRTOS y su implentación comercial eCosPro.
	
	La capacidad, por defecto, del Kernel de Linux de ser compilado para arquitecturas OpenRISC posibilitó tener un entorno de ejecución de amplia
	funcionalidad y gran utilización en el ámbito de desarrollo de Sistemas Embebidos.  
	
	\section{Estudio de capacidades del proyecto MinSoC}

		\subsection{Place and route}	
\begin{lstlisting}[frame=single,caption={Resultado de PAR},label={lst:salidas},breaklines]

Design Summary Report:

 Number of External IOBs                          29 out of 519     5%

   Number of External Input IOBs                 15

      Number of External Input IBUFs             15
        Number of LOCed External Input IBUFs     14 out of 15     93%


   Number of External Output IOBs                13

      Number of External Output IOBs             13
        Number of LOCed External Output IOBs      9 out of 13     69%


   Number of External Bidir IOBs                  1

      Number of External Bidir IOBs               1
        Number of LOCed External Bidir IOBs       1 out of 1     100%


   Number of BSCANs                          1 out of 1     100%
   Number of BUFGMUXs                        5 out of 24     20%
   Number of DCMs                            1 out of 8      12%
   Number of DSP48As                         4 out of 84      4%
   Number of RAMB16BWERs                    70 out of 84     83%
   Number of Slices                       8475 out of 16640  50%
      Number of SLICEMs                    167 out of 8320    2%
\end{lstlisting}

	El proyecto MinSoC se encuentra enfocado a su utilización en sistemas embebidos de capacidades ajustadas sintetizables en una gran cantidad de FPGA
	de diversos desarrolladores. La facilidad de adaptación del proyecto para ser portado a otras arquitecturas reconfigurables le confiere gran
	versatilidad ampliando notablemente su gama de aplicación.
	Durante el desarrollo de las pruebas se prentendió esteblecer los límites de aplicación del proyecto que establezcan referencias sólidas para una
	futura elección del proyecto en aplicaciones reales.

\newpage
		\subsection{Resultados de la ejecución del multiplicador de matrices binarias}
		Se presentan a continuación los resultados de ejecución del programa multiplicador de matrices binarias. Se utilizó un gráfico que muestra el orden
		de las matrices multiplicadas vs. la cantidad de ticks de ejecución. El gráfico ~\ref{fig:mulmat} muestra con claridad que el tiempo de ejecución
		aumenta exponencialmente a medida que aumenta el orden de las matrices multiplicadas.  
		
\begin{figure}[h!]
 	\begin{center}
  	\includegraphics[width=0.6\textwidth,keepaspectratio=true]{./images/matrices}
  	\caption{Multiplicación de Matrices Binarias}
  	\label{fig:mulmat}
 	\end{center}
	\end{figure}

		Teniendo en cuenta que la aplicación fue ejecutada en una versión sintentizada básica del proyecto minSoC, este no posee unidades de multiplicación
		por hardware que efectuen cálculos con mayor eficiencia reduciendo el tiempo de ejecución. Pueden considerarse también las optimizaciones del
		compilador 


\newpage
	\section{Estudio de capacidades del proyecto ORPSoC}

		\subsection{Place and route}	
Los resultados de la herramienta de lugar y ruta se pueden encontrar en el archivo de registro <orpsoc.par>

\begin{lstlisting}[frame=single,caption={Resultado de PAR},label={lst:salidas},breaklines]

Design Summary Report:

 Number of External IOBs                         128 out of 519    24%

   Number of External Input IOBs                 21

      Number of External Input IBUFs             21

   Number of External Output IOBs                44

      Number of External Output DIFFMLRs          2
      Number of External Output DIFFSLRs          2
      Number of External Output IOBs             14
      Number of External Output IOBLRs           26

   Number of External Bidir IOBs                 63

      Number of External Bidir DIFFMLRs           4
      Number of External Bidir DIFFSLRs           4
      Number of External Bidir IOBs              22
      Number of External Bidir IOBLRs            33

   Number of BUFGMUXs                        7 out of 24     29%
   Number of DCMs                            2 out of 8      25%
   Number of DSP48As                         4 out of 84      4%
   Number of RAMB16BWERs                    16 out of 84     19%
   Number of Slices                       7772 out of 16640  46%
    Number of SLICEMs                    612 out of 8320    7%

\end{lstlisting}

		\subsection{Reporte de TIMING}	

\begin{lstlisting}[frame=single,caption={Reporte timing},label={lst:salidas},breaklines]
Timing summary:
---------------

Timing errors: 498  Score: 729028  (Setup/Max: 311057, Hold: 417971)

Constraints cover 395663179 paths, 94 nets, and 53193 connections

Design statistics:
  Minimum period:  45.166ns   (Maximum frequency:  22.141MHz)
  Maximum path delay from/to any node:  11.696ns
  Maximum net delay:   2.594ns
  Minimum input required time before clock:  26.237ns
  Maximum output delay after clock:  13.126ns

Analysis completed Sat Nov  2 11:19:56 2013 

Trace Settings:

Peak Memory Usage: 299 MB
\end{lstlisting}


		\subsection{Condiciones de entorno de ejecución para benchmark}
		En la siguente tabla ~\ref{tab:conbench} se muestran las condiciones del entorno de prueba durante los benchmarks.

\begin{table}[h!]
\begin{tabular}{ |p{5cm} |p{10cm}| }    
\hline
\multicolumn{2}{|>{\columncolor[gray]{.8}}c}{Condiciónes de entorno de prueba|}\\
		\hline
		Placa de desarrollo & S3ADSP1800A  \\
		\hline 
		FPGA & Xilinx Spartan-3 XC3SD1800A \\ 
		\hline 
		Reloj del procesador & 25 MHz\\ 
		\hline
		Caché de instrucciones  & 8 KB \\ 
		\hline
		Caché de datos	  & 8 KB\\ 
		\hline	
		MMU & Sí \\	
		\hline
		Multiplicador hardware & Sí \\		
		\hline	
		División hardware & Sí \\		
		\hline	
\end{tabular}
%\end{center}
\caption{Condiciones del entorno de prueba}
\label{tab:conbench}
\end{table}

				\subsection{Resultados de la ejecución del Benchmark CoreMark}
		
Esto le da un valor de CoreMark 41288895/25MHz = 1.65/MHz. Se trato de de mejorar este valor mediante la adición de algunas opciones del compilador.

Aquí están los resultados de tratar de optimizar la fase de compilación. Al ser compilado con el nivel de optimización -O2 se observaron los resultados mostrados en el bloque ~\ref{lst:salidas} 

\begin{lstlisting}[frame=single,caption={Optimización nivel -O2},label={lst:salidas},breaklines]
Iterations/Sec   : 39.292731
Iterations       : 600
Compiler version : GCC4.5.1-or32-1.0rc4
Compiler flags   : -O2  -mboard=s3adsp1800a -FUNROLL-LOOPS -DVALIDATION_RUN=1  
Memory location  : STACK
seedcrc          : 0x18f2
[0]crclist       : 0xe3c1
[0]crcmatrix     : 0x0747
[0]crcstate      : 0x8d84
[0]crcfinal      : 0xb8a0
\end{lstlisting}

\begin{lstlisting}[frame=single,caption={Optimización nivel -O2},label={lst:salidas},breaklines]
2K validation run parameters for coremark.
CoreMark Size    : 666
Total ticks      : 1528
Total time (secs): 15.280000
Iterations/Sec   : 39.267016
Iterations       : 600
Compiler version : GCC4.5.1-or32-1.0rc4
Compiler flags   : -O2  -mboard=s3adsp1800a -MSOFT-FLOAT -DVALIDATION_RUN=1  
Memory location  : STACK
seedcrc          : 0x18f2
[0]crclist       : 0xe3c1
[0]crcmatrix     : 0x0747
[0]crcstate      : 0x8d84
[0]crcfinal      : 0xb8a0
\end{lstlisting}

\begin{lstlisting}[frame=single,caption={Optimización nivel -O2},label={lst:salidas},breaklines]
2K validation run parameters for coremark.
CoreMark Size    : 666
Total ticks      : 1528
Total time (secs): 15.280000
Iterations/Sec   : 39.267016
Iterations       : 600
Compiler version : GCC4.5.1-or32-1.0rc4
Compiler flags   : -O2  -mboard=s3adsp1800a -FUNROLL-ALL-LOOPS -DVALIDATION_RUN=1  
Memory location  : STACK
seedcrc          : 0x18f2
[0]crclist       : 0xe3c1
[0]crcmatrix     : 0x0747
[0]crcstate      : 0x8d84
[0]crcfinal      : 0xb8a0
\end{lstlisting}

\begin{lstlisting}[frame=single,caption={Optimización nivel -O2},label={lst:salidas},breaklines]
2K validation run parameters for coremark.
CoreMark Size    : 666
Total ticks      : 1527
Total time (secs): 15.270000
Iterations/Sec   : 39.292731
Iterations       : 600
Compiler version : GCC4.5.1-or32-1.0rc4
Compiler flags   : -O2  -mboard=s3adsp1800a -FGCSE-SM -DVALIDATION_RUN=1  
Memory location  : STACK
seedcrc          : 0x18f2
[0]crclist       : 0xe3c1
[0]crcmatrix     : 0x0747
[0]crcstate      : 0x8d84
[0]crcfinal      : 0xb8a0
\end{lstlisting}

\begin{lstlisting}[frame=single,caption={Optimización nivel -O3},label={lst:salidas},breaklines]
2K validation run parameters for coremark.
CoreMark Size    : 666
Total ticks      : 1654
Total time (secs): 16.540000
Iterations/Sec   : 36.275695
Iterations       : 600
Compiler version : GCC4.5.1-or32-1.0rc4
Compiler flags   : -O3  -mboard=s3adsp1800a -DVALIDATION_RUN=1  
Memory location  : STACK
seedcrc          : 0x18f2
[0]crclist       : 0xe3c1
[0]crcmatrix     : 0x0747
[0]crcstate      : 0x8d84
[0]crcfinal      : 0xb8a0
\end{lstlisting}
\begin{lstlisting}[frame=single,caption={Optimización nivel -O3},label={lst:salidas},breaklines]
2K validation run parameters for coremark.
CoreMark Size    : 666
Total ticks      : 1654
Total time (secs): 16.540000
Iterations/Sec   : 36.275695
Iterations       : 600
Compiler version : GCC4.5.1-or32-1.0rc4
Compiler flags   : -O3  -mboard=s3adsp1800a -MHARD-DIV -MHARD-MULT -DVALIDATION_RUN=1  
Memory location  : STACK
seedcrc          : 0x18f2
[0]crclist       : 0xe3c1
[0]crcmatrix     : 0x0747
[0]crcstate      : 0x8d84
[0]crcfinal      : 0xb8a0
\end{lstlisting}
\begin{lstlisting}[frame=single,caption={Optimización nivel -O3},label={lst:salidas},breaklines]
2K validation run parameters for coremark.
CoreMark Size    : 666
Total ticks      : 1654
Total time (secs): 16.540000
Iterations/Sec   : 36.275695
Iterations       : 600
Compiler version : GCC4.5.1-or32-1.0rc4
Compiler flags   : -O3  -mboard=s3adsp1800a -FUNROLL-LOOPS -DVALIDATION_RUN=1  
Memory location  : STACK
seedcrc          : 0x18f2
[0]crclist       : 0xe3c1
[0]crcmatrix     : 0x0747
[0]crcstate      : 0x8d84
[0]crcfinal      : 0xb8a0
\end{lstlisting}
\begin{lstlisting}[frame=single,caption={Optimización nivel -O3},label={lst:salidas},breaklines]
2K validation run parameters for coremark.
CoreMark Size    : 666
Total ticks      : 1655
Total time (secs): 16.550000
Iterations/Sec   : 36.253776
Iterations       : 600
Compiler version : GCC4.5.1-or32-1.0rc4
Compiler flags   : -O3  -mboard=s3adsp1800a -FUNROLL-LOOPS -DVALIDATION_RUN=1  
Memory location  : STACK
seedcrc          : 0x18f2
[0]crclist       : 0xe3c1
[0]crcmatrix     : 0x0747
[0]crcstate      : 0x8d84
[0]crcfinal      : 0xb8a0
\end{lstlisting}
\begin{lstlisting}[frame=single,caption={Optimización nivel -O3},label={lst:salidas},breaklines]
2K validation run parameters for coremark.
CoreMark Size    : 666
Total ticks      : 1654
Total time (secs): 16.540000
Iterations/Sec   : 36.275695
Iterations       : 600
Compiler version : GCC4.5.1-or32-1.0rc4
Compiler flags   : -O3  -mboard=s3adsp1800a -FGCSE-SM -DVALIDATION_RUN=1  
Memory location  : STACK
seedcrc          : 0x18f2
[0]crclist       : 0xe3c1
[0]crcmatrix     : 0x0747
[0]crcstate      : 0x8d84
[0]crcfinal      : 0xb8a0
\end{lstlisting}
\begin{lstlisting}[frame=single,caption={Optimización nivel -O3},label={lst:salidas},breaklines]
2K validation run parameters for coremark.
CoreMark Size    : 666
Total ticks      : 1655
Total time (secs): 16.550000
Iterations/Sec   : 36.253776
Iterations       : 600
Compiler version : GCC4.5.1-or32-1.0rc4
Compiler flags   : -O3  -mboard=s3adsp1800a -MSOFT-FLOAT -DVALIDATION_RUN=1  
Memory location  : STACK
seedcrc          : 0x18f2
[0]crclist       : 0xe3c1
[0]crcmatrix     : 0x0747
[0]crcstate      : 0x8d84
[0]crcfinal      : 0xb8a0
\end{lstlisting}
\begin{lstlisting}[frame=single,caption={Optimización nivel -O3},label={lst:salidas},breaklines]
2K validation run parameters for coremark.
CoreMark Size    : 666
Total ticks      : 1654
Total time (secs): 16.540000
Iterations/Sec   : 36.275695
Iterations       : 600
Compiler version : GCC4.5.1-or32-1.0rc4
Compiler flags   : -O3  -mboard=s3adsp1800a -FUNROLL-ALL-LOOPS -DVALIDATION_RUN=1  
Memory location  : STACK
seedcrc          : 0x18f2
[0]crclist       : 0xe3c1
[0]crcmatrix     : 0x0747
[0]crcstate      : 0x8d84
[0]crcfinal      : 0xb8a0
\end{lstlisting}
\begin{lstlisting}[frame=single,caption={Optimización nivel -O3},label={lst:salidas},breaklines]
2K validation run parameters for coremark.
CoreMark Size    : 666
Total ticks      : 1655
Total time (secs): 16.550000
Iterations/Sec   : 36.253776
Iterations       : 600
Compiler version : GCC4.5.1-or32-1.0rc4
Compiler flags   : -O3  -mboard=s3adsp1800a -FUNROLL-LOOPS -MSOFT-FLOAT -FUNROLL-ALL-LOOPS -DVALIDATION_RUN=1  
Memory location  : STACK
seedcrc          : 0x18f2
[0]crclist       : 0xe3c1
[0]crcmatrix     : 0x0747
[0]crcstate      : 0x8d84
[0]crcfinal      : 0xb8a0
\end{lstlisting}
\begin{lstlisting}[frame=single,caption={Optimización nivel -O3},label={lst:salidas},breaklines]
2K validation run parameters for coremark.
CoreMark Size    : 666
Total ticks      : 1654
Total time (secs): 16.540000
Iterations/Sec   : 36.275695
Iterations       : 600
Compiler version : GCC4.5.1-or32-1.0rc4
Compiler flags   : -O3  -mboard=s3adsp1800a -FUNROLL-LOOPS -MSOFT-FLOAT -FUNROLL-ALL-LOOPS -FGCSE-SM -DVALIDATION_RUN=1  
Memory location  : STACK
seedcrc          : 0x18f2
[0]crclist       : 0xe3c1
[0]crcmatrix     : 0x0747
[0]crcstate      : 0x8d84
[0]crcfinal      : 0xb8a0
\end{lstlisting}

	\subsection {Presentación de los resultados de optimización} 
\begin{table}[h!]
\begin{center}
\begin{tabular}{ |l |l| l|}
\hline
\rowcolor[gray]{0.8} Opciones del compilador&-O2&-O3 \\
\hline
Sin extras 					&1.65 			&1.54\\
\hline
-mhard-div -mhard-mu 	& 1.57			&1.45 \\
\hline
-funroll-loops			 	& 1.57			& 1.45 \\
\hline
-fgcse-sm					& 1.57			& 1.45 \\
\hline
-msoft-float 				& 1.57			&1.45  \\
\hline
-funroll-all-loops	 		& 1.57			& 1.45 \\
\hline
Todos	 					& 1.57			& 1.45 \\
\hline
\end{tabular}
\end{center}
\caption{Comparación de compilación con distintos niveles de optimización}
\end{table}
\newpage
		\subsection{Resultados de la ejecución del Benchmark Drystone}

Esto le da un valor deDrystone 41288895/25MHz = 1.65/MHz. Se trato de de mejorar este valor mediante la adición de algunas opciones del compilador.

Aquí están los resultados de tratar de optimizar la fase de compilación. Al ser compilado con el nivel de optimización -O2 se observaron los resultados mostrados en el bloque ~\ref{lst:salidasdry} 

\begin{lstlisting}[frame=single,caption={Sin optimizaciones },label={lst:salidas},breaklines]

Execution starts, 1000000 runs through Dhrystone
Timer ticks, 100/s., (7397 - 0) =	7397
Number of Runs 1000000
Elapsed time 73.97s
Processor at 50 MHz
Microseconds for one run through Dhrystone: ( 73970000 uS / 1000k ) = 73 uS
Dhrystones per Second:                      13698 
\end{lstlisting}

\begin{lstlisting}[frame=single,caption={Optimización nivel -O2},label={lst:salidas},breaklines]
Execution starts, 1000000 runs through Dhrystone
Timer ticks, 100/s., (7398 - 0) =	7398
Number of Runs 1000000
Elapsed time 73.98s
Processor at 50 MHz
Microseconds for one run through Dhrystone: ( 73980000 uS / 1000k ) = 73 uS
Dhrystones per Second:                      13698 
\end{lstlisting}

\begin{lstlisting}[frame=single,caption={Optimización nivel -O2},label={lst:salidas},breaklines]
Execution starts, 500000 runs through Dhrystone
Timer ticks, 100/s., (3699 - 0) =	3699
Number of Runs 500000
Elapsed time 36.99s
Processor at 50 MHz
Microseconds for one run through Dhrystone: ( 36990000 uS / 500k ) = 73 uS
Dhrystones per Second:                      13888 
\end{lstlisting}

\begin{lstlisting}[frame=single,caption={Optimización nivel -O3},label={lst:salidas},breaklines]
Execution starts, 500000 runs through Dhrystone
Timer ticks, 100/s., (2738 - 0) =	2738
Number of Runs 500000
Elapsed time 27.38s
Processor at 50 MHz
Microseconds for one run through Dhrystone: ( 27380000 uS / 500k ) = 54 uS
Dhrystones per Second:                      18518 
\end{lstlisting}


	\subsection {Presentación de los resultados de optimización} 
\begin{table}[h!]
\begin{center}
\begin{tabular}{ |l |l| l|}
\hline
\rowcolor[gray]{0.8} Opciones del compilador&-O2&-O3 \\
\hline
Sin extras 					&1.65 			&1.54\\
\hline
-mhard-div -mhard-mu 		& 1.57			&1.45 \\
\hline
-funroll-loops			 	& 1.57			& 1.45 \\
\hline
-fgcse-sm					& 1.57			& 1.45 \\
\hline
-msoft-float 				& 1.57			&1.45  \\
\hline
-funroll-all-loops	 		& 1.57			& 1.45 \\
\hline
Todos	 					& 1.57			& 1.45 \\
\hline
\end{tabular}
\end{center}
\caption{Comparación de compilación con distintos niveles de optimización}
\end{table}

	\section{Estudio de factibilidad para la implentación de RTOS Embebidos}
	Durante las pruebas realizadas en el prototipo ~\ref{chap:proto4}	

	
		\subsection{Resultados del programa de prueba de hilos}
		
			
		
	
	\section{Estudio de factibilidad para la implentación Linux}
	
	
	 