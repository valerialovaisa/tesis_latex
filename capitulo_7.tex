%%%%%%%%%%%ESTUDIO DEL PROBLEMA%%%%%%%%%%%%%%%%
\chapter{Estudio del Problema}
	\section{Introducción}
	Como primera acción se analizó la factibilidad de implementación de un SoC con licencia OpenSource en FPGA, razón que condujo a la investigación del
	tópico en búsqueda de información necesaria que determine si existe tal factibilidad. Las características relevadas durante la investigación
	aportaron información respecto al hardware, las herramientas de desarrollo y el sistema operativo necesario para llevar a cabo la implementación. 
	
    Se debió discernir entre las diversas alternativas de hardware que se tenían disponibles al momento del desarrollo de este trabajo. Esta situación
    derivó en una valoración de los diferentes entornos de desarrollo y sistemas operativos asociados que cumplan con el paradigma del software
    libre.
	
	\section{Requerimientos del Usuario}
	Se presenta en este apartado un listado de los requerimientos elecitados que tienen como objetivo comprender el dominio del problema y permiten
	trabajar en la realización de una solución eficiente.
	
		\subsection{En cuanto al Hardware}
		\begin{tabular}{ p{2.5cm} p{8cm} p{3cm} }
		\hline 
		\rowcolor[gray]{0.8} N\textordmasculine Req & Descripción & Tipo\\
		\hline
		RQX-HW 1 &  Se debe implementar un Microprocesador Softcore de núcleo simple & Cantidad y tipo de núcleos. \\
		\hline
		RQX-HW 2 &  El SoC seleccionado debe poseer la menor cantidad de restricciones respecto de su implementación en placas de desarrollo de diversos
		fabricantes. & Portabilidad a nivel Hardware\\
		\hline
		RQX-HW 3 & La placa de desarrollo elegida debe poseer al menos 32 MB de memoria RAM disponible que permita al ejecución del kernel de linux. &
		Memoria Disponible\\
		\hline
		RQX-HW 4 & La placa de desarrollo elegida debe poseer al menos 8 MB de memoria flash disponible que permita guardar la configuración de la FPGA y un
		bootloader & Memoria Disponible\\
		\hline
		\end{tabular}
					
		\subsection{En cuanto a las Licencias}
		\begin{tabular}{ p{2.5cm} p{8cm} p{3cm} }
		\hline 
		\rowcolor[gray]{0.8} N\textordmasculine Req & Descripción  & Tipo\\
		\hline 
		RQX-LC 1 &  Todo el hardware implementado debe tener licencia LGPL o GPLv2 en su defecto & Licencias de hardware\\
		\hline 
		RQX-LC 2 &  Las herramientas de desarrollo utilizadas deben poseer licencias LGPL o GPLv2 en su defecto & Licencias de Software\\
		\hline
		RQX-LC 3 & El sistema operativo elegido debe poseer licencia LGPL o GPLv2 & Licencias de Software\\
		\hline
		\end{tabular}
			
		\subsection{En cuanto a las Herramientas de Desarrollo}
		\begin{tabular}{ p{2.5cm} p{8cm} p{3cm} }
		\hline 
		\rowcolor[gray]{0.8} N\textordmasculine Req & Descripción  & Tipo\\
		\hline 
		RQX-HD 1 &  Las herramientas de desarrollo elegidas en base a la arquitectura a implementar deben tener la menor cantidad de restricciones respecto
		del sistema operativo donde serán ejecutadas & Portabilidad\\
		\hline 
		RQX-HD 2 &  Las herramientas de desarrollo deben brindar la capacidad de desarrollar y depurar programas para la arquiectura seleccionada &
		XXXXXXXXXXXX\\ % Revisar tipo de requerimiento
		\hline
		RQX-HD 3 &  Las herramientas de desarrollo de la placa seleccionada deben proveer soporte para el acceso a sus periféricos on board &
		XXXXXXXXXXXX\\ % Revisar tipo de requerimiento
		\hline 
		\end{tabular}
		
		\subsection{En cuanto Sistema Operativo} 	 
		\begin{tabular}{ p{2.5cm} p{8cm} p{3cm} }
		\hline 
		\rowcolor[gray]{0.8} N\textordmasculine Req & Descripción  & Tipo\\
		\hline 
		RQX-SO 1 &  El sistema operativo elegido debe disponer de drivers y librerías que permitan el acceso a todos los dispositivos incluídos en el SoC &
		XXXXXXXXXXXXX\\ % Revisar tipo de requerimiento
		\hline 
		RQX-SO 2 &  El sistema operativo elegido debe tener la capacidad de ejecución multitarea e hilos & XXXXXXXXXXXX\\ % Revisar tipo de
		% requerimiento
		\hline
		RQX-SO 3 &  El sistema operativo elegido debe posibilitar la ejecución de aplicaciones de tiempo real & XXXXXXXXXXXX\\ % Revisar tipo de
		% requerimiento
		\hline 
		\end{tabular}
		
	
	\section{Estudio de componentes y viabilidad para el proyecto}
		
			\subsection{Objetivo}
			Se estudiaron los componentes del proyecto y sus diversas alternativas de implementación por medio de un análisis comparativo que permitió evidenciar
			las características relevantes de cada uno de ellas. Inicialmente se realizó una comparativa de las prestaciones de los microprocesadores softcore
			más importantes para evaluar su capacidad de procesamiento 
			
			\subsection{Selección del Microprocesador Soft-Core}
			
				\subsubsection{Comparación de Microprocesadores Soft-Core} 
	
				\subsubsection{Conclusiones de la elección del micro Soft-Core}
 			
 			\subsection{Selección de la Placa de Desarrollo}
 				\subsubsection{Análisis de las alternativas} 
				Los proyectos MinSoC y OrpSoc cuentan actualmente con soporte para diversas placas de desarrollo. 
				\paragraph{Xilinx}
				La empresa Xilinx provee kits de desarrollo de diversas características y prestaciones. A continuación se detallan algunas de las alternativas que
				son soportadas por los SoC elegidos.
				\subparagraph{S3ADSP1800A}
				El dispositivo XtremeDSP™ Starter Platform cuenta con una FPGA de la familia Spartan®-3A que permite la evaluación diseños para diferentes
				apliaciones tales como Prototipado General, Sistemas Embebidos, Video Digital, DSP, Procesamiento de Imagenes, Comunicaciones digitales y
				Coprocesamiento. Esta plataforma provee acceso a las capacidades de la familia de FPGA Spartan®-3A y cuenta con periféricos,conectores e
				interfaces estándar de la industria. Fue diseñada para para ser utilizada con Xilinx System Generator para aplicaciones DSP y las herramientas de
				diseño ISE® provistas por el fabricante. 
				
				%Traducir esto ???
				Las características generales del kit son:
				
				\begin{tabular}{ p{4cm} p{10cm} }
				\rowcolor[gray]{0.8} Caracteristica & Descripción \\		
				\hline FPGA   & XC3SD1800A-4FGG676C Spartan-3A DSP FPGA\\
				\hline Clocks & 125 MHz LVTTL SMT oscillator\\
				\hline        & LVTTL oscillator socket\\
				\hline		  & 25.175 MHz LVTTL SMT oscillator (video clock)\\
				\hline		  & 25 MHz Ethernet clock (accessible to FPGA)\\
				\hline Memory & 128 MB (32M x 32) DDR2 SDRAM\\
				\hline		  & 16Mx8 parallel / BPI configuration flash\\
				\hline 		  & 64 Mb SPI configuration / storage flash (with 4 extra SPI selects)\\
				\hline Interfaces & 10/100/1000 PHY\\
				\hline			  & JTAG programming/configuration port\\
				\hline            & RS232 Port\\
				\hline			  & Low-cost VGA\\
				\hline			  & 4 SPI select lines\\
				\hline Buttons and Switches & 8 user LEDs\\
				\hline  		  & 8-position user DIP switch\\
				\hline            & 4 user push button switches\\
				\hline 			  & Reset push button switch\\
				\hline User I/O and Expansion & Digilent 6-pin header\\
				\hline			 			  & EXP expansion connector\\
				\hline 						  & 30-pin GPIO connector: can be used for System ACE™ Compact Flash daughter card (not included)\\
				\hline Configuration and Debug & JTAG\\
				\hline                         & System ACE module connector\\   
				\end{tabular}
				
				Las implementaciones MinSoC y ORPSoC proveen soporte nativo para los siguientes perisféricos:
				\begin{itemize}
				  \item Ethernet
				  \item GPIO (Solo ORPSoC)
				  \item DDR2 SDRAM (128MB) (Solo ORPSoC)
				  \item SPI
				  \item UART				
				\end{itemize}
				
				\paragraph{Digilent}
				
				 	 
				\paragraph{Altera}
				
				
 				\subsubsection{Conclusiones de la elección de la Placa de Desarrollo}
 			
 			\subsection{Selección de las herramientas de desarrollo} 	 
 		
 		
 			\subsection{Selección del Sistema Operativo}
 			