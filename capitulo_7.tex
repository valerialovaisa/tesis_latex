%%%%%%%%%%%ESTUDIO DEL PROBLEMA%%%%%%%%%%%%%%%%
\chapter{Estudio del Problema}
	\section{Introducción}
	Como primera acción se analizó la factibilidad de implementación de un
	SoC con licencia OpenSource en FPGA, razón que condujo a la investigación del
	tópico en búsqueda de información necesaria que determine si existe
	tal factibilidad. Las características relevadas durante la investigación
	aportaron información respecto al hardware, las herramientas de desarrollo y
	el sistema operativo necesario para llevar a cabo la implementación. 
	
	Se debió discernir entre las diversas alternativas de hardware que se tenían
	disponibles al momento del desarrollo de este trabajo. Esta situación derivó en
	una valoración de los diferentes entornos de desarrollo y sistemas operativos 
	asociados que cumplan con el paradigma del software libre. 
	
	\section{Requerimientos del Usuario}
	Se realizó en este apartado un listado de los requerimientos elecitados que
	tienen como objetivo comprender el dominio del problema para trabajar en una
	solición eficiente del problema.
		\subsection{En cuanto al Hardware}
		A continuación se listaron los requerimientos de hardware a utilizar para la
		implementación del sistema.
		
		\begin{tabular}{||c|c|c||}
		\hline
		N° Req & Descripción  & Tipo\\
		\hline
		1 &  Se debe implementar un Microprocesador Softcore de núcleo simple &
		Cantidad y tipo de núcleos. \\
		2 &  El SoC seleccionado debe poseer la menor cantidad de restricciones
		respecto de su implementación en placas de desarrollo de diversos fabricantes.
		& Portabilidad a nivel Hardware\\
		3 & La placa de desarrollo elegida debe poseer al menos 32 MB de memoria RAM
		disponible que permita al ejecución del kernel de linux. & Memoria Disponible
		\\
		
		\hline
		\end{tabular}
					
	
			\subsection{En cuanto a las Licencias}
			\subsection{En cuanto a las Herramientas de Desarrollo}
			\subsection{En cuanto Sistema Operativo} 	 
	\section{Estudio de componentes y de la viabilidad para el proyecto}	
			\subsection{Objetivo} 	 
			\subsection{Comparación de los Soft-Core} 
	\section{Conclusiones de la elección del micro
	 	Soft-Core}
 			\subsection{Placas de Desarrollo}
				\subsubsection{Xilinx}
				\subsubsection{Digilent} 	 
				\subsubsection{Altera}
 			\subsection{CONCLUSIONES DE LA ELECCIÓN DE LA PLACA DE
 		DESARROLLO}
 			\subsection{SELECCIÓN DE LAS HERRAMIENTAS DE DESARROLLO} 	 
 			\subsection{Elección de Sistema Operativo}
 			