%%%%%%%%%%%%%%%%%%%%%%%%%%%%%%%%%%%%%%%%% CAPITULO 3 %%%%%%%%%%%%%%%%%%%%%%%%%%%%%%%%%
\chapter{Benchmarking}
	\section{Introducción}
	En informática benchmark representa la acción de ejecutar un programa de computadora, un conjunto de programas, u otras aplicaciones con el fin
	de evaluar el rendimiento relativo de un objeto, normalmente mediante la ejecución de una serie de pruebas estándar.  El Benchmarking se asocia
	generalmente con la evaluación de las características de rendimiento del hardware, por ejemplo, el rendimiento de operaciones de punto flotante de la
	CPU pero hay circunstancias en que la técnica también es aplicable al software. Los benchmarks de software son ejecutados para evaluar el
	rendimiento de compiladores o sistemas de gestión de bases de datos .
	
	Los Benchmarks proveen un método de comparación de las prestaciones de varios subsistemas que conforman la arquitectura de un sistema. Debido a que
	las arquitecturas de computadoras se han vuelto cada vez más complejas, resulta difícil comparar las prestaciones de varios sistemas de computadoras
	con simplemente evaluar sus especificaciones. Es necesario entonces la utilización de tests que permitan la comparación de diferentes arquitecturas.
	Por ej., los procesadores Pentium 4 de Intel generalmente operaban a mayor frecuencia que los Athlon XP de AMD, lo que no necesariamente debía
	interpretarse como una mayor capacidad de procesamiento.
	
	Son diseñados para imitar una clase particular de carga de trabajo en un componente del sistema. Los benchmarks sintéticos realizan esto mediante
	programas especialmente creados para dirigir la carga de trabajo en un compenente específico. Los benchmarks de aplicaciones ejecutan programas
	de uso cotidiano de los usuarios. Mientras los benchmarks de aplicaciones generalmente proporcionan una mejor medición de la performace del sistema
	respecto de aplicaciones cotidianas, los benchmarks sintéticos son más efectivos para probar compenentes individuales como discos rígidos o
	dispositivos de red.

	Los benchmarks tienen gran importancia en el diseño de CPUs, proveyendo a los diseñadores la habilidad de medir y realizar compartivas duranta las
	decisiones respecto a la microarquitectura. Hasta el año 2000 utilizaban el benchmark de Standard Performance Evaluation Corporation (SPEC) aunque
	las versiones SPEC para sistemas operativos Unix eran aún de gran tamaño y difíciles de usar sin modificar.
	
	Es sabido que algunos fabricantes de computadoras modificaron sus sistemas para mostrar altos indicadores de performace luego de la ejecución de los
	tests benchmark que no eran reflejados luego durante la ejecucción real de aplicaciones. Durante los 80' algunos compiladores detectaban una
	operación matematica específica utilizada en los benchmarks de punto flotante mas conocidos y reemplazaban la misma con una operacion matemática
	equivalente de ejecución más rápida. Sin embargo, esta transformación rara vez era útil en la ejecución de programas fuera del benchmark hasta que a
	mediados de los 90', cuando las arquitecturas Reduced instruction set computing (RISC) y Very long instruction word (VLIW) enfatizaron la importancia
	de los compiladores en lo relativo a la performace. Actualmente, los benchmarks son utilizados por los desarrolladores de compiladores para
	incrementar no solo sus indicadores de benchmark sino también su performance en aplicaciones reales. Las CPUs que poseen mas unidades de ejecución
	usualmente completan tareas reales y de benchmark en menos tiempo que los supuestos más rápidos, procesadores de mayores frecuencias de clock.
	
	\section{Tipos de Benchmarks}
	Como se ha mencionado en el apartado anterior existen diferentes tipo de benchmarks que permiten obtener un valor cuantitativo de la performace de un
	componente en diferentes circunstancias de ejecución.
	
		\begin{tabular}{ p{2.5cm} p{8cm} p{3cm} }
		\hline 
		\rowcolor[gray]{0.8} Tipo & Descripción & Ejemplo \\
		\hline
		De programas reales  &  Diseñados para evaluar el rendimiento de el procesador en aplicaciones de uso cotidiano: procesadores de texto
		como \LaTeX, software de diseño asistido por computadora (CAD) y aplicaciones generales de usuario \& SPECint y SPECfp\\
		\hline
		Microbenchmark  &  Diseñados para medir la performance de una pequeña y específica porción de código \\
		\hline
		Kernel			&  Se basan en el análisis y conocimiento de que en la mayoría de los casos solo el 10 \% del código ejecutado utiliza el 80\% de los
		recursos de CPU utilizando estas porciones de código clave para realizar el benchmark.\cite{EtiquetaBM01} & linpack y livermore loops\\
		\hline
		Benchmark de componentes & Programas diseñados para medir la performance de componentes básicos. Detectan automáticamente parametros de hardware
		como el numero de registros, tamaño de cache, latencia de memoria,etc. &  SPEC CPU2006\\
		\hline
		
		\end{tabular}
	
	Component Benchmark/ micro-benchmark
		programs designed to measure performance of a computer's basic components [3]
		automatic detection of computer's hardware parameters like number of registers, cache size, memory latency
	Synthetic Benchmark
		Procedure for programming synthetic benchmark:
			take statistics of all types of operations from many application programs
			get proportion of each operation
			write program based on the proportion above
	
		Types of Synthetic Benchmark are:
			Whetstone
			Dhrystone
		These were the first general purpose industry standard computer benchmarks. They do not necessarily obtain high scores on modern pipelinedcomputers.
	I/O benchmarks
	
	Database benchmarks: to measure the throughput and response times of database management systems (DBMS')
	Parallel benchmarks: used on machines with multiple cores, processors or systems consisting of multiple machines
	
	\section{Benchmarking de Sistemas Embebidos}
	The Embedded Microprocessor Benchmark Consortium (EEMBC) develops benchmark software to help system designers select the optimal processors, and
	benchmark tools to help consumers and IT professionals select the appropriate smart phones/tablets and networking firewall appliances. EEMBC
	organizes its benchmark suites targeting Automotive, Digital Media, Java, Multicore Processors, Networking, Office Automation, Signal Processing,
	Smartphones/Tablets and Browsers.
 	
	\section{CPU core benchmarking}
 	A pesar de que no se corresponde con la forma en que utilizaría un procesador en una aplicación real , a veces es importante aislar el núcleo de la
 	CPU de los otros elementos del procesador y centrarse en un elemento clave. Por ejemplo , es posible que desee tener la capacidad de hacer caso
 	omiso de la memoria y los efectos de E / S y se centran principalmente en la operación de el pipeline. Este es el dominio de CoreMark . CoreMark es
 	capaz de probar la estructura de pipeline básica de un procesador , así como la capacidad de prueba de lectura / escritura de operaciones básicas ,
 	operaciones de enteros y operaciones de control

	\section{CoreMark}

	CoreMark es un punto de referencia que tiene como objetivo medir el rendimiento de las unidades centrales de procesamiento ( CPU) utilizados en
	sistemas embebidos. Fue desarrollado en 2009 por Shay Gal -On en EEMBC y está destinado a convertirse en un estándar de la industria , en sustitución
	de la referencia Dhrystone anticuada . El código está escrito en código C y contiene las implementaciones de los algoritmos siguientes :
	procesamiento de lista ( encontrar y ordenar ) , Matrix (matemáticas) manipulación ( operaciones con matrices comunes ) , máquina de estados (
	determinar si un flujo de entrada contiene números válidos ) y CRC
