%%%%%%%%%%%%%%%%%%%%%%%%%%%%%%%%%%%%%%%%% CAPITULO 3 %%%%%%%%%%%%%%%%%%%%%%%%%%%%%%%%%
\chapter{Benchmarking}

\section{Introducción}

En informática, \textit{benchmark} es la acción de ejecutar un
programa de computadora o un conjunto de programas con el fin de
evaluar el rendimiento. Dicho rendimiento se refiere generalmente al
hardware, por ejemplo, el rendimiento de operaciones de punto flotante
de la CPU. Sin embargo, hay circunstancias en que la técnica también
es aplicable al software. Por ejemplo, existen benchmarks de software
que son ejecutados para evaluar el rendimiento de compiladores o
sistemas de gestión de bases de datos.

El uso de \textit{benchmark} está motivado por la creciente
complejidad en los sistemas informáticos, tanto a nivel de hardware
como de software, lo cual dificulta enormemente su comparación. En
particular, las especificaciones técnicas del sistema por sí solas no
permiten predecir el rendimiento del mismo para resolver un problema
en particular. Es necesario utilizar tests que permitan su
comparación. Por ejemplo, los procesadores Phenom II de AMD
generalmente operan a mayor frecuencia que los i7 de Intel, lo
que no necesariamente implica una mayor capacidad de procesamiento. En
particular, las CPUs que poseen mas unidades de ejecución usualmente
completan tareas reales y de benchmark en menos tiempo que los
supuestos más rápidos, procesadores de mayores frecuencias de clock.

Entre la amplia variedad de \textit{benchmarks} existentes podemos
diferenciar dos grupos principales. El primer grupo lo conforman los
\textit{benchmarks} de aplicaciones los cuales se encargan de analizar
el rendimiento cuando se ejecutan programas de uso cotidiano para el
usuario. El segundo grupo lo conforman los \textit{benchmarks}
sintéticos los cuales están diseñados para imitar una clase particular
de carga de trabajo en un componente del sistema. Estos últimos son
más efectivos para probar componentes individuales como discos rígidos
o dispositivos de red.
	
Los \textit{benchmarks} tienen gran importancia en el diseño de
CPUs. Proveen a los diseñadores la habilidad de medir y realizar
comparativas durante las etapas de diseño permitiendo tomar mejores
decisiones a la hora de optimizar la micro-arquitectura. Por ejemplo,
una de las baterías de tests utilizados a tal fin es la provista por
la Standard Performance Evaluation Corporation (SPEC). Tales tests se
convirtieron en un patrón de referencia entre los diferentes
fabricantes de computadoras. Por tal motivo, ocurrió que varios de
ellos modificaron sus sistemas con el único fin de mejorar los
resultados frente a dichos tests sin que tales modificaciones
impliquen una mejora del desempeño en aplicaciones reales. Por
ejemplo, durante los 80', algunos compiladores detectaban una
operación matemática específica utilizada en los \textit{benchmarks}
de punto flotante mas conocidos y reemplazaban la misma con una
operación matemática equivalente de ejecución más rápida. Sin embargo,
esta transformación rara vez era útil en la ejecución de programas
fuera del \textit{benchmark}. Luego, a mediados de los 90', con la
introducción de las arquitecturas Reduced Instruction Set Computing
(RISC) y Very Long Instruction Word (VLIW) se enfatizó la importancia
de los compiladores en lo relativo a desempeño. Actualmente, los
\textit{benchmarks} son utilizados también por los desarrolladores de
compiladores para incrementar no sólo sus indicadores de
\textit{benchmark} sino también su desempeño en aplicaciones
reales. 



\section{Tipos de Benchmarks}

Además del la diferenciación general entre \textit{benchmarks} de
hardware y software, existe una especificación más clasificación que se
describe brevemente en la siguiente tabla:

\begin{tabular}{ p{2.5cm} p{8cm} p{3cm} }
\hline 
\rowcolor[gray]{0.8} Tipo & Descripción & Ejemplo \\
\hline
De programas reales & Diseñados para evaluar el
rendimiento de el procesador en aplicaciones de uso
cotidiano. Por ejemplo para procesadores de texto como \LaTeX, software
de diseño asistido por computadora (CAD) y
aplicaciones generales de usuario & SPECint y SPECfp\\
\hline
Microbenchmark & Diseñados para medir la performance
de una pequeña y específica porción de código \\
\hline
Kernel & Se basan en el análisis y conocimiento de
que en la mayoría de los casos solo el 10 \% del
código ejecutado utiliza el 80\% de los recursos de
CPU utilizando estas porciones de código clave para
realizar el benchmark & linpack y
livermore loops\\
\hline
Benchmark de componentes & Programas diseñados para
medir la performance de componentes básicos. Detectan
automáticamente parámetros de hardware como el numero
de registros, tamaño de cache, latencia de
memoria,etc. & SPEC CPU2006\\
\hline
Benchmarks Sintéticos & Diseñados para medir la
performance de una pequeña y específica porción de
código & Whetstone y Dhrystone\\
\hline
Benchmarks I/O & Utilizados para medir la performace
de los dispositivos de entrada y salida & \\
\hline
Benchmark de Base de Datos & Pretenden medir el
throughtput y los tiempos de respuesta de los sistemas
de gestión de base de datos & Transaction Processing
Performance Council (TPC)\\
\hline
Benchmarks Paralelos & Diseñados para ser utilizados
en procesadores multicore o sistemas distribuidos &
NAS Parallel Benchmarks (NPB)\\
\hline
\end{tabular}
	
\section{Benchmarking de Sistemas Embebidos}

La utilización de benchmarking se extiende a los Sistemas Embebidos
proporcionando datos que permiten la comparación de los mismos. Por
ejemplo, el Embedded Microprocessor Benchmark Consortium (EEMBC)
desarrolla una serie de aplicaciones benchmark que facilitan a los
diseñadores de sistemas la selección de los procesadores óptimos para
sus desarrollos. EEMBC organiza su conjunto de benchmarks apuntando a
diferentes ramas de aplicación tales como Automotriz, Sistemas
Digitales, Java, Procesadores Multicore, Redes, Procesamiento de
señales, Smartphones/tablets y Browsers.
	
En benchmarking de sistema embebidos es común realizar tests que se
focalicen en el núcleo de la CPU aislándolo de los otros elementos del
procesador. Es posible, por ejemplo, que desee tener la capacidad de
hacer caso omiso de la memoria y los efectos de E/S con el fin de
focalizar el análisis principalmente en la operación de un pipeline.

A continuación se describen dos de los principales benchmarks que
fueron utilizados para evaluar el sistema desarrollado en la presente
tesis (véase el capítulo de resultados).

\subsection{CoreMark}	

CoreMark es un Benchmark que pretende medir el desempeño de las CPU
utilizadas en Sistemas Embebidos. Fue desarrollado en 2009 por Shay
Gal-On en el EEMBC y se ha convertido un estándar de facto de la
industria llegando a reemplazar al antiguo benchmark
Dhrystone. CoreMark esta desarrollado en C y contiene implementaciones
de los siguientes algoritmos:
	
\begin{itemize}
\item Procesamiento de listas (Find and sort)
\item Manipulación de matrices (Operaciones comunes de matrices)
\item Máquina de estados 
\item Códigos de redundancia cíclica (CRC)  
\end{itemize}	 
	
Las actuales plataformas con procesadores embebidos no solo están
compuestas por un CPU y un subsistema de memoria. De hecho, los SoC
pueden estar compuestos por más de un núcleo de CPU, aceleradores de
hardware, periféricos complejos y grandes cantidades de
memoria. CoreMark es el punto de partida para la evaluación de éstos
complejos dispositivos. El benchmark es capaz de probar la estructura
de pipeline básica de un procesador, así como la capacidad de prueba
de lectura/escritura de operaciones básicas, operaciones de enteros y
operaciones de control. Respecto de los periféricos y mecanismos de
interrupción, el EEMBC desarrolla una colección de tests basados en
escenarios y conducidos por hardware que evalúan dichos periféricos
y la capacidad del CPU de prestar servicio a los mismos y a otras
interrupciones a nivel sistema.
	
Los resultados del CoreMark pueden ser encontrados en la página principal   
del proyecto \url{http://www.eembc.org/coremark/} y se presentan en el siguiente formato:
	
CoreMark 1.0 : N / C [/ P] [/ M]
	
Donde:
\begin{itemize}
\item N: Numero de iteraciones por segundo 
\item C: Versión del compilador y flags
\item P: Parámetros específicos de memoria y datos
\item M: Tipo de ejecución de algoritmo paralelo y el numero de contextos.
\end{itemize}	 
	
Por ejemplo: CoreMark 1.0 : 128 / GCC 4.1.2 -O2 -fprofile-use / Heap
in TCRAM / FORK:2
	
\subsection{Dhrystone}

 		Es un benchmark sintético y comercial. Intenta medir la velocidad del sistema en cuanto a rendimiento no numérico, expresando los
		resultados en DPS (instrucciones Dhrystones Por Segundo). El rendimiento de Dhyrstone se calcula a partir de la siguiente fórmula: Iteraciones
		Dhrystone por segundo = reloj del procesador * número de pasadas / tiempo de ejecución. Para que el resultado sea válido el código Dhrystone debe
		ser ejecutado al menos por dos segundos. Es un pequeño benchmark sintético que pretende ser representativo de programación entera de
		sistemas. 
		
		Contiene muchas instrucciones simples, llamadas a procedimiento y condicionales, y pocas de coma flotante y bucles. Usa pocas variables globales y
		ejecuta operaciones con punteros. Está compuesto por 12 procedimientos incluidos en un bucle de medida con 94 sentencias. No se puede variar su
		tamaño. Está compuesto por un 53\% de instrucciones de asignación, 32\% de instrucciones de control y un 15\% de llamadas a procedimiento.
		Dhrystone es compacto (no más de 1,5 KB), ampliamente disponible en el dominio público, y sencillo de ejecutar. Por ser tan pequeño el Dhrystone
		entra completamente en la caché interna, de esta manera no mide el resto del sistema pero presenta la ventaja de que mide solamente la capacidad
		del procesador para trabajar con enteros.
		
%{\color{red}\textbf{agregar breve descripción acá.}}

