\chapter{Diseño e Implementación}
	\section{Introducción}
	Debido a que se adoptó un modelo de desarrollo en espiral se plantearon una
	serie de prototipos que incluyen mayor funcionalidad en cada iteración del
	modelo. Inicialmente se planteó un prototipo de básico para determinar
	factibilidad de puesta en funcionamiento de un SoC con microprocesador
	OpenRISC y el desarrollo de aplicaciones que se ejecuten en él.
	Debido a su funcionalidad reducida y menor consumo de recursos para su
	síntesiss se seleccionó el proyecto MinSoC para el desarrollo del primer
	prototipo con el cual se evaluaron las diferentes herramientas
	necesarias para que el sistema se encuentre funcional.
	
%DUDO QUE EN NUESTRO CASO ESTO PUEDA IR ACA ----------- PABLO!	
	\section{Descripción de la Arquitectura}

	 
	\section{Criterio para la realización de testing}
	\section{Entorno de ejecución}
		\section{Entorno de ejecución Standalone}
		\section{Entorno de ejecución linux}
	
	\section{PROTOTIPO UNO: Implementación del SoC MinSoc en FPGA}
		\section{Introducción}
		\section{Requerimientos del prototipo}
		\section{Implementación}
		\section{Diagrama de Secuencia}
		\section{Testing}
		\section{Conclusión}
	\section{PROTOTIPO DOS: Implementación del SoC OrpSoc en FPGA}
		\section{Introducción}
		\section{Requerimientos del prototipo}
		\section{Implementación}
		\section{Diagrama de Secuencia}
		\section{Testing}
		\section{Conclusión}
	\section{PROTOTIPO TRES: Implementación del SoC OrpSoc en FPGA con Sistema Operativo eCos}
		\section{Introducción}
		\section{Requerimientos del prototipo}
		\section{Implementación}
		\section{Diagrama de Secuencia}
		\section{Testing}
		\section{Conclusión}
	\section{PROTOTIPO tres: Implementación OrpSoc en FPGA con Linux}
		\section{Introducción}
		\section{Requerimientos del prototipo}
		\section{Implementación}
		\section{Diagrama de Secuencia}
		\section{Testing}
		\section{Conclusión}
