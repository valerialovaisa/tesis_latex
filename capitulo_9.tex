\chapter{Diseño e Implementación}
	\section{Introducción}
	Debido a que se adoptó un modelo de desarrollo en espiral se plantearon una serie de prototipos que incluyen mayor funcionalidad en cada iteración
	del modelo. Inicialmente se planteó un prototipo de básico para determinar factibilidad de puesta en funcionamiento de un SoC con microprocesador
	OpenRISC y el desarrollo de aplicaciones que se ejecuten en él. Debido a su funcionalidad reducida y menor consumo de recursos para su
	síntesis se seleccionó el proyecto MinSoC para el desarrollo del primer prototipo con el cual se evaluaron y documentaron las capacidades del mismo
	permitiendo valorar este proyecto para aplicaciones donde no se requieren grandes 
		 
	\section{Criterio para la realización de testing}
	\section{Entorno de ejecución}
		\section{Entorno de ejecución Standalone}
		\section{Entorno de ejecución linux}
	
	\section{PROTOTIPO UNO: Implementación del SoC MinSoC en FPGA}
		\section{Introducción}
		En este primer prototipo se implementó el proyecto MinSoC en la placa de desarrollo S3ADSP1800A con el fin de verificar el funcionamiento
		del procesador y sus periféricos. Para lograr este objetivo se requiere de la instalación y puesta en funcionamiento de las herramientas
		de síntensis, place \& route(PAR) y programación de la FPGA Spartan 3A. 
		
		\section{Requerimientos del prototipo}
		\begin{tabular}{ p{2.5cm} p{8cm} p{3cm} }
		\hline 
		\rowcolor[gray]{0.8} N\textordmasculine Req & Descripción\\
		\hline 
		RQX-PA 1 & El prototipo debe implementar un SoC MinSoC en la placa desarrollo S3ADSP1800A\\ 
		\hline 
		RQX-PA 2 & El prototipo debe garantizar el correcto funcionamiento de todos los periféricos conectados al SoC mediante el bus Wishbone\\ 
		\hline 
		RQX-PA 3 & El prototipo debe interactuar correctamente con las interfaces hardware de la placa de desarrollo S3ADSP1800A\\ 
		\hline
		RQX-PA 4 & El prototipo debe se capaz de ejecutar programas con capacidad de depuración generados mediante compilación
		cruzada en una arquitectura x86 y/o x86\_64 para arquitectura la OpenRISC\\
		\hline
		RQX-PA 5 & Se debe lograr depurar paso a paso y mediante break points las aplicaciones desarrolladas\\
		\hline
		RQX-PA 6 & El prototipo debe ser evaluado mediante el benchmark CoreMark para el posterior análisis de las capacidades del SoC\\
		\hline		
		\end{tabular}
		
		\section{Descripción de la Arquitectura}
		
			
		\section{Implementación}
		
		\section{Diseño de aplicaciones de prueba}
		
		\section{Diagrama de Secuencia}
		
		\section{Testing}
		
		\section{Conclusión}
		Se pudo implementar con éxito el prototipo número uno cuyos resultados muestran que es de vital importancia 
			
	\section{PROTOTIPO DOS: Implementación del SoC ORPSoc en FPGA}
		\section{Introducción}
		Para el prototipo dos se planteó la implementación de un SoC ORPSoC 
		
		\section{Requerimientos del prototipo}

		\section{Implementación}

		\section{Diagrama de Secuencia}

		\section{Testing}

		\section{Conclusión}

	\section{PROTOTIPO TRES: Implementación del SoC OrpSoc en FPGA con Sistema Operativo eCos}
		\section{Introducción}
		\section{Requerimientos del prototipo}
		\section{Implementación}
		\section{Diagrama de Secuencia}
		\section{Testing}
		\section{Conclusión}
	\section{PROTOTIPO tres: Implementación OrpSoc en FPGA con Linux}
		\section{Introducción}
		\section{Requerimientos del prototipo}
		\section{Implementación}
		\section{Diagrama de Secuencia}
		\section{Testing}
		\section{Conclusión}
