\chapter{Implementación y Diseño}
	\section{Introducción}
	Debido a que se adoptó un modelo de desarrollo en espiral se plantearon una serie de prototipos que incluyen mayor funcionalidad en cada iteración
	del modelo. Inicialmente se planteó un prototipo de básico para determinar factibilidad de puesta en funcionamiento de un SoC con microprocesador
	OpenRISC y el desarrollo de aplicaciones que se ejecuten en él. Debido a su funcionalidad reducida y menor consumo de recursos  para su síntesis se
	seleccionó el proyecto MinSoC para el desarrollo del primer prototipo con el cual se evaluaron y documentaron las capacidades del mismo permitiendo
	valorar este proyecto para aplicaciones donde no se requieren grandes cantidades de memoria y juegan un papel primordial las entradas/salidas junto
	a la capacidad de procesamiento.  
		 
		\subsection{Entorno de ejecución}
		Actualmente OpenRISC es soportado por un conjunto de herramientas de desarrollo(toolchain) de 32 bits ofreciendo soporte para los lenguajes C y C++
		con librerías estáticas. El toolchain se encuentra disponible en dos formas: una para la ejecución de aplicaciones bajo el sistema operativo Linux y
		otra para la ejecución de aplicaciones standalone o bare metal que son aquellas que se ejecutan e interactúan directamente con el hardware sin la
		necesidad de un sistema operativo que proporcione soporte para la utilización de los periféricos.
		
			\subsubsection{Entorno de ejecución Standalone - Bare Metal}
	    	Para la ejecución de aplicaciones Bare Metal el toolchain se basa en la librería newlib que es una implementación estándar utilizada en
	    	sistemas embebidos. 
	    
			\subsubsection{Entorno de ejecución Linux}
			Por otro lado, para el uso de aplicaciones bajo el sistema operativo Linux el toolchain puede ser compilado en base a la librería uClibc que es la
			opción para sistemas embebidos de la librería glibc utilizada en los sistemas estándar.
		
		\subsection{Criterio para la realización de testing}
		
	
	\section{PROTOTIPO UNO: Implementación del SoC MinSoC en FPGA}
		\subsection{Introducción}
		En este primer prototipo se implementó el proyecto MinSoC en la placa de desarrollo S3ADSP1800A con el fin de verificar el funcionamiento
		del procesador y sus periféricos. Para lograr este objetivo se requiere de la instalación y puesta en funcionamiento de las herramientas
		de síntensis, place \& route(PAR) y programación de la FPGA Spartan 3A. 
		
		\subsection{Requerimientos del prototipo}
		\begin{tabular}{ p{2.5cm} p{8cm} p{3cm} }
		\hline 
		\rowcolor[gray]{0.8} N\textordmasculine Req & Descripción\\
		\hline 
		RQX-PA 1 & El prototipo debe implementar un SoC MinSoC en la placa desarrollo S3ADSP1800A\\ 
		\hline 
		RQX-PA 2 & El prototipo debe garantizar el correcto funcionamiento de todos los periféricos conectados al SoC mediante el bus Wishbone\\ 
		\hline 
		RQX-PA 3 & El prototipo debe interactuar correctamente con las interfaces hardware soportadas de la placa de desarrollo S3ADSP1800A\\ 
		\hline
		RQX-PA 4 & El prototipo debe se capaz de ejecutar programas con capacidad de depuración generados mediante compilación
		cruzada en una arquitectura x86 y/o x86\_64 para arquitectura la OpenRISC\\
		\hline
		RQX-PA 5 & Se debe lograr depurar paso a paso y mediante break points las aplicaciones desarrolladas\\
		\hline
		RQX-PA 6 & El prototipo debe ser evaluado mediante el benchmark CoreMark para el posterior análisis de las capacidades del SoC\\
		\hline		
		\end{tabular}
		
		\subsection{Descripción de la Arquitectura}
		
			
		\subsection{Implementación}
		
		\subsection{Diseño y selección de aplicaciones de prueba}
		
		\subsection{Diagrama de Secuencia}
		
		\subsection{Testing}
		
		\subsection{Conclusión}
		Se pudo implementar con éxito el prototipo número uno cuyos resultados muestran que es de vital importancia 
			
	\section{PROTOTIPO DOS: Implementación del SoC ORPSoc en FPGA}
		\subsection{Introducción}
		Para el prototipo dos se planteó la implementación de un SoC ORPSoC que dispone de mayor cantidad de periféricos que el MinSoC entre ellos un módulo
		que permite manejar las memorias RAM DDR2 incluídas en la placa de desarrollo S3ADSP1800A. Se verificó inicialmente si la herramientas utilizadas en
		el prototipo uno proporcionaban los mismos resultados en este prototipo y luego se ejecutaron las pruebas de funcionamiento del procesador, los
		periféricos y las herramientas de desarrollo y depuración. 
		
		\subsection{Requerimientos del prototipo}
		\begin{tabular}{ p{2.5cm} p{8cm} p{3cm} }
		\hline 
		\rowcolor[gray]{0.8} N\textordmasculine Req & Descripción\\
		\hline 
		RQX-PB 1 & El prototipo debe implementar un SoC ORPSoC en la placa desarrollo S3ADSP1800A\\ 
		\hline 
		RQX-PB 2 & El prototipo debe garantizar el correcto funcionamiento de todos los periféricos conectados al SoC mediante el bus Wishbone\\ 
		\hline 
		RQX-PB 3 & El prototipo debe interactuar correctamente con las interfaces hardware soportadas de la placa de desarrollo S3ADSP1800A\\ 
		\hline
		RQX-PB 4 & Se debe garantizar el acceso a las memorias RAM DDR2 de la plataforma S3ADSP1800A\\
		\hline
		RQX-PB 5 & El prototipo debe se capaz de ejecutar programas con capacidad de depuración generados mediante compilación
		cruzada en una arquitectura x86 y/o x86\_64 para arquitectura la OpenRISC\\
		\hline
		RQX-PB 6 & Se debe lograr depurar paso a paso y mediante break points las aplicaciones desarrolladas\\
		\hline
		RQX-PB 7 & El prototipo debe ser evaluado mediante el benchmark CoreMark para el posterior análisis de las capacidades del SoC\\
		\hline		
		\end{tabular}

		\subsection{Descripción de la Arquitectura}

	
		\subsection{Implementación}

		
		\subsection{Diseño y selección de aplicaciones de prueba}
		
		
		\subsection{Testing}


		\subsection{Conclusión}


	\section{PROTOTIPO TRES: Implementación del SoC ORPSoC en FPGA con Sistema Operativo eCos}
		\subsection{Introducción}
		Para tercer prototipo se añadió al prototipo dos la funcionalidad aportada por el Sistema Operativo eCos(embedded configurable operating system) que
		adiciona la capacidad de ejecución un proceso con múltiples hilos. 

		\subsection{Requerimientos del prototipo}
		
		\begin{tabular}{ p{2.5cm} p{8cm} p{3cm} }
		\hline 
		\rowcolor[gray]{0.8} N\textordmasculine Req & Descripción\\
		\hline 
		RQX-PC 1 & El prototipo debe implementar el sistema operativos eCos\\ 
		\hline 
		RQX-PC 2 & Se deben garantizar el correcto funcionamiento de todos los periféricos utilizando las librerias provistas por el SO\\ 
		\hline 
		RQX-PC 3 & Se requiere la ejecución de múltiples hilos y la ejecución de programas de prueba para determinar los límites de esta capacidad \\ 
		\hline
		RQX-PC 4 & \\
		\hline
		RQX-PC 5 & \\
		\hline
		RQX-PC 6 & \\
		\hline
		RQX-PC 7 & \\
		\hline		
		\end{tabular}
		
		
		\subsection{Implementación}
		
		\subsection{Testing}
		
		\subsection{Conclusión}
		
		
	\section{PROTOTIPO CUATRO: Implementación del SoC ORPSoC en FPGA con Sistema Operativos Linux}
		\subsection{Introducción}
		El núcleo Linux, combinado con un conjunto de algunas otras utilidades de software libre, puede ajustarse dentro del limitado espacio de hardware 
	    de los sistemas embedidos. Una instalación típica de un Linux embebido ocupa en promedio 2 MB. En el prototipo cuatro se utilizó una versión
	    reducida del núcleo de Linux y herramientas provistas por el paquete BusyBox. 
		
		\subsection{Requerimientos del prototipo}
		\begin{tabular}{ p{2.5cm} p{8cm} p{3cm} }
		\hline 
		\rowcolor[gray]{0.8} N\textordmasculine Req & Descripción\\
		\hline 
		RQX-PC 1 & El prototipo debe implementar el sistema operativos eCos\\ 
		\hline 
		RQX-PC 2 & Se deben garantizar el correcto funcionamiento de todos los periféricos utilizando las librerias provistas por el SO\\ 
		\hline 
		RQX-PC 3 & Se requiere la ejecución de múltiples hilos y la ejecución de programas de prueba para determinar los límites de esta capacidad \\ 
		\hline
		RQX-PC 4 & \\
		\hline
		RQX-PC 5 & \\
		\hline
		RQX-PC 6 & \\
		\hline
		RQX-PC 7 & \\
		\hline		
		\end{tabular}
		
		\subsection{Implementación}
		
		\subsection{Testing}
		
		\subsection{Conclusión}
