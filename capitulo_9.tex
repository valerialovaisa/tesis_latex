\chapter{Diseño e Implementación}\label{chap:disenoeimpl}

	\section{Introducción}

	Debido a que se adoptó un modelo de desarrollo en espiral se plantearon una serie de prototipos que incluyen mayor funcionalidad en cada iteración
	del modelo. Inicialmente se planteó un prototipo básico para determinar factibilidad de puesta en funcionamiento de un SoC con microprocesador
	OpenRISC y el desarrollo de aplicaciones que se ejecuten en él. Debido a su funcionalidad reducida y menor consumo de recursos  para su síntesis, se
	seleccionó el proyecto MinSoC para el desarrollo del primer prototipo con el cual se evaluaron y documentaron sus capacidades, permitiendo
	valorar este proyecto para aplicaciones donde no se requieren grandes cantidades de memoria y juegan un papel primordial las entradas/salidas junto
	a la capacidad de procesamiento.  \\
		 
		\subsection{Entorno de ejecución}
		Actualmente OpenRISC está soportado por un conjunto de herramientas de desarrollo (toolchain) de 32 bits, ofreciendo soporte para los lenguajes C y C++
		con bibliotecas estáticas. El toolchain se encuentra disponible en dos formas: una para la ejecución de aplicaciones bajo el sistema operativo Linux y
		otra para la ejecución de aplicaciones standalone o bare metal que son aquellas que se ejecutan e interactúan directamente con el hardware sin la
		necesidad de un sistema operativo que proporcione soporte para la utilización de los periféricos.
		
			\subsubsection{Entorno de ejecución Standalone - Bare Metal}
	    	Para la ejecución de aplicaciones Bare Metal, el toolchain se basa en la biblioteca newlib que es una implementación estándar utilizada en
	    	sistemas embebidos. 
	    
			\subsubsection{Entorno de ejecución Linux}
			Para el uso de aplicaciones bajo el sistema operativo Linux el toolchain puede compilarse en base a la biblioteca uClibc que es la
			opción para sistemas embebidos de la biblioteca glibc utilizada en los sistemas estándar.

		\subsection{Matriz de riesgo}
En cada prototipo se tienen en cuenta los posibles riesgos ya mencionadas y se amplían según la nueva visión del sistema.
		
		\subsection{Criterio para la realización de testing}%%corregir esto
		El testing desarrollado apunta a verificar el cumplimiento de los requerimientos individualmente e incrementalmente. Se pretende utilizar las
		aplicaciones de prueba provistas y desarrollar algunas que verfiquen los requerimientos de compilación, depuración y ejecución de aplicaciones para
		OpenRISC. 
		Los testing sirven para poder detectar la presencia de errores, pero aún si un testing no arroja resultados erróneos, esto no nos garantiza el
		correcto funcionamiento del sistema. El tipo de testing realizado es de verificación. La verificación es el proceso de evaluación de un sistema o
		componente para determinar si el producto cumple con lo que se ha diseñado, es decir, si cada fase de desarrollo dada cumple con los requisitos
		impuestos al inicio de dicha fase.

\newpage
\chapter{Prototipo Uno : Implementación del SoC MinSoC en FPGA}

	\section{Introducción}
		
	En este primer prototipo se implementó el proyecto MinSoC en la placa de desarrollo de Xilinx con el fin de verificar el funcionamiento del procesador y sus periféricos. Para lograr este objetivo se requiere de la instalación y puesta en funcionamiento de las herramientas de síntensis, place \& route(PAR) y programación de la FPGA Spartan 3A. 

	\section{Requerimientos del prototipo}
En el Apartado~\ref{part:partestudio} hemos desarrollado el estudio del problema, donde se establecieron los requerimientos del usuario. Luego realizamos un estudio de la viabilidad, donde se selecciono la placa de desarrollo y las herramientas de trabajo que nos permitió plantear las bases sobre las cuales desarrollar el primer prototipo. Partimos de los siguientes requerimientos, expresados en la tabla~\ref{tab:tdr1}.

		\begin{table}[h]
		\centering
		\begin{tabular}{ p{2.5cm} p{12.5cm} }
		\hline 
		\rowcolor[gray]{0.8} N\textordmasculine Req & Descripción\\
		\hline 
		RQX-PA 1 & El prototipo debe implementar un SoC MinSoC en la placa desarrollo S3ADSP1800A\\ 
		\hline 
		RQX-PA 2 & El prototipo debe garantizar el correcto funcionamiento de todos los periféricos conectados al SoC mediante el bus Wishbone\\ 
		\hline 
		RQX-PA 3 & El prototipo debe interactuar correctamente con las interfaces hardware soportadas de la placa de desarrollo S3ADSP1800A\\ 
		\hline
		RQX-PA 4 & El prototipo debe ser capaz de ejecutar programas con capacidad de depuración generados mediante compilación
		cruzada en una arquitectura x86 y/o x86\_64 para la arquitectura OpenRISC\\
		\hline
		RQX-PA 5 & Se debe lograr depurar paso a paso y mediante break points las aplicaciones desarrolladas\\
		\hline		
		\end{tabular}
		\caption{Tabla de requerimientos del prototipo uno}
		\label{tab:tdr1}
		\end{table}

%\newpage
	
	\section{Matriz de Riesgo para el Prototipo Uno}
La gestión de riesgos es importante particularmente debido a las incertidumbres inherentes con las que se enfrentan muchos proyectos. En la tabla~\ref{tab:riegos} Tabla de riegos vs. Requerimientos del prototipo uno, se muestra el resultado del análisis de la identificación de los riesgos asociados a los requerimientos del prototipo uno, ordenado por importancia y ocurrencia.

\hspace{-15mm}
\vspace{-10mm}
		\begin{table}[h!]
		\centering
		\begin{tabular}{ p{2.5cm} p{9cm} p{1.5cm} p{2cm} }
		\hline 
		\rowcolor[gray]{0.8} N\textordmasculine Req Asociados& Riegos Asociados & Severidad  & Ocurrencia \\
		\hline
		RQX-PA 1& El código RTL disponible en la web tenga fallas reconocidas durante el proceso de síntesis 	 & Crítica       & Probable \\
		\hline
				& Vencimiento de licencias de las herramientas privativas para el proceso de síntesis  & Menor  & Ocasional\\
		\hline
				& Falta de documentación de apoyo para la implentación
del SoC & Crítico & Ocasional\\	 
		\hline

		RQX-PA 2,RQX-PA 3 & Imposibilidad de acceso a alguno de los periféricos de la placa de desarrollo &  Catastrófico  & Probable\\
		\hline
		& Soporte de acceso a periféricos inexistente o con presencia
de bugs & Crítica  & Ocasional\\	 
		\hline
		RQX-PA 4& Compilación cruzada realizada con compiladores obsoletos & Severo  &  Ocasional\\ 
		\hline
		&Problemas en la compilación e instalación de las herramientas de desarrollo de software  & Severo  &  Ocasional\\ 
		\hline
		RQX-PA 5& Problemas con el driver provisto por el fabricante del  dispositivo programador JTAG -USB & Severo&  Ocasional\\
		\hline
		\end{tabular}
		\caption{Tabla de riegos vs. Requerimientos del prototipo uno}
		\label{tab:riegos}
		\end{table}

En la planificación de riesgos se considero cada uno de los riesgos identificados y se desarrollaron estrategias para manejar dichos riesgos. Por cada uno de los riesgos se pensaron acciones a llevar a cabo para minimizar el impacto que tendrá sobre el prototipo en caso de ocurrir, como así también qué información sería necesario observar para anticiparnos a estos problemas.
En la tabla~\ref{tab:planificación} podemos ver los resultados de la planificación de riesgos.

		\begin{table}[h!]
		\centering
		\begin{tabular}{ p{4cm} p{4cm} p{4cm} p{3cm} }
		\hline 
		\rowcolor[gray]{0.8} Riesgo & Consecuencia & Estrategia preventiva & Estrategia de contingencia\\
		\hline
		El código RTL disponible en la web tiene fallas reconocidas durante el proceso de síntesis.&Retraso en los tiempos de implementación.& Análisis
		previo de guías de resolución de problemas y preguntas frecuentes & Revisión de código y consulta a los desarrolladores \\
		\hline
		Vencimiento de licencias de las herramientas privativas para el proceso de síntesis & Imposibilidad de ejecución de las herramientas requeridas & Verificación previa de los vencimientos de las licencias & Renovación de licencia \\	 
		\hline
		Falta de documentación de apoyo para la implentación
del Soc& Pérdida de tiempo en realizar la implementación
del SoC & Revisión previa de la documentación existente para la elección del
SoC & Dedicación de mayor tiempo para la implementación\\ 
		\hline
		 Imposibilidad de acceso a alguno de los periféricos de la placa de desarrollo & Imposibilitad de acceso a los periféricos del kit &Verificación previa de los recursos utilizados por el SoC a implementar & Depuración del codigo RTL del proyecto\\		
		\hline
		Problemas con el driver provisto por el fabricante del  dispositivo programador JTAG -USB  & Imposibilidad de configuración de la FPGA y depuración de aplicaciones &Análisis previo de los driver de los dispositivos a utilizar &  Utilización de dispositivos alternativos.\\		
		\hline
		\end{tabular}
		\caption{Planificación de riesgos del prototipo uno}
		\label{tab:planificación}
		\end{table}



		\newpage
		\section{Descripción de la Estructura del Prototipo Uno}
			
		La figura~\ref{fig:minsoc} muestra un diagrama de despliegue del prototipo planteado. Se utiliza una estación de trabajo corriendo las herramientas
		de compilación (GCC) y depuración (GDB) en sus versiones de compilación cruzada para la arquitecturas destino OpenRISC. Se requiere de un servidor que
		proporcione un puerto para la depuración con GDB, acción realizada por la aplicación Advanced JTAG Bridge que provee la interfaz de comunicación
		entre el TAP JTAG conectado al SoC a través el cable JTAG y la estación de trabajo.
		
		\begin{figure}[!h]
 		\begin{center}
  		\includegraphics[width=1\textwidth,keepaspectratio=true]{./images/proto1}
  		\caption{Diagrama de despliegue prototipo uno}
  		\label{fig:minsoc}
 		\end{center}
		\end{figure}
		
		El módulo Advanced Debug Interface proporciona soporte para la depuración via JTAG, está conectado directamente al microprocesador y a su vez al bus
		general Wishbone donde se conectan el resto de los periféricos. Se conectan al sistema mediante el bus antes mencionado: un modulo de RAM
		sintetizado, un módulo de arranque (startup), un módulo SPI que interactúa con la memoria flash on board de la placa S3ADSP1800A, un módulo UART y
		un módulo ETHERNET. El sistema inyecta un pequeño código de arranque directamente al bus de instrucciones al iniciar o luego de un reset. 


			
		\section{Implementación}

La implementación se realiza normalmente en cinco pasos básicos: 
\begin {itemize}
\item Descarga del script de instalación.
\item Ejecución del script de instalación.
\item Configuración de la placa específica para la síntesis.
\item Generación del flujo de bits.
\item Programación de la FPGA Spartan 3a.
 \end {itemize}
Para mas detalle de estas cinco etapas en la implementación sobre la placa de desarrollo  de Xilinx puede ver el anexo~\ref{app:apendice1}

% Consulte la Guía de MinSoc en el Anexo ~\ref{app:apendice1} para la realización pasos a seguir en la implementación sobre la placa de desarrollo  S3ADSP1800A.
		
		%%%%%%%\subsection{Diagrama de Secuencia}
		%\subsection{Selección y Desarrollo de aplicaciones de prueba}
\newpage
		\section{Testing} \label {testing:proto1}

Para llevar a cabo el testing del prototipo se planificó una serie de casos de prueba sencillos que validen cada uno de los requerimiento planteados.
El principal requisito para el comienzo de las pruebas fueron las herramientas de compilación y depuración funcionando correctamente. En una primera
instancia fue necesario disponer de algún medio de respuesta de parte del SoC, razón por la cual se realizó un caso de prueba que compruebe la
comunicación SoC - PC mediante UART (véase tabla~\ref{tab:cp1}).  En una segunda instancia de prueba, se probó la unidad Ethernet para garantizar correcto funcionamiento, estableciendo una conexión entre el SoC y la PC de pruebas (véase tabla~\ref{tab:cp2}). 

Para evaluar las capacidades básicas del microprocesador se desarrolló un programa de prueba multiplicador de matrices binarias. Las matrices se
generan a partir de una secuencia binaria pseudoaleatoria. El programa incrementa iteración a iteración el grado de las matrices multiplicadas y
realiza los productos correspondientes registrando el tiempo que tarda en realizar el producto completo (véase tabla~\ref{tab:cp3}). En un último caso de prueba, el programa se
depura paso a paso y mediante breakpoints para verificar su correcto funcionamiento(véase tabla~\ref{tab:cp4}).

 \begin{table}[h!]
		\centering
		\begin{tabular}{ p{5cm} p{10cm}  }
		\hline 
	    \rowcolor[gray]{0.8}  Caso de Prueba&  Interactuar correctamente con la interfaz hardware UART\\
		\hline 
		Código de Requerimiento & RQX-PA 3\\ 
		\hline 
		Requerimiento  &  El prototipo debe interactuar correctamente con las interfaces hardware soportadas de la placa de desarrollo S3ADSP1800A\\ 
		\hline 
		Código de Testing & T001\\ 
		\hline
		Propósito & Comprobación del correcto funcionamiento del componente UART \\
		\hline
		Realizado Por & Lovaisa Michelini, Valeria \\
		\hline	
		Entorno de Ejecución & Bare Metal \\
		\hline
		Precondiciones &  \begin {itemize}
							\item Instanciación en la FPGA del SoC.
							\item Ejecución de una terminal serie en la PC
							\item Comunicación establecida entre la placa de desarrollo y la PC
							\end {itemize}\\
		\hline
		Secuencia de Ejecución & Ejecución de la aplicación de prueba \verb|uart.or32|  \\
		\hline
		Postcondiciones & MinSoc corriendo sobre FPGA, lo que se manifiesta en la impresión de pantalla de la terminal remota en la PC\\
		\hline
 \multicolumn{2}{>{\columncolor[gray]{.8}}c}{Resultados}\\
		\hline
		Resultados Esperados & Poder observar por la terminal serial una cadena de string.\\
		\hline	
		Resultados Obtenidos & Al ejecutarse la aplicación de prueba del UART se pudo observar la cadena de string esperada "HELLO WORLD" \\
		\hline
		\end{tabular}
		\label{tab:cp1}
		\caption{Caso de prueba T001}
		\end{table}

\newpage
\begin{table}[h!]
		\centering
		\begin{tabular}{ p{5cm} p{10cm}  }
		\hline 
	\rowcolor[gray]{0.8}  Caso de Prueba&  Interactuar correctamente con la interfaz hardware de Ethernet\\
		\hline 
		Código de Requerimiento & RQX-PA 3\\ 
		\hline 
		Requerimiento  &  El prototipo debe Interactuar correctamente con las interfaces hardware soportadas de la placa de desarrollo S3ADSP1800A\\ 
		\hline 
		Código de Testing & T002\\ 
		\hline
		Propósito &  El envío de paquetes de Ethernet mediante broadcast desde el SoC  \\
		\hline
		Realizado Por & Gomez, Pablo \\
		\hline	
		Entorno de Ejecución & Bare Metal \\
		\hline
		Precondiciones & \begin {itemize}
							\item Instanciación en la FPGA del SoC.
							\item Ejecución de una terminal serie en la PC
							\item Ejecución de una aplicación sniffer en la PC
							\item Comunicación establecida entre la placa de desarrollo y la PC
							\end {itemize} \\
		\hline
		Secuencia de Ejecución &  Ejecutación de la aplicación de prueba \verb|eth.or32|\\
		\hline
		Postcondiciones &  MinSoc corriendo sobre FPGA, lo que se manifiesta en la impresión de pantalla de la terminal serie en la PC y en los paquetes captados por la aplicación wireshark \\
		\hline
 		\multicolumn{2}{>{\columncolor[gray]{.8}}c}{Resultados}\\
		\hline
		Resultados Esperados & Poder observar por la terminal serial una cadena de string indicando el inicio del envío de paquetes Ethernet y verificar la llegada de los mismos por medio de una aplicación sniffer \\
		\hline	
		Resultados Obtenidos & Al ejecutarse un programa de prueba pudo observarse la cadena de string ``HELLO WORLD`` y observar 0xFF02B4050 en los paquetes captados por la aplicación \verb|wireshark| \\
		\hline
		\end{tabular}
		\label{tab:cp2}
		\caption{Caso de prueba T002}
		\end{table}

%%%%%%%%%%%%%%%%%%%%%%%%%%%%%%%%%%%%%%%%%
\newpage
\begin{table}[h!]
		\centering
		\begin{tabular}{ p{5cm} p{10cm}  }
		\hline 
		\rowcolor[gray]{0.8}  Caso de Prueba& Ejecución de la aplicación multiplicadora de matrices binarias generadas mediante una Secuencia binaria pseudo aleatoria (PRBS) \\
		\hline 
		Código de Requerimiento & RQX-PA 4\\ 
		\hline 
		Requerimiento  &  El prototipo debe ser capaz de ejecutar programas con capacidad de depuración, generados mediante compilación cruzada en una arquitectura x86 y/o x86\_64 para la arquitecturan OpenRISC\\ 
		\hline 
		Código de Testing & T003\\ 
		\hline
		Propósito &  Compilación y Ejecución de un  programa sencillo en lenguaje C.Verificación de la capacidad de procesamiento del sistema mediante la realización de los productos con matrices cuadradas incrementando su tamaño iteración a iteración.  
\\
		\hline
		Realizado Por & Gomez, Pablo \\
		\hline	
		Entorno de Ejecución & Bare Metal \\
		\hline
		Precondiciones &\begin {itemize}
							\item Instanciación en la FPGA del SoC.
							\item Ejecución de una terminal serie en la PC
							\item Descarga e instalación del compilador cruzado 
							\item Comunicación establecida entre la placa de desarrollo y la PC
							\end {itemize}
 \\
		\hline
		Secuencia de Ejecución & Ejecución de la aplicación multiplicadora de matrices \verb|multmat.or32| \\
		\hline
		Postcondiciones & Aplicación corriendo sobre el MinSoc instanciado en la FPGA, lo que se manifiesta en la impresión de pantalla de los tiempos de resolución del conjunto de matrices cuadradas multiplicadas en la terminal serie en la PC \\
		\hline
 		\multicolumn{2}{>{\columncolor[gray]{.8}}c}{Resultados}\\
		\hline
		Resultados Esperados & Poder ver la ejecución correcta de un programa de prueba desarrollado \\
		\hline	
		Resultados Obtenidos & Al ser ejecutado se observaron los resultados mostrados en el Bloque~\ref{lst:salidamult}   \\
		\hline
		\end{tabular}
		\caption{Caso de prueba T003}
		\label{tab:cp3}
		\end{table}

\newpage
\begin{lstlisting}[caption={Salida de la terminal serie durante la ejecución del programa multmat.or32},label={lst:salidamult}]
Tick de ejecucion setup tick :0x00000029 En us = 0x00000802
0x00000001
Tick de ejecucion seteo de variables :0x000177a6
Tick de ejecucion de la multiplicacion :0x00000316
0x00000002
Tick de ejecucion seteo de variables :0x0001be62
Tick de ejecucion de la multiplicacion :0x00000fc8
0x00000003
Tick de ejecucion seteo de variables :0x0002344b
Tick de ejecucion de la multiplicacion :0x00002e98
0x00000004
Tick de ejecucion seteo de variables :0x0002d952
Tick de ejecucion de la multiplicacion :0x000067a8
0x00000005
Tick de ejecucion seteo de variables :0x0003ad86
Tick de ejecucion de la multiplicacion :0x0000c31a
0x00000006
Tick de ejecucion seteo de variables :0x0004b0e7
Tick de ejecucion de la multiplicacion :0x00014910
0x00000007
Tick de ejecucion seteo de variables :0x0005e366
Tick de ejecucion de la multiplicacion :0x000201ac
0x00000008
Tick de ejecucion seteo de variables :0x00074512
Tick de ejecucion de la multiplicacion :0x0002f510
0x00000009
Tick de ejecucion seteo de variables :0x0008d5eb
Tick de ejecucion de la multiplicacion :0x00042b5e
0x0000000a
Tick de ejecucion seteo de variables :0x000a95e7
Tick de ejecucion de la multiplicacion :0x0005acb8
0x0000000b
Tick de ejecucion seteo de variables :0x000c850b
Tick de ejecucion de la multiplicacion :0x00078140
0x0000000c
Tick de ejecucion seteo de variables :0x000ea357
Tick de ejecucion de la multiplicacion :0x0009b118
0x0000000d
Tick de ejecucion seteo de variables :0x0010f0d0
Tick de ejecucion de la multiplicacion :0x000c4462
0x0000000e
Tick de ejecucion seteo de variables :0x00136d62
Tick de ejecucion de la multiplicacion :0x000f4340
0x0000000f
Tick de ejecucion seteo de variables :0x00161921
Tick de ejecucion de la multiplicacion :0x0012b5d4
Fin del programa

\end{lstlisting}

%%%%%%

\newpage
\begin{table}[h!]
		\centering
		\begin{tabular}{ p{5cm} p{10cm}  }
		\hline 
		\rowcolor[gray]{0.8}  Caso de Prueba&  Se debe lograr depurar paso a paso y mediante breakpoints las aplicaciones desarrolladas\\
		\hline 
		Código de Requerimiento & RQX-PA 5\\ 
		\hline 
		Requerimiento  &  Se debe lograr depurar paso a paso y mediante breakpoints las aplicaciones desarrolladas\\ 
		\hline 
		Código de Testing & T004\\ 
		\hline
		Propósito &   Compilación en modo debug y Ejecución de un programa en lenguaje C mediante GDB. Verificar si la cadena de debug está funcionando.\\
		\hline
		Realizado Por & Gomez, Pablo \\
		\hline	
		Entorno de Ejecución & Bare Metal \\
		\hline
		Precondiciones & \begin {itemize}
							\item Instanciación en la FPGA del SoC
							\item Ejecución de una terminal serie en la PC
							\item Aplicación compilada con flag de depuración 
							\item Comunicación establecida entre la placa de desarrollo y la PC
							\end {itemize}
\\
		\hline
		Secuencia de Ejecución &  Ejecución de la aplicación multiplicadora de matrices \verb|multmat.or32|\\
		
		\hline
		Postcondiciones & Aplicación corriendo en modo debug sobre el MinSoc instanciado en la FPGA, lo que se manifiesta en la detención de la ejecución en cada punto de parada.\\
		\hline
 		\multicolumn{2}{>{\columncolor[gray]{.8}}c}{Resultados}\\
		\hline
		Resultados Esperados & Poder observar paso a paso la ejecución del programa. \\
		\hline	
		Resultados Obtenidos & El programa puede verse a través de la terminal serie detenido en cada punto de parada. \\
		\hline
		\end{tabular}
		\caption{Caso de prueba T004}
		\label{tab:cp4}
		\end{table}


		\section{Conclusión}
		Este prototipo permitió mostrar la posibilidad de instanciar sobre el kit de desarrollo de Xilinx un System on Chip Básico Open Source en una FPGA
		Spartan 3A. La mayoría del tiempo invertido en la implementación de este prototipo se utilizo en la determinación de las herramientas de
		desarrollo necesarias, su instalación y puesta en funcionamiento. Es importante resaltar que la documentación sobre el tema era insuficiente o
		escasa. 
		
		Además existieron algunos problemas durante la síntesis del SoC. Sin embargo, una vez funcionando, las herramientas de diseño de hardware (Sintesis,
		PAR , Implementación) y las de diseño de software (toolchain) se ejecutaron con facilidad la mayoría de las aplicaciones de prueba seleccionadas y
		desarrolladas.
		
\newpage		
\chapter{Prototipo Dos : Implementación del SoC ORPSoC en FPGA}
		\section{Introducción}

		Para el prototipo dos se planteó la implementación de un sistema ORPSoC que dispone de mayor cantidad de periféricos que el MinSoC, entre ellos un
		módulo que permite manejar las memorias RAM DDR2 incluídas en la placa de desarrollo S3ADSP1800A. Se verificó inicialmente si la herramientas
		utilizadas en el prototipo uno proporcionaban los mismos resultados en este prototipo y luego se ejecutaron las pruebas de funcionamiento del
		procesador, los periféricos y las herramientas de desarrollo y depuración. 
		
%\newpage
%\clearpage				
		\section{Requerimientos del Prototipo}
Para la base de desarrollo del prototipo dos se partió de los requerimientos, expresados en la tabla~\ref{tab:tdr2}.

		\begin{table}[h!]
		\centering
		\begin{tabular}{ p{2.5cm} p{14.5cm} }
		\hline 
		\rowcolor[gray]{0.8} N\textordmasculine Req & Descripción\\
		\hline 
		RQX-PB 1 & Al prototipo debe implementar un SoC ORPSoC en la placa desarrollo S3ADSP1800A\\ 
		\hline 
		RQX-PB 2 & El prototipo debe garantizar el correcto funcionamiento de todos los periféricos conectados al SoC mediante el bus Wishbone\\ 
		\hline 
		RQX-PB 3 & El prototipo debe interactuar correctamente con las interfaces hardware soportadas de la placa de desarrollo S3ADSP1800A\\ 
		\hline
		RQX-PB 4 & Se debe garantizar el acceso a las memorias RAM DDR2 de la plataforma S3ADSP1800A\\
		\hline
		RQX-PB 5 & El prototipo debe ser capaz de ejecutar programas con capacidad de depuración generados mediante compilación
		cruzada en una arquitectura x86 y/o x86\_64 para la arquitectura OpenRISC\\
		\hline
		RQX-PB 6 & Se debe lograr depurar paso a paso y mediante breakpoints las aplicaciones desarrolladas\\
		\hline
		RQX-PB 7 & El prototipo debe evaluarse mediante el benchmark CoreMark para el posterior análisis de las capacidades del SoC\\
		\hline		
		\end{tabular}
		\caption{Tabla de requerimientos del prototipo dos}
		\label{tab:tdr2}
		\end{table}


		\section{Matriz de Riesgo para el Prototipo Dos} 
En la tabla~\ref{tab:riegos2} Tabla de riegos vs. Requerimientos del prototipo dos se muestra al igual que en el prototipo anterior el resultado del análisis de la identificación de los riesgos asociados a los requerimientos del prototipo dos, ordenado por importancia y ocurrencia.
		\begin{table}[h!]
		\centering
		\begin{tabular}{ p{2.5cm} p{9cm} p{1.5cm} p{2cm} }
		\hline 
		\rowcolor[gray]{0.8} N\textordmasculine Req Asociados& Riegos Asociados & Severidad  & Ocurrencia \\
		\hline
		RQX-PB 1& EL código RTL disponible en la web tenga fallas reconocidas durante el proceso de síntesis & Crítica       & Probable \\
		\hline				
				& Falta de documentación de apoyo para la implentación
del Soc & Crítico & Ocasional\\	 
		\hline
		RQX-PB 2,RQX-PB 3,RQX-PA 4 & Imposibilidad de acceso a SDRAM DDR2 de el kit& Catastrófico & Probable\\
		\hline
		RQX-PB 5&Problemas en la compilación e instalación de las herramientas de desarrollo de software  & Severo  &  Ocasional\\ 
		\hline
		RQX-PB 6& Problemas de compactibilidad con el modulo de debug del proyecto ORPSoC  & Crítico&  Ocasional\\
		\hline
		RQX-PB 7 & Problemas con el compilador para la generación del binario del benchmark CoreMark  & Crítico&  Ocasional\\
		\hline
		\end{tabular}
		\caption{Tabla de riegos vs. Requerimientos del prototipo dos}
		\label{tab:riegos2}
		\end{table}

En la tabla~\ref{tab:planificación2} podemos ver los resultados de la planificación de riesgos.

 		\begin{table}[h!]
		\centering
		\begin{tabular}{ p{4cm} p{4cm} p{4cm} p{3cm} }
		\hline 
		\rowcolor[gray]{0.8} Riesgo & Consecuencia & Estrategia preventiva & Estrategia de contingencia\\
		\hline
	Código RTL disponible en la web tiene fallas reconocidas durante el proceso de síntesis.&Retraso en los tiempos de implementación.& Análisis previo de guías de resolución de problemas y preguntas frecuentes & Revisión de código y consulta a los desarrolladores \\		 
		\hline
		Falta de documentación de apoyo para la implentación
del Soc& Perdida de tiempo en realizar la implementación
del SoC & Revisión previa de la documentación existente para la elección del
SoC & Dedicación de mayor tiempo para la implementación\\ 
		\hline
		 Imposibilidad de acceso a SDRAM DDR2 de el kit & Falta de capacidad para la ejecución de aplicaciones & Análisis previo de guía de resolución de problemas y documentación del core RAM & Depuración del código RTL del proyecto\\
		\hline
		Problemas en la compilación e instalación de las herramientas de desarrollo de software & Imposibilidad de desarrollo de aplicaciones que se ejecuten en el proyecto ORPSoC & Verificar la posibilidad de ejecución de las herramientas en diferentes SO & Utilización de las herramientas en alguno de los SO compatibles\\			
		\hline
		Problemas de compatibilidad con el modulo de debug del proyecto OrpSoC con el programador JTAG-USB & Imposibilidad de depuración de aplicaciones &Análisis de la documentación del proyecto& Utilización de otros  dispositivos compatibles.\\		
		\hline
		Problemas con el compilador para la generación del binario del benchmark CoreMark & Imposibilidad de medir el rendimiento de la unidad central de procesamiento (CPU) & Verificar la posibilidad de ejecución de las herramientas en diferentes SO & Utilización de métodos alternativos para medir el redimiento en sistemas embebidos.\\
		\hline
		\end{tabular}
		\caption{Planificación de riesgos del prototipo dos}
		\label{tab:planificación2}
		\end{table}


\newpage
		\section{Descripción de la Estructura del Prototipo Dos}
		La figura~\ref{fig:orpsoc} muestra un esquema de la arquitectura planteada en el prototipo. Al igual que en el prototipo uno se utiliza una estación
		de trabajo corriendo las herramientas de compilación (GCC) y depuración (GDB) en sus versiones de compilación cruzada. Se requiere de un servidor
		que proporcione un puerto para la depuración con GDB y, a diferencia del prototipo uno, este servicio lo proporciona la aplicación OpenRISC Debug
		Proxy.
		
		\begin{figure}[!h]
 		\begin{center}
  		\includegraphics[width=1\textwidth,keepaspectratio=true]{./images/ORPSoCdia}
  		\caption{Arquitectura prototipo dos}
  		\label{fig:orpsoc} 
 		\end{center}
		\end{figure}
	
		\section{Implementación}

		La implementación se realiza siguiendo las mismas etapas que en el prototipo anterior. Para mas detalle en la implementación sobre la placa de desarrollo Xilinx puede ver el anexo~\ref{app:apendice2}.
%Puede consultarse la Guía de ORPSoC en el Anexo ~\ref{app:apendice2} para una comprensión más profunda de los pasos a seguir en la implementación sobre la placa de desarrollo S3ADSP1800A.
El módulo de Debug proporciona soporte para la depuración via JTAG. Se conectan al sistema mediante un Bus Wishbone: un modulo de RAM que provee
	acceso a las memorias SDRAM on board de la placa S3ADSP1800A, un módulo SPI que interactúa con la memoria flash on board de la placa antes
	mencionada, un módulo UART y un módulo ETHERNET.
		

		\section{Testing}
		
		El Prototipo Dos se evalúa de forma similar al anterior debido a que el cambio del SoC instanciado requiere nuevamente la verificación de los
		dispositivos básicos de comunicación así como de las herramientas de diseño de hardware y software.
		
		Inicialmente se realizó un caso de prueba que compruebe la comunicación SoC - PC  Ethernet que de igual manera verifica el correcto funcionamiento
		de la interfaz UART (véase tabla~\ref{tab:cp5}). La característica más importante de la implementación de ORPSoC es su capacidad de interactuar con las memorias SDRAM on board
		dispuestas en la placa de desarrollo. Se realizó un caso de prueba especial que valide el funcionamiento de las mismas (véase tabla~\ref{tab:cp6}).

		Para comprobar el correcto funcionamiento de las capacidades de depuración se realizó un caso de prueba simple mediante un programa que calcula la
		cantidad de números primos existente en un intervalo seleccionado(véase tabla~\ref{tab:cp7}). Ante la necesidad de evaluar las capacidades del sistema embebido se realizó un caso de prueba basado en el Benchmark Coremark (véase tabla~\ref{tab:cp8}).

\newpage
		\begin{table}[h!]
		\centering
		\begin{tabular}{ p{5cm} p{10cm}  }
		\hline 
		\rowcolor[gray]{0.8}  Caso de Prueba&  Interactuar correctamente con las interfaz hardware de Ethernet\\
		\hline 
		Código de Requerimiento & RQX-PB 2 , RQX-PB 3\\ 
		\hline 
		Requerimiento  &El prototipo debe garantizar el correcto funcionamiento de todos los periféricos conectados al SoC mediante el bus Wishbone\\ 
						&  El prototipo debe Interactuar correctamente con las interfaces hardware soportadas de la placa de desarrollo S3ADSP1800A\\
		\hline 
		Código de Testing & T005\\ 
		\hline
		Propósito &  El envío de paquetes echo del protocolo ICMP desde el SoC  \\
		\hline
		Realizado Por & Gomez, Pablo \\
		\hline	
		Entorno de Ejecución & Bare Metal \\
		\hline
		Precondiciones & \begin {itemize}
							\item Instanciación en la FPGA del SoC.
							\item Ejecución de una terminal serie en la PC
							\item Ejecución de una aplicación snifer en la PC
							\item Comunicación establecida entre la placa de desarrollo y la PC
							\item Herramientas de compilación cruzada funcionando correctamente
							\item Programas de prueba compilados correctamente
							\end {itemize} \\
		\hline
		Secuencia de Ejecución &  Ejecutación de la aplicación de prueba \verb|ethmac-ping.or32|\\
		\hline
		Postcondiciones &  El programa de prueba ejecutándose sobre la RAM de la plataforma, lo que se manifiesta en la impresión de pantalla de la terminar serie en la PC y en los paquetes captados por la aplicación \verb|wireshark| \\
		\hline
 		\multicolumn{2}{>{\columncolor[gray]{.8}}c}{Resultados}\\
		\hline
		Resultados Esperados & Poder observar por la terminal serial una cadena de string indicando el inicio del envío de paquetes ICMP y verificar la llegada de los mismos por medio de una aplicación ICMP \\
		\hline	
		Resultados Obtenidos & Al ejecutarse el programa de prueba pudo observarse la cadena mostrada en la código~\ref{lst:reseth}\\
		\hline
		\end{tabular}
		\caption{Caso de prueba T005}
		\label{tab:cp5}
		\end{table}

\begin{lstlisting}[frame=single,caption={Salida de la terminal serie durante la ejecución del programa ethmac-ping.or32},label={lst:reseth}]
eth ping program
board IP : 192.168.1.2
\end{lstlisting}

\newpage
		\begin{table}[h!]
		\centering
		\begin{tabular}{ p{5cm} p{10cm}  }
		\hline 
	    \rowcolor[gray]{0.8}  Caso de Prueba& Interactuar correctamente con la memoria RAM. \\
		\hline 
		Código de Requerimiento & RQX-PB 4 \\ 
		\hline 
		Requerimiento  &  Se debe garantizar el acceso a las memorias RAM DDR2 de la plataforma S3ADSP1800A\\ 	

		\hline 
		Código de Testing & T006\\ 
		\hline
		Propósito & Comprobación del correcto funcionamiento del módulo de RAM implementado en el SoC\\
		\hline
		Realizado Por & Lovaisa Michelini, Valeria \\
		\hline	
		Entorno de Ejecución & Bare Metal \\
		\hline
		Precondiciones &  \begin {itemize}
							\item Instanciación del SoC en la FPGA.
							\item Ejecución de una terminal serie en la PC
							\item Comunicación establecida entre la placa de desarrollo y la PC
							\item Herramientas de compilación cruzada funcionando correctamente
							\item Programas de prueba compilados correctamente
							\end {itemize}\\
		\hline
		Secuencia de Ejecución & Ejecutación de la aplicación de prueba \verb|sdra-row.or32| \\
		\hline
		Postcondiciones & ORPSoC corriendo sobre FPGA, y la aplicación ejecutándose sobre la memoria RAM lo que se manifiesta en la impresión de pantalla de la terminar remota en la PC\\
		\hline
 		\multicolumn{2}{>{\columncolor[gray]{.8}}c}{Resultados}\\
		\hline
		Resultados Esperados & Poder observar por la terminal serial una cadena de string.\\
		\hline	
		Resultados Obtenidos & Al ejecutarse la aplicación de prueba de RAM \verb|sdra-row.o32| se pudo observar la cadena de string esperada "TEST OK!" \\
		\hline
		\end{tabular}
		\caption{Caso de prueba T006}
		\label{tab:cp6}
		\end{table}

\newpage
		\begin{table}[h!]
		\centering
		\begin{tabular}{ p{5cm} p{10cm}  }
		\hline 
		\rowcolor[gray]{0.8}  Caso de Prueba& Ejecución de la aplicación compilada en modo debug para números primos\\
		\hline 
		Código de Requerimiento & RQX-PA 5\\ 
		\hline 
		Requerimiento  & El prototipo debe ser capaz de ejecutar programas con capacidad de depuración generados mediante compilación cruzada en una arquitectura x86 y/o x86-64 para la arquitectura OpenRISC\\ 
		\hline 
		Código de Testing & T007\\ 
		\hline
		Propósito &  Compilación en modo debug y ejecución de un programa de calculo de números primos en lenguaje C.
\\
		\hline
		Realizado Por & Gomez, Pablo \\
		\hline	
		Entorno de Ejecución & Bare Metal \\
		\hline
		Precondiciones &\begin {itemize}
							\item Instanciación en la FPGA del SoC.
							\item Ejecución de una terminal serie en la PC
							\item Descarga e instalación del complicador cruzado 
							\item Comunicación establecida entre la placa de desarrollo y la PC
							\item Herramientas de compilación cruzada funcionando correctamente
							\item Programas de prueba compilados correctamente
							\end {itemize}
 \\
		\hline
		Secuencia de Ejecución & Ejecución de la aplicación \verb|numprim.or32|\\
		\hline
		Postcondiciones & Aplicación corriendo sobre el ORPSoC instanciado en la FPGA, lo que se manifiesta en la impresión de pantalla de los números primos \\
		\hline
 		\multicolumn{2}{>{\columncolor[gray]{.8}}c}{Resultados}\\
		\hline
		Resultados Esperados & Poder observar por la terminal serial el resultado de la ejecución de la aplicación \\
		\hline	
		Resultados Obtenidos & Al ejecutarse el programa contador de números primos pudo observarse la salida mostrada en la código~\ref{lst:rescom}\\
		\hline
		\end{tabular}
		\caption{Caso de prueba T007}
		\label{tab:cp7}
		\end{table}

\begin{lstlisting}[frame=single,caption={Salida de la terminal serie durante la ejecución del programanumprim.elf},label={lst:rescom}]
Cantidad de numero primos entre 1 y 2000 : 303
\end{lstlisting}


\newpage
		\begin{table}[h!]
		\centering
		\begin{tabular}{ p{5cm} p{10cm}  }
		\hline 
	    \rowcolor[gray]{0.8}  Caso de Prueba&  Ejecución del Benchmark CoreMark\\
		\hline 
		Código de Requerimiento & RQX-PB 7\\ 
		\hline 
		Requerimiento  &  El prototipo debe ser evaluado mediante el benchmark CoreMark para el posterior análisis de las capacidades del SoC\\ 
		\hline 
		Código de Testing & T008\\ 
		\hline
		Propósito & Medir el rendimiento de la unidad central de procesamiento embebida en el SoC\\ 
		\hline
		Realizado Por & Lovaisa Michelini, Valeria \\
		\hline	
		Entorno de Ejecución & Bare Metal \\
		\hline
		Precondiciones &  \begin {itemize}
							\item Instanciación del SoC en la FPGA.
							\item Ejecución de una terminal serie en la PC
							\item Comunicación establecida entre la placa de desarrollo y la PC
							\item Herramientas de compilación cruzada funcionando correctamente
							\item Programa coremark compilado correctamente
							\end {itemize}\\
		\hline
		Secuencia de Ejecución & Ejecución de la aplicación \verb|coremark| \\
		\hline
		Postcondiciones & ORPSoC corriendo sobre FPGA, lo que se manifiesta en la impresión de pantalla de la terminar serial en la PC de los resultados del programa de prueba \\
		\hline
 		\multicolumn{2}{>{\columncolor[gray]{.8}}c}{Resultados}\\
		\hline
		Resultados Esperados & Poder observar por la terminal serial los resultados de medición de performance.\\
		\hline	
		Resultados Obtenidos & Al ejecutarse la aplicación \verb|coremark| se pudo observar el código~\ref{lst:rescrm} \\
		\hline
		\end{tabular}
		\caption{Caso de prueba T008}
		\label{tab:cp8}
		\end{table}


\newpage

\begin{lstlisting}[frame=single,caption={Salida de la terminal serie de los resultados de la ejecución del benchmark},label={lst:rescrm},breaklines]
2K performance run parameters for coremark.
CoreMark Size    : 666
Total ticks      : 1500
Total time (secs): 15.000000
Iterations/Sec   : 40.000000
Iterations       : 600
Compiler version : GCC4.5.1-or32-1.0rc4
Compiler flags   : -O2  -mboard=s3adsp1800a -DPERFORMANCE_RUN=1  
Memory location  : STACK
seedcrc          : 0xe9f5
[0]crclist       : 0xe714
[0]crcmatrix     : 0x1fd7
[0]crcstate      : 0x8e3a
[0]crcfinal      : 0xbd59
Correct operation validated. See readme.txt for run and reporting rules.
CoreMark 1.0 : 40.000000 / GCC4.5.1-or32-1.0rc4 -O2  -mboard=s3adsp1800a -DPERFORMANCE_RUN=1   / STACK
\end{lstlisting}

		\section{Conclusión}
		El Prototipo Dos permitió evaluar la factibilidad de implementación de un SoC Softcore con capacidades que le confieren la posibilidad de ejecutar
		sin problemas un Sistema Operativo Embebido. Al poseer mayores prestaciones de memoria que las del proyecto MinSoC permite ejecutar aplicaciones
		complejas de exigencias elevadas compiladas para la arquitectura OpenRISC. 
		 
		Si bien el proyecto ORPSoC se encuentra documentado en mayor detalle, la documentación resulta bastante acotada cuando se pretende implementar
		proyectos para aplicaciones reales fuera de aquellas pruebas básicas del sistema. 		


\newpage
\chapter{Prototipo Tres : Implementación del SoC ORPSoC en FPGA con Sistema Operativo eCos}
		\section{Introducción}

		Para el tercer prototipo se añadió al prototipo dos la funcionalidad aportada por el Sistema Operativo de tiempo real eCos (embedded configurable operating system) que adiciona la capacidad de la ejecución de múltiples hilos. 

		\section{Requerimientos del Prototipo}
En la tabla~\ref{tab:tdr3} se muestran los requerimientos para la implemtación del prototipo tres.		

		\begin{table}[h!]
		\centering		
		\begin{tabular}{ p{2.5cm} p{14.5cm}}
		\hline 
		\rowcolor[gray]{0.8} N\textordmasculine Req & Descripción\\
		\hline 
		RQX-PC 1 & El prototipo debe implementar un sistema operativo de tiempo real\\ 
		\hline 
		RQX-PC 2 & Se debe garantizar el correcto funcionamiento de todos los periféricos utilizando los driver y bibliotecas provistas por el SO \\ 
		\hline 
		RQX-PC 3 & Se debe poder ejecutar programas que requieren la ejecución de múltiples hilos  \\ 
		\hline
		RQX-PC 4 & Se requiere la ejecución de programas de prueba para determinar las limitaciones en la ejecución de hilos \\ 
		\hline		
		\end{tabular}
		\caption{Tabla de requerimientos del prototipo tres}
		\label{tab:tdr3}
		\end{table}
\newpage		

		\section{Matriz de Riesgo para el Prototipo Tres} 
En la tabla~\ref{tab:riegos3} se muestra el resultado del análisis de la identificación de los riesgos asociados a los requerimientos del prototipo tres, ordenado por importancia y ocurrencia.
		\begin{table}[h!]
		\centering
		\begin{tabular}{ p{2.5cm} p{9cm} p{1.5cm} p{2cm} }
		\hline 
		\rowcolor[gray]{0.8} N\textordmasculine Req Asociados& Riegos Asociados & Severidad  & Ocurrencia \\
		\hline
		RQX-PC 1& Fallas en la generación de la imagen del SO para montar en el SoC & Crítica       & Probable \\
		\hline				
				& Falta de documentación de apoyo para la implentación del SO bajo licencias de Software Libre & Crítico & Ocasional\\	
		\hline				
				 & Limitaciones para la ejecución de SO de tiempo real & Crítico & Ocasional\\	
 		
 		\hline	
		RQX-PC 2 	& Bibliotecas necesarias inexistentes o con fallos& Crítico & Ocasional\\	
		
		\hline				
 					 & Drivers necesarios inexistentes o con fallos  & Severo  &  Ocasional\\ 
		\hline	
 		RQX-PC 3	&Problemas en la compilación e instalación de las herramientas de desarrollo de software& Severo  &  Ocasional\\ 
		\hline
		RQX-PB 4 & Problemas con el compilador para la generación del binario del programa de prueba  & Crítico&  Ocasional\\
		\hline
		\end{tabular}
		\caption{Tabla de riegos vs. Requerimientos del prototipo tres}
		\label{tab:riegos3}
		\end{table}

\newpage
En la tabla~\ref{tab:planificación3} podemos ver los resultados de la planificación de riesgos.
 		\begin{table}[h!]
		\centering
		\begin{tabular}{ p{4cm} p{4cm} p{4cm} p{3cm} }
		\hline 
		\rowcolor[gray]{0.8} Riesgo & Consecuencia & Estrategia preventiva & Estrategia de contingencia\\
		\hline
		Fallas en la generación de la imagen del SO para ser montada en el SoC &Retraso en los tiempos de implementación.& Análisis previo de guías de resolución de problemas y preguntas frecuentes & Revisión de código y consulta a los desarrolladores\\		 
		\hline
		Falta de documentación de apoyo para la implentación del SO bajo licencias de Software Libre& Perdida de tiempo en realizar la implementación del SO & Revisión previa de la documentación existente para la elección del
SO & Dedicación de mayor tiempo para la implementación\\ 
		\hline
		 Limitaciones para la ejecución de SO de tiempo real & Inviabilidad para la ejecución
de aplicaciones en tiempo real & Análisis de los RTOS &Modificación del SO para el cumplimiento de las capacidades de ejecución en tiempo real\\
		\hline
		Bibliotecas necesarias inexistentes o con fallos& Imposibilidad de desarrollo de aplicaciones para ejecutar sobre el SO de tiempo real& Análisis previo de las bibliotecas disponibles y revisión de posibles bugs informados& Desarrollo de las bibliotecas necesarias o corrección de errores\\			
		\hline
		Drivers necesarios inexistentes o con fallos & Imposibilidad de utilización
de los dispositivos necesarios&Análisis previo de los driver de los dispositivos a utilizar& Desarrollo de los drivers necesarios\\		
		\hline
		 Problemas en la compilación e instalación de las herramientas de desarrollo de software & Imposibilidad de desarrollo de aplicaciones que se ejecuten sobre el SO& Verificar la posibilidad de ejecución de las herramientas
en diferentes SO & Utilización de las herramientas en alguno de los SO compatibles\\
		\hline
		 Problemas con el compilador para la generación del binario del programa de prueba& Imposibilidad de medir las limitaciones en la ejecución de hilos & Consulta previa a la documentación del SO y guías de preguntas frecuentes & Depuración de código de prueba\\
		\hline
		\end{tabular}
		\caption{Planificación de riesgos del prototipo tres}
		\label{tab:planificación3}
		\end{table}

\newpage
		
		\section{Descripción de la Estructura del Prototipo Tres}
		La figura~\ref{fig:ecos} muestra el diagrama de despliegue del prototipo tres. Este prototipo utiliza como base al prototipo dos pero se ejecutan 
		aplicaciones compiladas bajo ecOS.
		
		\begin{figure}[h!]
 		\begin{center}
  		\includegraphics[width=1\textwidth,keepaspectratio=true]{./images/ecos}
  		\caption{Diagrama de despliegue del prototipo tres}
  		\label{fig:ecos} 
 		\end{center}
		\end{figure}
	
		\section{Implementación}	
		
		La implementación de sistema operativo de tiempo real ecOS se realiza en una serie de etapas detalladas a continuación. 
		\begin {itemize}
		\item Descarga del código fuente del proyecto
		\item Descarga de la aplicación de configuración de ecOS
		\item Instalación de dependencias de compilación
		\item Compilación de la herramienta de configuración
		\item Configuración del sistema
		\item Contrucción del sistema 
		\item Compilación de programas de prueba
		\item Ejecución de aplicaciones bajo ecOS
 		\end {itemize}
En el anexo~\ref{app:apendice4} se encuentra una guía mas detallada de los pasos antes nombrados para la implemetación sobre el kit de desarrollo.


		%Consulte la Guía de ecOS en el Anexo ~\ref{app:apendice4} para una comprensión más profunda de los pasos a seguir en la implementación sobre la		placa de desarrollo S3ADSP1800A.
		
\newpage
		%%%%%%%%%%%%%%%%%%%%%%%%%%%%TESTING TRES
		\section{Testing}

Para llevar a cabo el testing del prototipo tres se planificó una serie de casos de prueba para validar los requerimiento planteados.
El principal requisito para el comienzo de las pruebas fue implementar el sistema operativo de tiempo real eCos sobre el SoC, para lo cual necesitamos tener las herramientas de compilación funcionando correctamente. En una primera
instancia fue necesario disponer de algún medio de respuesta de parte del sistema operativo, razón por la cual se realizó un caso de prueba que garantice el funcionamiento  de la comunicación SO - SoC - PC mediante UART (véase tabla~\ref{tab:cp9}). En una segunda instancia de prueba, se probó la capacidad del microprocesador softcore OpenRisc para ejecutar un programa de prueba realizado en lenguaje C que ejecute hilos, mediante un sistema operativo que maneja multihilo (véase tabla~\ref{tab:cp10}).

		\begin{table}[h!]
		\centering
		\begin{tabular}{ p{5cm} p{10cm}  }
		\hline 
		\rowcolor[gray]{0.8}  Caso de Prueba &  Interactuar correctamente con la interfaz UART\\
		\hline 
		Código de Requerimiento & RQX-PC 2\\ 
		\hline 
		Requerimiento & Se debe garantizar el correcto funcionamiento de todos los periféricos utilizando los driver y bibliotecas provistas por el SO \\ 
		\hline 
		Código de Testing & T009\\ 
		\hline
		Propósito &  Comprobación del correcto funcionamiento del componente UART\\
		\hline
		Realizado Por & Gomez, Pablo ; Lovaisa, Valeria \\
		\hline	
		Entorno de Ejecución & SO - ecOS\\
		\hline
		Precondiciones & \begin {itemize}
							\item Instanciación en la FPGA del SoC
							\item Ejecución de una terminal serie en la PC 
 							\item Comunicación establecida entre la placa de desarrollo y la PC
							\item Herramientas de compilación cruzada funcionando correctamente
							\item Sistema operativo correctamente instalado
							\item Programas de prueba compilados correctamente
							\end {itemize} \\
		\hline
		Secuencia de Ejecución &  Ejecución de la aplicación de prueba \verb|hello.elf|\\
		\hline
		Postcondiciones &  El programa de prueba ejecutándose sobre el sistema operativo ecOS, lo que se manifiesta en la impresión de pantalla en la
		terminal serie de la PC\\
		\hline
 		\multicolumn{2}{>{\columncolor[gray]{.8}}c}{Resultados}\\
		\hline
		Resultados Esperados & Poder observar por la terminal serial una cadena de string \\
		\hline	
		Resultados Obtenidos & Al ejecutarse el programa de prueba pudo observarse la cadena string esperada "Hello, ecOS World"\\
		\hline
		\end{tabular}
		\caption{Caso de prueba T009}
		\label{tab:cp9}
		\end{table}

\newpage			
		\begin{table}[h!]
		\centering
		\begin{tabular}{ p{5cm} p{10cm}  }
		\hline 
		\rowcolor[gray]{0.8}  Caso de Prueba& Ejecución de hilos\\
		\hline 
		Código de Requerimiento & RQX-PC 3, RQX-PC 4\\ 
		\hline 
		Requerimiento  	& Se debe poder ejecutar programas que requiere la ejecución de múltiples hilos\\
		\hline 
						& Problemas con el compilador para la generación del binario del programa de prueba\\
		\hline 
		Código de Testing & T010\\ 
		\hline
		Propósito &  Compilación y ejecución de un programa de prueba que ejecute hilos\\
		\hline
		Realizado Por & Gomez , Pablo\\
		\hline	
		Entorno de Ejecución & SO - ecOS \\
		\hline
		Precondiciones &    \begin {itemize}
							\item Instanciación en la FPGA del SoC
							\item Ejecución de una terminal serie en la PC
							\item Comunicación establecida entre la placa de desarrollo y la PC
							\item Herramientas de compilación cruzada funcionando correctamente
							\item Sistema operativo correctamenta instalado
							\item Programas de prueba compilados correctamente
							\end {itemize}
		\\
		\hline
		Secuencia de Ejecución & Ejecución de la aplicación de prueba \verb|twothreads.elf| \\
		\hline
		Postcondiciones & Aplicación ejecutándose sobre el sistema operativo \\
		\hline
 		\multicolumn{2}{>{\columncolor[gray]{.8}}c}{Resultados}\\
		\hline
		Resultados Esperados & Poder observar por la terminal serie el resultado de la ejecución de la aplicación\\
		\hline	
		Resultados Obtenidos & Al ejecutarse el programa de prueba de hilos pudo observarse la salida mostrada en la código~\ref{lst:salhilos} \\
		\hline
		\end{tabular}
		\caption{Caso de prueba T010}
		\label{tab:cp10}
		\end{table}

\newpage
\begin{lstlisting}[frame=single,caption={Salida de la ejecución del programa de prueba twothreads},label={lst:salhilos}]
Entering twothreads' cyg_user_start () function
Beginning execution; thread data is 0
Beginning execution; thread data is 1
Thead 0: and now a delay of 239 clock ticks
Thead 1: and now a delay of 230 clock ticks
Thead 1: and now a delay of 221 clock ticks
Thead 0: and now a delay of 214 clock ticks
Thead 1: and now a delay of 224 clock ticks
Thead 0: and now a delay of 243 clock ticks
Thead 1: and now a delay of 210 clock ticks
Thead 0: and now a delay of 224 clock ticks
Thead 1: and now a delay of 207 clock ticks
Thead 0: and now a delay of 244 clock ticks
\end{lstlisting}

		\section{Conclusión}
		
		Con el prototipo tres se verificó el correcto funcionamiento y soporte para sistemas de tiempo real sobre el microprocesador OpenRISC 1200. Mediante
		la ejecución de los casos de prueba se probó, en primera instancia, que los periféricos funcionaran correctamente indicando a su vez que el sistema
		operativo se encuentra corriendo en el núcleo como se esperaba. Se comprobó además, la capacidad del kernel de ecOS para ejecutar hilos a nivel de software. 
		Todo lo anteriormente dicho permitió corroborar que mediante la implementación de un microprocesador Softcore OpenRISC en conjunto con un sistema
		operativo, puede darse una solución integral, efectiva para aplicaciones de tiempo real de código abierto.
		
\newpage
\chapter{Prototipo Cuatro : Implementación del SoC ORPSoC en FPGA con Sistema Operativos Linux} \label{chap:proto4}
		\section{Introducción}
		El núcleo Linux, combinado con un conjunto utilidades de software libre, puede ajustarse al espacio limitado de hardware 
	    de los sistemas embedidos. Una instalación típica de un Linux embebido ocupa en promedio 2 MB. En el prototipo cuatro se utilizó una versión
	    reducida del núcleo de Linux y herramientas provistas por el paquete BusyBox. 
		
		\section{Requerimientos del Prototipo}
Para la implementación del prototipo cuatro se tomaron los requerimientos, expuestos en la tabla~\ref{tab:tdr4}.		

		\begin{table}[h!]
		\centering	
		\begin{tabular}{ p{2.5cm} p{14.5cm} }
		\hline 
		\rowcolor[gray]{0.8} N\textordmasculine Req  & Descripción\\
		\hline                             	RQX-PD 1 & El prototipo debe implementar el sistema operativos Linux\\ 
		\hline  						RQX-PD 2 & Se requiere la instalación de un gestor de arraque que posibilite el inicio del sistema operativo\\ 
		\hline 						RQX-PD 3 & Se debe poder grabar como aplicación firmware el gestor de arranque en la memoria flash SPI de la placa de desarrollo\\
		\hline 						RQX-PD 4 & El prototipo debe tener capacidad de ejecución de aplicaciones sobre el sistema operativo Linux\\
		\hline 
		\end{tabular}
		\caption{Tabla de requerimientos del prototipo cuatro}
		\label{tab:riegos4}
		\end{table}
		
		\newpage
		\section{Matriz de Riesgo para el Prototipo Cuatro} 
En la tabla~\ref{tab:riegos4} Tabla de riegos vs. Requerimientos del prototipo cuatro se muestra el resultado del análisis de la identificación de los riesgos asociados a los requerimientos, ordenado por importancia y ocurrencia.

		\begin{table}[h!]
		\centering
		\begin{tabular}{ p{2.5cm} p{9cm} p{1.5cm} p{2cm} }
		\hline 
		\rowcolor[gray]{0.8} N\textordmasculine Req Asociados  & Riegos Asociados & Severidad  & Ocurrencia \\
		\hline RQX-PC 1 & Fallas en la generación de la imagen del SO para ser montada en el SoC & Crítica       & Probable \\
		\hline			& Falta de documentación de apoyo para la implentación del SO bajo licencias de Software Libre & Crítico & Ocasional\\	
		\hline			& Limitaciones para la ejecución de SO de tiempo real & Crítico & Ocasional\\	
 		\hline RQX-PC 2 & bibliotecas necesarias inexistentes o con fallos& Crítico & Ocasional\\	
		\hline			& Drivers necesarios inexistentes o con fallos  & Severo  &  Ocasional\\ 
		\hline RQX-PC 3	& Problemas en la compilación e instalación de las herramientas de desarrollo de software& Severo  &  Ocasional\\ 
		\hline RQX-PB 4 & Problemas con el compilador para la generación del binario del programa de prueba  & Crítico&  Ocasional\\
		\hline
		\end{tabular}
		\caption{Tabla de riegos vs. Requerimientos del prototipo cuatro}
		\label{tab:riegos4}
		\end{table}

En la tabla~\ref{tab:planificación4} podemos ver los resultados de la planificación de riesgos.

		\newpage
 		\begin{table}[h!]
		\centering
		\begin{tabular}{ p{4cm} p{4cm} p{4cm} p{3cm} }
		\hline 
		\rowcolor[gray]{0.8} Riesgo & Consecuencia & Estrategia preventiva & Estrategia de contingencia\\
		\hline Errores en la configuración de compilación del kernel para el SoC utilizado & 
		       Retraso en los tiempos de implementación & 
		       Análisis previo de archivos de configuración del SoC seleccionado o similares & 
		       Revisión y correción de archivos de configuración\\
		\hline Fallas en la generación de la imagen del SO y/o el gestor arranque para ser montada en el SoC & 
		       Retraso en los tiempos de implementación & 
		       Análisis previo de guías de resolución de problemas y preguntas frecuentes & 
		       Revisión de código y consulta a los desarrolladores\\
		\hline Falta de documentación de apoyo para la implentación del SO y/o gestor de arranque para la arquitectura OpenRISC & 
		       Demoras en la implementación del SO & 
		       Revisión previa de la documentación existente y guías de resolución de problemas & 
		       Mayor inversión de tiempo para la implementación\\
		\hline Falta de soporte del SO Linux y/o el gestor de arranque para su implementación en el hardware disponible & 
			   Fallos en la ejecución del sistema operativo & 
		       Análisis previo de limitaciones en la documentación de los desarrolladores y guías de resolución de problemas & 
		       Revisión y desarrollo de código del SO necesario para proveer soporte al hardware\\
		\hline Bibliotecas necesarias inexistentes o con fallos & 
			   Imposibilidad de desarrollo de aplicaciones para ejecutar sobre el SO &
			   Análisis previo de las bibliotecas disponibles y revisión de posibles bugs informados SO & 
			   Desarrollo de las bibliotecas necesarias o corrección de errores\\
		\hline Drivers necesarios inexistentes o con fallos & 
			   Imposibilidad de utilización de los dispositivos necesarios & 
			   Análisis previo de los driver de los dispositivos a utilizar & 
			   Desarrollo de los drivers necesarios\\		
		\hline Problemas en la compilación e instalación de las herramientas de desarrollo de software & 
			   Imposibilidad de desarrollo de aplicaciones que se ejecuten sobre Linux Embebido & 
			   Verificación previa de la documentación de los desarrolladores y guías de resolución de problemas & 
			   Revisión de codigo y corrección de errores\\
		\hline
		\end{tabular}
		\caption{Planificación de riesgos del prototipo cuatro}
		\label{tab:planificación4}
		\end{table}

		\newpage
		
		\section{Descripción de la Estructura del Prototipo Cuatro}
		Se muestra a continuación figura~\ref{fig:proto4} el diagrama de despliegue del prototipo cuatro. Se utiliza como base nuevamente el prototipo
		dos con un sistema operativo linux ejecutándose en él.

		\begin{figure}[h!]
 		\begin{center}
  		\includegraphics[width=1\textwidth,keepaspectratio=true]{./images/proto4}
  		\caption{Diagrama de despliegue del prototipo cuatro}
  		\label{fig:proto4} 
 		\end{center}
		\end{figure}
	
		\section{Implementación}	
		
La implementación del sistema operativo Linux se realiza en una serie de etapas detalladas a continuación. 
		\begin {itemize}
		\item Descarga del código fuente del proyecto
		\item Configuración de las variables de entorno
		\item Configuración del sistema
		\item Confuguracion del kernel en el simulador or1ksim
		\item Instalación sobre la placa 
		\end {itemize}

Para mas detalle de estas etapas en la implementación sobre el kit de desarrollo de Xilinx puede ver el anexo~\ref{app:apendice5}
 
		%Consulte la Guía de Linux en el Anexo ~\ref{app:apendice5} para una comprensión más profunda de los pasos a seguir en la implementación sobre la		placa de desarrollo S3ADSP1800A.
		\newpage
				
		\section{Testing}	
Para el Prototipo cuatro se planificaron una serie de casos de prueba para validar los requerimiento planteados. El primer caso de prueba que se realizo fue para validar el correcto funcionamiento del gestor de arranque uBoot, mediante la carga de un programa de prueba realizado en lenguaje C (véase tabla~\ref{tab:cp11}). El siguiente caso se realizo cargando el gestor de arranque directamente desde la memoria FLASH SPI (véase tabla~\ref{tab:cp12}). Luego para validar el funcionamiento de la imagen creada del kernel de Linux en una arquitectura OpenRISC, se lo ejecuto sobre el simulador de la arquitectura OpenRISC conocido como \verb|or1ksim| (véase tabla~\ref{tab:cp13}). Finalmente, el ultimo caso de prueba que se realizo fue para validar la implemento del Kernel de Linux sobre el el kit de desarrollo S3ADSP1800A (véase tabla~\ref{tab:cp14}).

		
		\begin{table}[h!]
		\centering
		\begin{tabular}{ p{5cm} p{10cm}  }
		\hline 
		\rowcolor[gray]{0.8} 	 Caso de Prueba & Carga de programa mediante el gestor de arranque UBoot\\
		\hline  		Código de Requerimiento & RQX-PD 2\\ 
		\hline  				  Requerimiento & Se requiere la instalación de un gestor de arraque que posibilite el inicio del sistema operativo\\
		\hline 				  Código de Testing & T011\\ 
		\hline 						  Propósito & Comprobación del correcto funcionamiento del gestor de arranque\\
		\hline					  Realizado Por & Lovaisa, Valeria \\
		\hline	 		   Entorno de Ejecución & Bare Metal\\
		\hline		   		   	 Precondiciones & \begin {itemize}
												  \item Instanciación en la FPGA del SoC ORPSoC
												  \item Ejecución de una terminal serie en la PC 
 												  \item Comunicación establecida entre la placa de desarrollo y la PC mediante Serial y eth
 												  \item Servidor TFTP funcionando correctamente sobre el SO host
												  \item Herramientas de compilación cruzada funcionando correctamente
												  \item Gestor de arranque y aplicación de prueba compilados correctamente
												  \item Imagen de arranque sin errores
												  \end {itemize} \\
		\hline			 Secuencia de Ejecución &  Ejecutación de la aplicación de prueba \verb|numprim.elf| a través de Uboot\\
		\hline					Postcondiciones &  El gestor de arranque inicia correctamente el programa de prueba, lo que se manifiesta en la impresión de la salida
		del gestor de arranque y el programa prueba en la terminal serie de la PC\\
		\hline	\multicolumn{2}{>{\columncolor[gray]{.8}}c}{Resultados}\\
		\hline			   Resultados Esperados & Poder observar por la terminal serial las salidas de los programas \\
		\hline	 		   Resultados Obtenidos & Al ejecutarse el programa de prueba pudo observarse la salida mostrada en la código~\ref{lst:salidauboot}\\
		\hline	
		\end{tabular}
		\caption{Caso de prueba T011}
		\label{tab:cp11}
		\end{table}

\newpage		
\begin{lstlisting}[frame=single,caption={Salida de la ejecución del programa de prueba cargado por uboot},label={lst:salidauboot}]
U-Boot 2013.04-g9181a73-dirty (Oct 30 2013 - 19:58:28)

CPU:   OpenRISC-1200 (rev 8) @ 25 MHz
       D-Cache: 4096 bytes, 16 bytes/line, 1 way(s)
       I-Cache: 512 bytes, 16 bytes/line, 1 way(s)
       DMMU: 64 sets, 1 way(s)
       IMMU: 64 sets, 1 way(s)
       MAC unit: yes
       Debug unit: yes
       Performance counters: no
       Power management: no
       Interrupt controller: yes
       Timer: yes
       Custom unit(s): no
       Supported instructions:
           ORBIS32: yes
           ORBIS64: no
           ORFPX32: no
           ORFPX64: no
           Hardware multiplier: no
           Hardware divider: no
BOARD: s3adsp1800a
SF: Detected M25P64 with page size 64 KiB, total 8 MiB
NET:   ETHOC-0
===> ftd tftp u hello- .ub

ethoc
Using ETHOC-0 device
TFTP from server 192.168.1.98; our IP address is 192.168.1.32
Filename 'hello.ub'.
Load address: 0x100000

Loading: *####
	1.4 MiB/s
done 
Bytes transtferred = 7908(136f8 hex)
==> set boot 0x100000
## Booting kernel from Legacy Image at 00100000 ...
   Image Name:   hello
   Image Type:   OpenRISC 1000 Linux Standalone Program (uncompressed)
   Data Size:    79544 Bytes = 77.7 KiB
   Load Address: 00000000
   Entry Point:  00000100
   Verifying Checksum ... OK
   Loading Standalone Program ... OK
Cantidad de numero primos entre 1 y 2000 : 303

\end{lstlisting}
		
		
		\newpage
		\begin{table}[h!]
		\centering
		\begin{tabular}{ p{5cm} p{10cm}  }
		\hline 
		\rowcolor[gray]{0.8} 	 Caso de Prueba & Inicio del gestor de arranque desde la memoria Flash SPI\\
		\hline  		Código de Requerimiento & RQX-PD 3\\ 
		\hline  				  Requerimiento & Se debe poder grabar como aplicación firmware el gestor de arranque en la memoria flash SPI de la placa de desarrollo\\
		\hline 				  Código de Testing & T012\\ 
		\hline 						  Propósito & Verificación del inicio del gestor de arranque almacenado en la Flash SPI \\
		\hline					  Realizado Por & Gomez , Pablo \\
		\hline	 		   Entorno de Ejecución & Bare Metal\\
		\hline		   		   	 Precondiciones & \begin {itemize}
												  \item Instanciación en la FPGA del SoC ORPSoC
												  \item Ejecución de una terminal serie en la PC 
 												  \item Comunicación establecida entre la placa de desarrollo y la PC mediante Serial
 												  \item Herramientas de compilación cruzada funcionando correctamente
												  \item Gestor de arranque compilado correctamente
												  \item Imagen del gestor de arranque almacenada en la memoria Flash SPI de la placa de desarrollo
												  \end {itemize} \\
		\hline			 Secuencia de Ejecución &  Arranque inicial o reset del hardware\\
		\hline					Postcondiciones &  El gestor de arranque inicia correctamente, lo que se manifiesta en la impresión de la salida
		del gestor de arranque en la terminal serie de la PC\\
		\hline	\multicolumn{2}{>{\columncolor[gray]{.8}}c}{Resultados}\\
		\hline			   Resultados Esperados & Poder observar por la terminal serial las salida del Uboot \\
		\hline	 		   Resultados Obtenidos & Al ejecutarse el programa de prueba no pudo observarse la salida esperada\\
		\hline	
		\end{tabular}
		\caption{Caso de prueba T012}
		\label{tab:cp12}
		\end{table}	
\newpage

		\begin{table}[h!]
		\centering
		\begin{tabular}{ p{5cm} p{10cm}  }
		\hline 
		\rowcolor[gray]{0.8} 	 Caso de Prueba & Simulación del SO Linux Embebido en or1ksim\\
		\hline  		Código de Requerimiento & RQX-PD 1\\ 
		\hline  				  Requerimiento & El prototipo debe implementar el sistema operativos Linux\\
		\hline 				  Código de Testing & T013\\ 
		\hline 						  Propósito & Ejecutar correctamente un Sistema Operativo Linux Embebido en simulador de la arquitectura\\
		\hline					  Realizado Por & Lovaisa, Valeria \\
		\hline	 		   Entorno de Ejecución & Bare Metal\\
		\hline		   		   	 Precondiciones & \begin {itemize}
												  \item Herramienta de simulación or1ksim correctamente instalada
												  \item Ejecución de una terminal serie en la PC 
 												  \item Comunicación establecida entre el simulador y la PC
 												  \item Herramientas de compilación cruzada funcionando correctamente
 												  \item Configuración del kernel adaptada al SoC para simulación
												  \item Imágen del Sistema Operativo Linux compilada correctamente 
												  \end {itemize}\\
		\hline			 Secuencia de Ejecución & Inicio del Sistema Operativo \\
		\hline					Postcondiciones & Se inicia correctamente el Sistema Operativo, lo que se manifiesta en la impresión de la salida
		de la secuencia de arranque de linux en la terminal serie de la PC\\
		\hline	\multicolumn{2}{>{\columncolor[gray]{.8}}c}{Resultados}\\
		\hline			   Resultados Esperados & Lograr observar por la terminal serial la secuencia de arranque del Sistema Operativo\\
		\hline	 		   Resultados Obtenidos & Al ejecutarse el Kernel del SO pudo observarse la salida esperada, mostrada en la código~\ref{lst:sallinux}\\
		\hline	
		\end{tabular}
		\caption{Caso de prueba T013}
		\label{tab:cp13}
		\end{table}

\begin{lstlisting}[frame=single,caption={Salida de la secuencia de inicio de linux embebido},label={lst:sallinux}]
...
Freeing unused Kernel memory: 1416k freed

init started: BusuyBox v1.19.0.giy (2013 - 10 - 25 08:10:12 CET)
Configuring loopbak device

Please press Enter to activate this console.
#ls
bin	  dev   etc   init   mnt   proc   root   sbin   sys   usr   var
\end{lstlisting}		

\newpage 

		\begin{table}[h!]
		\centering
		\begin{tabular}{ p{5cm} p{10cm}  }
		\hline 
		\rowcolor[gray]{0.8} 	 Caso de Prueba & Implementación del SO Linux Embebido\\
		\hline  		Código de Requerimiento & RQX-PD 1\\ 
		\hline  				  Requerimiento & El prototipo debe implementar el Sistema Operativos Linux\\
		\hline 				  Código de Testing & T013\\ 
		\hline 						  Propósito & Implementar correctamente un Sistema Operativo Linux Embebido sobre un SoC ORPSoC\\
		\hline					  Realizado Por & Lovaisa, Valeria \\
		\hline	 		   Entorno de Ejecución & Bare Metal\\
		\hline		   		   	 Precondiciones & \begin {itemize}
												  \item Instanciación en la FPGA del SoC ORPSoC
												  \item Ejecución de una terminal serie en la PC 
 												  \item Comunicación establecida entre la placa de desarrollo y la PC
 												  \item Herramientas de compilación cruzada funcionando correctamente
 												  \item Configuración del kernel adaptada al SoC
												  \item Imágen del Sistema Operativo Linux compilada correctamente 
												  \end {itemize}\\
		\hline			 Secuencia de Ejecución & Inicio del Sistema Operativo \\
		\hline					Postcondiciones & Se inicia correctamente el Sistema Operativo, lo que se manifiesta en la impresión de la salida
		de la secuencia de arranque de linux en la terminal serie de la PC\\
		\hline	\multicolumn{2}{>{\columncolor[gray]{.8}}c}{Resultados}\\
		\hline			   Resultados Esperados & Lograr observar por la terminal serial la secuencia de arranque del Sistema Operativo\\
		\hline	 		   Resultados Obtenidos & Al ejecutarse el Kernel del SO pudo observarse la salida mostrada en la código~\ref{lst:sallinux1}\\
		\hline	
		\end{tabular}
		\caption{Caso de prueba T014}
		\label{tab:cp14}
		\end{table}
	
\newpage 
		
\begin{lstlisting}[frame=single,caption={Salida de la secuencia de inicio de linux embebido},label={lst:sallinux1}]
[ 0.000000] Compiled-in FDT at c02a8820 
[ 0.000000] Linux version 3.9.0-dirty 
			(cudar73@cudar73-desktop) (gcc version 4.5.1-or323 
[ 0.000000] CPU: OpenRISC-12 (revision 8) @25 MHz 
[ 0.000000] -- dcache: 4096 bytes total, 16 bytes/line, 1 way(s) 
[ 0.000000] -- icache: 512 bytes total, 16 bytes/line, 1 way(s) 
[ 0.000000] -- dmmu: 64 entries, 1 way(s) 
[ 0.000000] -- immu: 64 entries, 1 way(s) 
[ 0.000000] -- additional features: 
[ 0.000000] -- debug unit 
[ 0.000000] -- PIC 
[ 0.000000] -- timer 
[ 0.000000] setup_memory: Memory: 0x0-0x8000000 
[ 0.000000] Reserved - 0x07ffda90-0x00002570 
[ 0.000000] Setting up paging and PTEs. 
[ 0.000000] map_ram: Memory: 0x0-0x8000000 
[ 0.000000] itlb_miss_handler c0002198 
[ 0.000000] dtlb_miss_handler c0002000 
[ 0.000000] OpenRISC Linux -- http://openrisc.net 
[ 0.000000] Built 1 zonelists in Zone order, 
			mobility grouping on. Total pages: 16312 
[ 0.000000] Kernel command line: console=uart,mmio,0x90000000,115200 
[ 0.000000] Early serial console at MMIO 0x90000000 (options '115200') 
[ 0.000000] bootconsole [uart0] enabled 
[ 0.000000] PID hash table entries: 512 (order: -2, 2048 bytes) 
[ 0.000000] Dentry cache hash table entries: 16384 (order: 3, 65536 bytes) 
[ 0.000000] Inode-cache hash table entries: 8192 (order: 2, 32768 bytes) 
[ 0.000000] Memory: 126088k/131072k available 
			(2270k kernel code, 4984k reserved, 345k ) 
[ 0.000000] mem_init_done ........................................... 
[ 0.000000] NR_IRQS:32 nr_irqs:32 0 
[ 0.000000] 50.00 BogoMIPS (lpj=250000) 
[ 0.000000] pid_max: default: 32768 minimum: 301 
[ 0.020000] Mount-cache hash table entries: 1024 
[ 0.390000] devtmpfs: initialized 
[ 0.460000] NET: Registered protocol family 16 
[ 0.810000] Switching to clocksource openrisc_timer 
[ 0.960000] NET: Registered protocol family 2 
[ 1.040000] TCP established hash table entries: 1024 (order: 0, 8192 bytes) 
[ 1.050000] TCP bind hash table entries: 1024 (order: -1, 4096 bytes) 
[ 1.060000] TCP: Hash tables configured (established 1024 bind 1024) 
[ 1.110000] TCP: reno registered 
[ 1.110000] UDP hash table entries: 512 (order: 0, 8192 bytes) 
[ 1.130000] UDP-Lite hash table entries: 512 (order: 0, 8192 bytes) 
[ 1.160000] NET: Registered protocol family 1 
[ 1.240000] RPC: Registered named UNIX socket transport module. 
[ 1.250000] RPC: Registered udp transport module. 
[ 1.250000] RPC: Registered tcp transport module. 
[ 1.260000] RPC: Registered tcp NFSv4.1 backchannel transport module. 
[ 1.300000] Unable to handle kernel NULL pointer 
			dereference at virtual address 0x000002 
\end{lstlisting}	
		
%		\begin{table}[h!]
%		\centering
%		\begin{tabular}{ p{5cm} p{10cm}  }
%		\hline 
%		\rowcolor[gray]{0.8} 	 Caso de Prueba & Ejecución de aplicaciones de prueba sobre Linux\\
%		\hline  		Código de Requerimiento & RQX-PD 4\\ 
%		\hline  				  Requerimiento & El prototipo debe tener capacidad de ejecución y depuración de aplicaciones sobre el sistema operativo Linux\\
%		\hline 				  Código de Testing & T014\\ 
%		\hline 						  Propósito & Ejecución de aplicaciones de prueba sobre entorno Linux\\
%		\hline					  Realizado Por & Gomez , Pablo \\
%		\hline	 		   Entorno de Ejecución & Linux\\
%		\hline		   		   	 Precondiciones & \begin {itemize}
%												  \item Instanciación en la FPGA del SoC ORPSoC
%												  \item Ejecución de una terminal serie en la PC 
 %												  \item Comunicación establecida entre la placa de desarrollo y la PC
 %												  \item Herramientas de compilación cruzada para su ejecución sobre Linux funcionando correctamente
%												  \item SO Linux ejecutándose correctamente sobre el SoC ORPSoC 
%												  \end {itemize} \\
%		\hline			 Secuencia de Ejecución & \begin {itemize}
%						 					 	  \item Inicio del gestor de arranque
%												  \item Inicio del Sistema Operativo
%												  \item Ejecución de la aplicación de prueba
%						 						  \end {itemize} \\
%		\hline					Postcondiciones &  Se ejecuta correctamente el programa de prueba, lo que se manifiesta en la impresión de la salida
%		del bash de linux en la terminal serie de la PC\\
%		\hline	\multicolumn{2}{>{\columncolor[gray]{.8}}c}{Resultados}\\
%		\hline			   Resultados Esperados & Poder observar por la terminal serial las salida de ejecución del programa de prueba \\
%		\hline	 		   Resultados Obtenidos & Al ejecutarse el programa de prueba no pudo observarse la salida esperada%mostrada en la Referencia ~\ref{lst:salidalinux}\\
%		\hline	
%%		\caption{Caso de prueba T014}
%		\end{table}

%\begin{lstlisting}[frame=single,caption={Salida de la ejecución del programa de prueba ejecutado en Linux Embebido},label={lst:salidalinux}]
%\# @ ACA VA EL RESULTADO DE CORRER UBOOT DESDE LA SPI
%\end{lstlisting}		
		
		\newpage
		\section{Conclusión}
		El prototipo cuatro ayudó a verificar la posibilidad de implementación de uno de los sistema operativo más conocidos de código abierto en Sistemas Embebidos.
		La posibilidad de obtener documentación de apoyo para la implementación de distintas soluciones que se ejecuten bajo este Kernel, facilita en gran
		medida el desarrollo de aplicaciones de mayor complejidad que las desarrolladas para sistemas operativos de tiempo real como lo es ecOS.   
		
		Se verificó el correcto funcionamiento del gestor de arranque uBoot, uno de los más conocidos y verificados. Este hecho posibilitó ejecutar desde el
		arranque del sistema la imagen de un programa de prueba con éxito. Asímismo, no se obtuvieron los resultados esperados al intentar cargar el gestor
		de arranque directamente desde la memoria FLASH SPI de la placa de desarrollo. Cabe destacar que inicialmente se utilizaron las herramientas
		provistas por el fabricante (Xilinx) para acceder a las memorias S33 y M25P64, que forman parte de los kits de desarrollo S3ADSP1800A, sin
		resultados favorables. Como resultado de la investigación ante este problema, se encontraron herramientas alternativas de código abierto que
		posibilitaron el acceso a las memorias mencionadas. Sin embargo, los intentos de cargar firmware directamente desde la memoria FLASH SPI no tuvieron
		resultados satisfactorios. Si se pretende obtener solución a este problema, será necesario realizar un análisis exhaustivo del módulo IP SPI
		implementado en el proyecto ORPSoC.
		
		El kernel de Linux pudo compilarse éxitosamente, con herramientas de compilación cruzada, sin mayores inconvenientes durante la construcción de su
		imágen y sus herramientas (Busybox). Luego, se ejecutó el kernel sobre el simulador \verb|or1ksim|. Pudo
		comprobarse así, el correcto funcionamiento de la imágen construida y su correcta implementación para una arquitectura OpenRISC. Sin embargo, al
		intentar ejecutar el Kernel sobre el kit de desarrollo S3ADSP1800A, surgieron inconvenientes respecto del acceso a memoria durante la secuencia de
		inicio del Sistema Operativo. En el análisis efectuado ante este problema se presentaron diversos impedimientos. El primero de ellos fue la escasa
		documentación que brinda el proveedor de la placa de desarrollo, que en ocaciones no funciona correctamente. En este trabajo se recurrió a
		distintas notas de aplicación proporcionadas por Xilinx sin lograr resultados satisfactorios. 
		
		Como alternativa que permita resolver este problema se recurrió a la documentación, foros y listas de correo de las comunidades Opensource. Es importante
		destacar que los tiempos de respuesta en este tipo de canales no se corresponden con los tiempos de ejecución de este trabajo y mucho menos con los
		tiempos de la industria. Finalmente, debe realizarse un profundo estudio sobre código del kernel que permita solucionar los problemas asociados a
		la adapatación del SO a la placa de desarrollo. Si lo mencionado anteriormente no logra resolver el problema se deberá realizar un estudio de
		depuración a nivel de hardware. Existen resultados satisfactorias que demuestan el correcto funcionamiento del SO en esta arquitectura sobre otras
		placas de desarrollo, tales como Atlys de Xilinx y DE-Nano de Terasic ~\cite{rte.se} ~\cite{denano}.
		
		