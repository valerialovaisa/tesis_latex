 \chapter{Conclusiones}

	\section{En cuanto a al hardware opensource}
	 
% Al día de hoy los diseños hardware de código abierto son una realidad palpable. Cualquiera puede descargarlos de la red y utilizarlos en sus
% diseños. Los componentes típicos de un sistema tales como un controlador USB, un microprocesador, un controlador de red, etc., tienen su alternativa
% en código abierto y son ya utilizados por empresas en productos comerciales, lo cual da una idea de su calidad. A medida que la cantidad de código
% abierto disponible aumenta, cada vez mas demostrará que es un enfoque valioso para el desarrollo de la tecnología . En el caso del hardware
% reconfigurable, se ha conseguido cerrar el ciclo completo de diseño en una máquina GNU/Linux, realizándose la compilación, simulación, síntesis y
% descarga en una FPGA. Para la compilación y simulación hemos empleado el Icarus Verilog junto con el GTKWAVE, ambos programas libres y para la
% síntesis el entorno ISE de Xilinx, ejecutado en una plataforma Linux. Se podrían realizar sintetizadores libres que generen un netlist en formato
% EDIF, pero actualmente no sería posible disponer de un entorno completamente libre puesto que los fabricantes no publican la información,
% considerada como secreto industrial. El primer paso para lograrlo sería la existencia de una OpenFPGA

	\section{En cuanto a los procesadores Soft Core} 
	
	
	
	\section{En cuanto a la implementación de SoC sobre FPGA} 
		
		
	\section{En cuanto a la implementación de sistemas operativos RT en arquitecturas sintetizables} 
		
	
	\section{En cuanto a la implementación Linux en arquitecturas sintetizables} 
		

%%%%%%%%%%%%%%%%%%%%%%%%%%%%%%%%%%%%%%555

	\section{Trabajos futuros}

La principal carencia de los sistemas que se han desarrollado es la imposibilidad de utilizar la caché de datos de los procesadores debido a la
ausencia de un mecanismo que garantice la coherencia de las cachés de los distintos procesadores. Actualmente se está trabajando en una solución a
este problema, que permitiría utilizar cachés de datos en el sistema aumentando considerablemente el rendimiento de éste. Una vez se puedan usar
cachés de datos se hará necesario también desarrollar nuevos controladores de memoria que soporten más canales XCL, ya que con los controladores
actuales sólo se podrían desarrollar sistemas con dos procesadores. Otra de las limitaciones de los sistemas que se han presentado es la gestión de
las interrupciones, que actualmente no están completamente soportadas. Se está trabajando en el desarrollo de un controlador de interrupciones que
permita manejarlas de forma más eficiente, interrumpiendo solamente a un procesador cada vez en lugar de a todos simultáneamente.

		\subsection{Desarrollo de nuevos módulos de hardware para su utilización en SoC}
		
		\subsection{Desarrollo de drivers para nuevos módulos}
		
		\subsection{Desarrollo de aplicaciones reales sobre sistemas embebidos basados en SoC sintetizables}
		
		
	
	 