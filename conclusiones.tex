 \chapter{Conclusiones}

	\section{En cuanto a al hardware opensource} 
%A día de hoy los diseños hardware de código abierto son una realidad palpable. Cualquiera puede descargarlos de la red y utilizarlos en sus diseños. Los componentes típicos de un sistema tales como un controlador USB, un microprocesador, un controlador de red, etc., tienen su alternativa en código abierto y son ya utilizados por empresas en productos comerciales, lo cual da una idea de su calidad
%A medida que la cantidad de código abierto disponible aumenta, cada vez mas demostrara que es un enfoque valioso para el desarrollo de la tecnología .

%En el caso del hardware recongurable, se ha conseguido cerrar el ciclo completo de diseño en una máquina GNU/Linux, realizándose la compilación, simulación, síntesis y descarga en una FPGA. Para la compilación y simulación hemos empleado el GHDL junto con el GTKWAVE, ambos programas libres y para la síntesis el entorno ISE de Xilinx, ejecutado a través de Wine

%Se podrían realizar sintetizadores libres que generen un netlist en formato EDIF, pero actualmente no sería posible disponer de un entorno completamente libre puesto que los fabricantes no publican la información, considerada como secreto industrial. El primer paso para lograrlo sería la existencia de una Open FPGA
	\section{En cuanto a los procesadores Soft Core} 
	
	
	\section{En cuanto a la implementación de SoC sobre FPGA} 
		
		
	\section{En cuanto a la implementación de sistemas operativos RT en arquitecturas sintetizables} 
		
	
	\section{En cuanto a la implementación Linux en arquitecturas sintetizables} 
		

	\section{Trabajos futuros}
		\subsection{Desarrollo de nuevos módulos de hardware para su utilización en SoC}
		
		\subsection{Desarrollo de drivers para nuevos módulos}
		
		\subsection{Desarrollo de aplicaciones reales sobre sistemas embebidos basados en SoC sintetizables}
		
		
	
	 