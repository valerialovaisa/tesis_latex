 \chapter{Conclusiones}

El objetivo general y los objetivos específicos planteados en la propuesta del presente proyecto integrador se cumplieron completamente y de forma
exitosa logrando mostrar el desarrollo del trabajo a través de la metodología propuesta, la cual se baso en el modelo de desarrollo en espiral con el
fin de identificar los riesgos y plantear soluciones en base a los mismos mediante la simulación, diseño de programas de prueba y construcción de
prototipos en cada ciclo de la espiral.
	 
	\section{En cuanto al hardware opensource} 

Del trabajo realizado sobre hardware sintetizable opensource durante el desarrollo de este trabajo se puedo analizar desde diversos enfoques sus
ventajas , desventajas y particularidades. Si bien hoy en día los diseños hardware de código abierto se encuentra creciendo en gran medida y son una
realidad palpable, cualquiera puede descargarlos de internet y utilizarlos en sus diseños. Los componentes típicos de un sistema tales como un
controlador USB, un microprocesador, un controlador de red, etc., tienen su alternativa en código abierto (como es el caso del microprocesador
OpenRISC) y son ya utilizados por empresas en productos comerciales, lo cual hace presuponer que la calidad de este tipo de diseños cumple al menos
con los estándares requeridos en productos que actualmente se encuentran en el mercado.

A medida que la cantidad de código abierto disponible aumenta, cada vez más demostrará que es un enfoque valioso para el desarrollo de la tecnología.
En el caso del hardware recongurable, se ha conseguido cerrar el ciclo completo de diseño en una máquina GNU/Linux, realizándose la compilación,
simulación, síntesis y descarga en una FPGA. Para la compilación y simulación de este trabajo hemos empleado el Icarus Verilog junto con el GTKWAVE,
ambos programas libres y para la síntesis el entorno ISE de Xilinx, ejecutado sobre una plataforma Linux. Se podrían realizar sintetizadores libres
que generen un netlist en formato EDIF, pero actualmente no sería posible disponer de un entorno completamente libre puesto que los fabricantes no
publican la información, considerada como secreto industrial. El primer paso para lograrlo sería la existencia de una ``OpenFPGA'' con un descripción
opensource que posibilite evitar la utilización de las herramientas de síntesis privativas de los fabricantes de FPGAs.

	\section{En cuanto a los procesadores softcore} 
	
Respecto de los microprocesadores softcore se analizaron, durante el desarrollo de este trabajo, diversas alternativas de implentención de las cuales
la mayoría son desarrollos de los principales fabricantes de FPGA. Si bien la utilización de estas implementaciones favorece al desarrollo con
herramientas confiables con soporte provisto por el fabricante, es sin dudas un limitante en cuanto a la madurez de las herramientas opensource.

Desde el punto de vista de las capacidades de los microprocesadores softcore evaluados en este trabajo puede concluirse que en general poseen
capacidades similares con mayor o menor nivel de configuración y posibilidades de expansión con el desarrollo de nuevos módulos. El analisis de
rendimiento del micro OpenRISC realizado en este trabajo cuyos resultados fueron expuestos en el Capitulo ~\ref {chap:resultados} permite posicionar
este microprocesador entre otros similares capacidades.

Puede decirse en base a lo expuesto anteriormente que las implementaciones de microprocesadores softcore son una solución viable para problemas
industriales (control, comunicaciones, DSP, PLCs) e inclusive puede utilizarse con fines academicos para el estudio de sistemas embebidos,
microprocesadores, diseño e implementación de hardware reconfigurable, entre otros.

	\section{En cuanto a la implementación de SoC sobre FPGA} 

Fueron analizados para este trabajo dos implementaciones de sistemas en chip con núcleo OpenRISC. El proyecto MinSoC mostró capacidades acotadas en
aplicaciones \textit{bare-metal} de pequeña escala como puede ser un control PID simple o algún dispositivo que no requiera de gran capacidad de
cálculo. Los módulos UART y Ethernet le brindan la posibilidad de comunicarse en sistemas de mayor envergadura mediante diversos protocolos como
puede ser Modbus o algún otro protocolo opensource. La capacidad de expansión del proyecto se ve limitada por la necesidad de desarrollar programas
especiales para cada uno de los nuevos módulos que se utilicen ya que no se cuenta con un sistema operativo con drivers para tal fin.

Para aplicaciones con mayores exigencias de procesamiento y posibilidad de ejecución de un sistema operativo se analizó en este trabajo el sistema en
chip ORPSoC de Opencores. De mayores prestaciones que el proyecto MinSoC y soportado por gran cantidad de plataformas reconfigurables, este proyecto
permitió la instanciación de un sistema operativo de tiempo real para aplicaciones que requieran esta característica.
			
	\section{En cuanto a la implementación Linux en arquitecturas sintetizables} 

Debido a la necesidad de ejecutar aplicaciones con mayores exigencias en la gestión de los recursos de hardware disponibles en el SoC y la
prestación servicios como gestion de interrupciones, excepciones y memoria, se evaluó en este trabajo la posibilidad de ejecutar el kernel de Linux y
sus herramientas de sistema provistas por BusyBox. 

Se compiló exitosamente el kernel de Linux con las herramientas de compilación cruzada disponibles para la arquitectura. Consecuentemente se ejecutó
con éxito el kernel sobre el simulador de la arquitectura OpenRISC. Sin embargo, el núcleo de Linux no pudo ejecutarse con éxito sobre la plataforma
de prueba debido a inconvenientes surgidos durante el acceso a memoria en la secuencia de inicio del Sistema Operativo. Sin embargo,puede señalarse
que mediante la depuración de los problemas surgidos se lograría obtener una imagen funcional del Kernel que permita realizar aplicaciones de prueba
sobre la plataforma real. Existen resultados satisfactorias que demuestan el correcto funcionamiento del SO en esta arquitectura sobre otras
placas de desarrollo.
				
\section{Trabajos futuros}

Este trabajo extiende las posibilidades de futuros trabajos en este tipo de plataformas debido a uno de sus principales objetivos:
el desarrollo de hardware y software de código abierto. Esto brinda la posibilidad de estudiar, modificar y mejorar el diseño mediante la
disponibilidad de su código fuente.    

		\begin {itemize}
		  \item Depuración del kernel de Linux para su correcto funcionamiento sobre la placa de desarrollo S3ADSP1800A.
		  \item Desarrollo de nuevos módulos de hardware para su utilización en SoC.
		  \item Desarrollo de drivers para nuevos módulos.
		  \item Desarrollo de aplicaciones reales sobre sistemas embebidos basados en SoC sintetizables.
		\end{itemize}
								
	
	 