
\chapter{Conclusiones}
 

Con el desarrollo realizado en el presente trabajo se alcanzó el objetivo de proveer un sistema integral FPGA-SoC-SO de código abierto
capaz de utilizarse en aplicaciones de pequeña envergadura.

\section{Hardware Open-Source}

El trabajo realizado en este proyecto sobre hardware sintetizable open-source representa una evidencia clara de que la tecnología de
código abierto en esta área fue suficiente para cumplir a los objetivos planteados en este proyecto. El proyecto ha mostrado que los componentes
más usuales de un sistema desarrollados en codigo abierto, tales como controladores USB, micropocesadores, controladores de red, etc; se integraron
y funcionaron correctamente en el sistema completo. Un ejemplo de esto lo constituyen los micropocesadores OpenRISC y muchos otros IP-Core
opensource.

En el presente trabajo se ha conseguido cerrar el ciclo completo de diseño con herramientas de desarrollo libres corriendo en una máquina GNU/Linux.
Tanto la compilación, simulación, síntesis y descarga en FPGA se realizaron enteramente en dicha plataforma. Para la compilación y
simulación se empleó exitosamente la herramienta \textit{Icarus Verilog} junto con \textit{GTKWave}; ambos programas libres. Para la síntesis se
utilizó el entorno ISE de Xilinx, el cual también fue ejecutado sobre una plataforma Linux. Las herramientas de código abierto y privativas
utilizadas durante este proyecto se complementaron correctamente, permitiendo cerrar el ciclo completo de desarrollo del sistema integral de código
abierto. Si bien se podría desarrollar una herramienta de síntesis opensource, las FPGAs continúan siendo el principal limitante para realizar esto, ya
que la información de sus detalles internos es una propiedad industrial de los principales fabricantes. 

\section{Procesadores Softcore}

Se analizaron diversas alternativas de implementación para los micropocesadores softcore. La mayoría de dichas alternativas son desarrollos llevados
a cabo por los principales fabricantes de FPGAs. Esto último permite la utilización de las herramientas provistas por dichos fabricantes pero
constituye un limitante desde el punto de vista de las herramientas opensource.

De la evaluación de los microprocesadores softcore realizada en este trabajo se concluye que en general poseen capacidades similares a sus
contrapartidas privativas. Incluso, muchos de ellos llegan a poseer un nivel de configuración y posibilidades de expansión mejor a estos últimos.
En particular, el análisis de rendimiento del microprocesador OpenRISC1200 que se llevó a cabo en el presente trabajo y cuyos resultados fueron
expuestos en el capítulo~\ref {chap:resultados}, permitió posicionar a este microprocesador en el mismo nivel de capacidades que sus contrapartidas
privativas.

En base a lo expuesto anteriormente, se puede concluir que la implementación de microprocesadores softcore representa una solución viable para
problemas industriales tales como: control, comunicaciones, DSP, PLCs. Además, los mismos pueden utilizarse con fines académicos para el estudio de
sistemas embebidos, microprocesadores, diseño e implementación de hardware reconfigurable, entre otros.

\section{Implementación de SoC sobre FPGA}

En el presente trabajo se analizaron dos implementaciones de sistemas en chip con núcleo OpenRISC a saber: MinSoC y ORPSoC.

El proyecto MinSoC mostró capacidades acotadas en aplicaciones \textit{bare-metal} de pequeña escala como puede ser un control PID simple o algún
dispositivo que no requiera de gran capacidad de cálculo. Los módulos UART y Ethernet le brindan al mismo la posibilidad de comunicarse con sistemas
de mayor envergadura mediante el protocolo Modbus o algún otro protocolo opensource. Debido a que no se cuenta con un sistema operativo que incluya
los drivers para la utilización de nuevos IP-Cores la capacidad de expansión de este se ve limitada por la necesidad de desarrollar programas
especiales para cada uno de los nuevos módulos que se utilicen.

Para aplicaciones con mayores exigencias de procesamiento y que requieran la posibilidad de ejecución de un sistema operativo se analizó el sistema
en chip ORPSoC de Opencores. Este sistema posee mayores prestaciones que el proyecto MinSoC y está soportado por gran cantidad de plataformas
reconfigurables. Con él, mismo fue posible llevar a cabo la instanciación de un sistema operativo de tiempo real.
 
\section{Implementación de los Sistemas Operativos ecOS y Linux en Arquitecturas Sintetizables}

El sistema operativo de ecOS se montó con éxito en el SoC ORPSoC permitiendo la realización de las pruebas funcionales del
sistema y algunos de sus principales IP-Cores, mostrando el correcto funcionamiento de algunas de las capacidades del SO, como lo es el manejo de
hilos. Sin bien se ejecutaron correctamente, algunos programas de prueba, el desarrollo de aplicaciones reales que corran bajo esta plataforma no forma
parte de los alcances de este proyecto.

Se evaluó la posibilidad de ejecutar el kernel de Linux y las herramientas de sistema provistas por BusyBox. Esto permite cubrir las necesidades en
aplicaciones con mayores exigencias en la gestión de los recursos de hardware disponibles en el SoC. Además, permite la prestación de servicios tales
como gestión de interrupciones, excepciones y memoria.
 
Como resultado se compiló exitosamente el kernel de Linux con las herramientas de compilación cruzada disponibles para la arquitectura.
Consecuentemente se ejecutó con éxito el kernel sobre el simulador de la arquitectura OpenRISC. Sin embargo, el núcleo de Linux no pudo ejecutarse
con éxito sobre la plataforma de prueba debido a inconvenientes surgidos durante el acceso a memoria en la secuencia de inicio del Sistema Operativo.
Se investigó y consultó sobre este problema en las comunidades y documentos oficiales del proyecto sin lograr una repuesta favorable que proporcione
una solución a este conflicto. Sin embargo, puede señalarse que mediante la depuración de los problemas surgidos se lograría obtener una imagen
funcional del Kernel que permita realizar aplicaciones de prueba sobre la plataforma real. Existen resultados satisfactorios que demuestran el correcto
funcionamiento del SO en esta arquitectura sobre otras placas de desarrollo, pero la verificación de estos resultados escapa a los alcances de este
trabajo.

\section{Trabajos Futuros}

%%Alternativamente, se podría realizar un motor de síntesis libre que genere un netlist en formato EDIF. Sin embargo, esto último no es suficiente
% porque no se encuentra disponible información sobre los detalles internos de las
%%FPGAs, al ser esta última considerada como secreto industrial. El primer paso para lograr la independencia sobre el uso de herramientas de síntesis
%%privadas sería la creación de una ``OpenFPGA'' con un descripción opensource.

El presente trabajo brinda las bases para futuros desarrollos en hardware y software de código abierto. Además del análisis llevado a cabo, la
disponibilidad del código fuente desarrollado facilita el estudio de los sistemas analizados y permite su adaptación y/o mejora para futuras
aplicaciones. Entre las lineas de trabajo que se desprenden de este proyecto se pueden mencionar las siguientes:

\begin {itemize}
\item Depuración del kernel de Linux para su correcto funcionamiento sobre la placa de desarrollo S3ADSP1800A.
\item Evaluación de la utilización de otras placas de desarrollo como soporte del sistema integral desarrollado en este proyecto.
\item Desarrollo de nuevos módulos de hardware para su utilización en SoC.
\item Desarrollo de drivers para nuevos IP-Cores.
\item Desarrollo de aplicaciones reales sobre sistemas embebidos basados en SoC sintetizables. 
\item Integración de unidad de procesamiento neuronal sobre microprocesador OpenRISC embebido para procesamiento de imágenes. (Proyecto del CUDAR
UTN-FRC).
\end{itemize}

Finalmente, otro campo de exploración puede ser un desarrollo OpenFPGA que posibilite el uso de herramientas opensource para síntesis.


