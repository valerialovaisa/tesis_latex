\chapter{Licencias}

\section{Software Libre y Código Abierto}

El \textit{software libre}, o en inglés \textit{free software}, ofrece
al usuario la libertad para ejecutarlo, copiarlo, modificarlo y
distribuirlo. Las características que tiene que cumplir el software
la Fundación para el Software Libre (FSF)~\cite{Etiqueta07}. La FSF
fue fundada por Richard Stallman en 1985 para promover la libertad del
usuario y defender los derechos de todo software
libre~\cite{Etiqueta14}. En particular, el proyecto GNU es patronizado
por la FSF. El software libre permite al usuario el ejercicio de
cuatro libertades básicas a saber:

\begin{itemize}
\item \textit{Libertad 0}: El software \textit{libre} puede estudiarse y adaptarse a las necesidades de quién lo use. Esta
  libertad requiere acceso al código fuente y tiene la ventaja de por
  ejemplo permitir descubrir funciones ocultas o saber como se realiza
  determinada tarea. Además, al poder adaptar el programa a las
  necesidades del usuario se pueden suprimir partes que no sean de
  interés, agregar otras partes que considere importantes, copiar
  una parte que realiza una tarea y/o asignarla a otro programa.
\item \textit{Libertad 1}: El software, sus copias y las
  modificaciones se pueden distribuir libremente. Esto significa que
  se posee la libertad de redistribuir el programa ya sea gratis o con
  algún costo.
\item \textit{Libertad 2}: Es posible mejorarlo y hacer pública esas
  mejoras. La libertad de hacer un programa mejor, implica que se
  puede hacer menores los requerimientos de hardware para funcionar,
  que tenga mayores prestaciones, que sus requerimientos no sean tan
  altos, que tenga menos errores. El poder liberar las mejoras al
  público quiere decir que si se realiza una mejora que permita un
  requerimiento menor de hardware, o que haga que ocupe menos espacio,
  se puede redistribuir ese programa mejorado o simplemente proponer
  la mejora en lugar público (un foro de noticias, una lista de
  correo, un sitio web, un FTP, un canal de chat).
\item \textit{Libertad 3}: El usuario al poseer el código fuente tiene
  poder de decisión, ya que podrá elegir quién puede modificar los
  programas que ha adquirido para mejorarlos (o bien mejorarlos el
  mismo). Es decir, esto permite que no exista un monopolio, porque en
  el caso de que un software sea discontinuado el usuario podrá,
  elegir a un desarrollador para continuar utilizando el software que
  fue discontinuado. Además el usuario no estará completamente a
  merced de tener que renovar su hardware y software constantemente
  según ocurre a menudo con las políticas de las empresas que producen
  software privativo y también será libre de vender o redistribuir
  software libre.
\end{itemize}

Cabe destacar que las libertades 1 y 3 tienen como prerrequisito que
se tenga acceso al código fuente. La libertad 2 hace referencia a la
libertad de modificar y redistribuir el software libremente licenciado
bajo algún tipo de licencia de software libre que beneficie a la
comunidad.


El término \textit{free} posee en inglés el doble significado
de ``gratuidad'' y ``libertad'' del software dependiendo del
contexto. Por tal motivo se comenzó a utilizar
en 1998 el término \textit{Código Abierto}, o en inglés ``Open
Source'' para intentar aclarar que la palabra libre, se refiere a la libertad y no al precio. Este software de \textit{Código Abierto} permite a las personas usar y modificar el software, compatibilizarlo con otros sistemas operativos o arquitecturas de hardware, compartirlo con otras personas y comercializarlo.
Desde el punto de vista de una ``traducción
estrictamente literal'', el significado textual de ``código abierto''
es que ``se puede examinar el código fuente'', por lo que puede
interpretarse como un término más débil y flexible que el del
\textit{software libre}. 

La definición oficial de \textit{código abierto} se encuentra en el
documento Open Source Definition (OSD) publicado por Open
Source Initiative (OSI). El OSI es un movimiento diferente al
movimiento del software libre que, aunque es incompatible con este
último desde el punto de vista filosófico, es completamente
equivalente desde el punto de vista práctico. El propósito de
establecer una definición oficial de software de \textit{código
  abierto} es asegurar que los programas distribuidos con ``Licencia
de código abierto'' estarán disponibles para su continua revisión y
mejora. Brindando además a los usuarios la posibilidad de adaptarlo a
sus necesidades y corregir errores rápidamente.

La principal diferencia entre los términos \textit{código abierto} y
\textit{libre} es que éste último tiene en cuenta los aspectos éticos
y filosóficos de la libertad mientras que el \textit{código abierto}
se basa únicamente en los aspectos técnicos. En un intento por unir
los mencionados términos, que se refieren a conceptos semejantes, se
intentó extender el uso del termino Free and Open Source Software
(FOSS) el cual se utiliza para referirse al software que se adhiere al
OSD y FSF.

% Es un término ambiguo, el acuerdo sobre la definición está
%en el documento llamado Open Source Definition (OSD), publicado por la
%Open Source Initiative (OSI). La OSD da grandes libertades a la hora
%de relicenciar software, lo que no sucede con otras licencias de
%código abierto. En particular la OSD permite ``mezclar'' software
%privativo con software \textit{Open Source}.
%El termino Free and Open Source Software (FOSS) se utiliza para
%referirse al software que se adhiere al OSD y FSF. El software \textit{libre} y
%de \textit{código abierto} es una sociedad inclusiva, el término abarca tanto
%el software \textit{libre} y software de {código abierto} a pesar de describir
%modelos de desarrollo similares, tienen diferentes culturas y
%filosofías.

\section{Licencias de Software Libre}

Mediante la \textit{licencia} un autor permite el uso de su creación a
otras personas. Es decir que la licencia es el instrumento que regula
la forma en que el usuario puede utilizar el software. Pueden existir
tantas licencias como acuerdos concretos se den entre el autor y el
licenciatario. Desde el punto de vista del software \textit{libre},
existen distintas variantes del concepto o grupos de licencias:

\begin {itemize}
\item \textit{General Public License (GNU GPL)}: Es la licencia más
  usada, garantiza a los usuarios finales (personas, organizaciones,
  compañías) la libertad de usar, estudiar, compartir (copiar) y
  modificar el software. Su propósito es declarar que el software
  cubierto por esta licencia es software libre y protegerlo de
  intentos de apropiación. Esta licencia fue creada originalmente por Richard
  Stallman para el proyecto GNU (GNU project). Su finalidad es proteger los derechos de
  los usuarios finales (usar, compartir, estudiar, modificar). Esta es
  la primera licencia copyleft para uso general. Copyleft significa
  que los trabajos derivados sólo pueden distribuirse bajo los
  términos de la misma licencia. Bajo esta filosofía, la licencia GPL
  garantiza a los destinatarios de un programa los derechos-libertades
  reunidos en la definición de software libre y usa copyleft para
  asegurar que el software está protegido cada vez que el trabajo es
  distribuido, modificado o ampliado.

  El software bajo licencia GPL puede aplicarse bajo todos los
  propósitos. Esto incluye los propósitos comerciales e incluso el uso
  como herramienta de creación de software propietario. En uso
  puramente privativo o interno, sin venta ni distribución, no es
  necesario hacer publico el código fuente para los usuarios. Sólo si
  un programa utiliza fragmentos de código GPL y el programa es
  distribuido, el código fuente en su totalidad debe estar disponible
  bajo la misma licencia. Lo mencionado es independiente de la
  plataforma utilizada. 

  Los usuarios o compañías que distribuyen sus trabajos bajo licencias
  GPL pueden hacerlo gratuitamente o cobrar por ellos. En particular,
  la GPL establece explícitamente que las obras GPL se puede vender a
  cualquier precio. Esto se deriva de la FSF, la cual argumenta que no
  se debe restringir la distribución comercial del software incluyendo
  la redistribución. La FSF permite al público crear nuevas licencias
  basadas en la GPL siempre y cuando las licencias derivadas no
  utilicen GPL sin permiso. Sin embargo, esto último no se recomienda
  ya que tal licencia puede ser incompatible con la GPL.

%Una de las más utilizadas es la Licencia Pública General de GNU (GNU GPL). El autor conserva los derechos de autor (copyright), y permite la redistribución y modificación bajo términos diseñados para asegurarse de que todas las versiones modificadas del software permanecen bajo los términos más restrictivos de la propia GNU GPL. Esto hace que sea imposible crear un producto con partes no licenciadas GPL: el conjunto tiene que ser GPL.
%Es decir, la licencia GNU GPL posibilita la modificación y redistribución del software, pero únicamente bajo esa misma licencia. Y añade que si se reutiliza en un mismo programa código "A" licenciado bajo licencia GNU GPL y código "B" licenciado bajo otro tipo de licencia libre, el código final "C", independientemente de la cantidad y calidad de cada uno de los códigos "A" y "B", debe estar bajo la licencia GNU GPL.
%En la práctica esto hace que las licencias de software libre se dividan en dos grandes grupos, aquellas que pueden ser mezcladas con código licenciado bajo GNU GPL (y que inevitablemente desaparecerán en el proceso, al ser el código resultante licenciado bajo GNU GPL) y las que no lo permiten al incluir mayores u otros requisitos que no contemplan ni admiten la GNU GPL y que por lo tanto no pueden ser enlazadas ni mezcladas con código gobernado por la licencia GNU GPL.
%En el sitio web oficial de GNU hay una lista de licencias que cumplen las condiciones impuestas por la GNU GPL y otras que no.5
%Aproximadamente el 60% del software licenciado como software libre emplea una licencia GPL o de manejo.
%
%Aquellos que no aceptan los términos de la GPL y sus condiciones no tienen permiso, en virtud del derecho de autor, a copiar o distribuir software con licencia GPL o trabajos derivados. Sin embargo, si no redistribuyen el programa, pueden utilizar el software en su organización a su gusto, y estas obras (incluidos los programas) construidas bajo este uso no requieren estár bajo esta licencia.
%
%La FSF permite al público crear nuevas licencias basadas en la GPL, siempre y cuando las licencias derivadas no utilicen GPL sin permiso. Esto no se recomienda, ya que tal licencia puede ser incompatible con la GPL. Otras licencias creadas por el proyecto GNU incluyen la GNU Lesser General Public License

\item \textit{Lesser General Public License (GNU LGPL)}: La licencia
  LGPL de software fue creada por la FSF para tener derechos menos
  restrictivos que GPL. Garantiza la libertad de compartir y modificar
  el software cubierto por ella y asegura que el software es libre
  para todos sus usuarios. Esta licencia permisiva se aplica a
  cualquier programa o trabajo que contenga una nota puesta por el
  propietario de los derechos del trabajo en la cual se establece que
  su trabajo puede distribuirse bajo los términos de está. El
  termino ``programa'', utilizado en lo subsecuente, se refiere a
  cualquier programa o trabajo original. Por otro lado, el ``trabajo
  basado en el programa'' se refiere al programa o cualquier trabajo
  derivado del mismo bajo la ley de derechos de autor. Esto es, un
  trabajo que contenga el programa o alguna porción de él, ya sea
  íntegra o con modificaciones o traducciones a otros idiomas. Otras
  actividades que no sean copia, distribución o modificación no están
  cubiertas en esta licencia y están fuera de su alcance. El acto de
  ejecutar el programa no está restringido y la salida de información
  del programa está cubierta sólo si su contenido constituye un
  trabajo basado en el programa, es independiente de si fue resultado
  de ejecutar el programa. Si esto es cierto o no depende de la
  función del programa. En resumen, si se tiene un programa que
  utiliza fragmentos de código LGPL no es necesario liberar el código
  de dicho programa.

\item \textit{Berkeley Software Distribution (BSD)}: La licencia BSD se denomina de esta forma porque se utiliza en gran cantidad del
  software distribuido junto a los sistemas operativos BSD. Es una
  licencia de software libre permisiva que tiene menos restricciones
  en comparación con otras como la GPL. La licencia BSD, al contrario
  que la GPL, permite el uso del código fuente en software no
  libre. El autor, bajo tal licencia, mantiene la protección de
  copyright únicamente para la renuncia de garantía y para requerir la
  adecuada atribución de la autoría en trabajos derivados. Sin
  embargo, permite la libre redistribución y modificación, incluso si
  dichos trabajos tienen propietario. Esta licencia asegura verdadero
  software libre, en el sentido que el usuario tiene libertad
  ilimitada con respecto al software.
\end {itemize}



%La GNU GPL se conoce como una \textit{licencia viral}, donde cualquier
%desarrollo que se haga bajo el uso de código licenciado bajo la GNU
%GPL debe ser entonces licenciado bajo la GNU GPL o cualquier
%\textit{licencia} autorizada por la FSF. En pocas palabras, una
%condición de uso del código licenciado bajo GPL es que su diseño se
%debe tener bajo la licencia GPL o una licencia compatible. Las
%licencias que se consideran compatibles con la GPL por la FSF son
%generalmente similares en las libertades que garantiza el software
%\textit{libre}.
%
%En la GNU se estipula que el código modificado debe estar disponible,
%y que cualquier desarrollo que utilice código bajo esta licencia tiene
%también que estar bajo la GNU GPL. Aunque, esto no es diferente a
%cualquier licencia comercial, donde el código fuente creado o
%modificado por los empleados de una empresa queda bajo la licencia
%exclusiva de la empresa. En el caso de la licencia de GNU, sin
%embargo, los usuarios están obligados a mantener sus diseños abiertos
%y libres, de la misma manera que los empleados están obligados a
%mantener su código propietario en secreto para cualquiera que no sea
%de la empresa.
%
%Otro punto de controversia es el uso combinado del código fuente donde
%cada uno esta bajo una licencia diferente, dando lugar al concepto de
%\textit{compatibilidad de la licencia}.
%
%Estas diferencias de opinión con respecto a lo que constituye un
%trabajo derivado, y la ambigüedad en torno a otros aspectos de la
%concesión de licencias de código abierto podría tener consecuencias
%para campos como el diseño de hardware, lo que se discutirá en una
%sección posterior.

\section{Desarrollos de Código Abierto}

Tras la adopción creciente de software de código abierto en la década
del 90, la organización OSI propuso que el software \textit{libre},
como fue conocido en ese entonces, tenía un lugar en la industria
comercial. El éxito del modelo sorprendió a muchas personas y demostró
que era un modelo de desarrollo viable. El aumento en la popularidad y
utilidad de Internet proporciono el inicio para las grandes
comunidades de código abierto.

Con el éxito del software \textit{libre} como GNU/Linux, Servidor HTTP
Apache, Mozilla Firefox y OpenOffice.org, muchas empresas
comenzaron a interactuar con la comunidad del software
\textit{libre}. Esto trajo aparejado una nueva dificultad a la hora de
elegir las licencias de software libre y en la selección de qué
software se liberará. Un ejemplo de una relativamente exitosa entrada
a la comunidad del software libre fue que ``Sun Microsystems'' liberó
StarOffice como OpenOffice.org bajo la GNU LGPL. Esto fue muy bien
recibido por la comunidad de software \textit{libre} que no tenía una
suite ofimática madura en ese momento. Un ejemplo de una entrada más
difícil para la comunidad del software libre es el de Real Networks,
qué escribió su propia licencia, y puso en libertad sólo una parte de
su suite de software. En particular, el códec de los programas
informáticos necesarios para ver archivos Real Video, no fue puesto en
libertad.

El sitio web \textit{OpenCores} con sus comienzos en el año 2000,
proporciona un sitio para la comunidad de hardware de código abierto.
Fue una de las primeras comunidades de desarrollo de hardware y
actualmente la más grande, con más de cien mil usuarios y cerca de mil
proyectos. El principal objetivo de OpenCores es diseñar y publicar
diseños de núcleos bajo una licencia de hardware de código abierto
siguiendo el modelo de Licencia LGPL usada para el software.

\subsection{Código abierto en la industria}

Actualmente exciten un gran número de aplicaciones de código
abierto. Por ejemplo se dispone de software multimedia, productividad
de oficina, herramientas de gráficos, sistemas operativos y de
comunicaciones. Los gobiernos y las grandes empresas están
aprovechando cada vez más el software de código abierto que existe y
se están convirtiendo en importantes contribuyentes a los proyectos
del software que adoptan. Las tendencias recientes han hecho mucho
para disipar la imagen de solitarios desarrolladores de código abierto
como los únicos contribuyentes de trabajos. Al estar las entidades
comerciales aumentando la adopción y utilización de los proyectos de
software libre, se están convirtiendo en los contribuyentes más
frecuentes. Actualmente el kernel de Linux tiene la mayor parte de sus
contribuciones de código proveniente de entidades comerciales. Esto se
debe en gran parte a que están trabajando con el núcleo Linux en sus
productos o porque desean asegurar el soporte de su hardware en el
kernel. Entre dichas empresas se pueden mencionar a Intel, IBM y
AMD. Lo mismo pasa para proyectos como Apache y MySQL.

La gran cantidad de software de código abierto disponible ofrece la
adopción de licencias FOSS con la posibilidad de elegir entre: la
publicación del trabajo derivado, el desarrollo de una solución
interna o la compra de una solución propietaria manteniendo el código
en privado.

Las entidades comerciales interesadas en mantener sus plataformas en
buenas condiciones son en gran parte las que ofrecen mayor soporte y
desarrollo a herramientas de programación como el compilador GCC y
binutils del proyecto GNU. El desarrollo de las herramientas de
software de código abierto es impulsado por GNU brindando soporte a
diferentes plataformas con el fin de fomentar el uso de un
compilador-optimizador de clase global, que atraiga a desarrolladores
entregando herramientas que funcionen en diferentes arquitecturas.

El aumento del uso comercial del software de código abierto trae como
consecuencia que las empresas aporten recursos a estos proyectos, disminuyendo por consiguiente el desarrollo y mantenimiento de sus
propios conjuntos de compiladores y herramientas. Por ejemplo en el
caso del GCC. Uno de los objetivos del código abierto es proporcionar
una base de software donde los desarrolladores se encuentren cómodos
para adoptarlo. Esto se debe al aumento del software existente
producto de la publicación de trabajos derivados, lo que evita la
tarea de empezar a desarrollar desde cero o comprar una solución
propietaria. Estas son solo algunas de las motivaciones para la
adopción y contribución de código abierto, pero podrían ser muchas más
y variadas dependiendo de la funcionalidad del proyecto.

Para las entidades comerciales la motivación de desarrollar código
abierto proviene de cobrar por servicios extras al
proyecto. Normalmente se requiere de conocimientos técnicos y
experiencia para implementar y/o personalizar un proyecto en
particular. Las empresas que adoptan y mantienen desarrollos de código
abierto no tiene el costo de regalías o licencias, solo el de
conocimiento necesarios para hacerlo.

A pesar de que el software de código abierto no es tradicionalmente un
generador continuo de productos innovadores, no se opone a la
estrategia de desarrollo elegida para una tecnología
innovadora. Cualquier diseño innovador de código abierto suele tener
retornos valiosos para la inversión requerida. Es una opción
interesante para los desarrolladores empezar con una base que les
ahorre tiempo y les permita competir frente a implementaciones
propietarias, además de proporcionar soporte técnico a
largo plazo para una aplicación.

Cuando el ciclo de vida de un producto llega a su fin, liberar el
código fuente permite a otros desarrolladores aportar conocimiento
brindando soluciones a problemas derivados de las actualizaciones
inevitable de las plataformas.

Esta breve mirada a la situación actual de código abierto ha indicado
que el mismo se ha convertido en una fuerza a tener en cuenta en el
mundo del software. Sin embargo, este no es el caso en la actualidad
para el desarrollo de hardware de código abierto.

\subsection{Desventajas de la adopción del código abierto en software}

Son muchos los que no están de acuerdo con el desarrollo de código
abierto en la actualidad a pesar del gran aumento en su popularidad y
uso. El software o hardware privativo puede ser desplazado del mercado
por una alternativa no privativa, sin costo. Los que obviamente van a
estar en desacuerdo son las entidades que se encuentren amenazadas por
una alternativa lo suficientemente innovadora que provoque una
disminución en sus ganancias. Una alternativa para los desarrolladores
menos favorecidos es mejorar el código abierto existente el cual
generalmente no es suficientemente innovador. Esto les permite
competir con cualquier producto del mercado al mismo tiempo que
adquieren conocimientos, ahorran tiempo y desarrollan un
negocio. Además, tal mejora contribuye a mejorar el proyecto original.

El inconveniente de mostrar el código del proyecto mejorado es que los
desarrolladores exponen sus técnicas innovadoras. Sin embargo, esto se
compensado por la reducción de la inversión en el desarrollo de su
producto. Por otro lado, la mejora introducida en el proyecto de
código abierto atrae a otros contribuyentes a trabajar en él.

Las implementaciones de código abierto varían en calidad y
funcionalidad dependiendo de la colaboración de los
desarrolladores. No hay dudas de que hay software de código abierto de
gran utilidad. Sin embargo, no debe ser visto sólo como un producto
innovador sino también como un paso en el camino hacia el avance
tecnológico.

\section{Código Libre para Hardware}

El OpenRisc y otros IP publicados en OpenCore no sólo muestran el
estado del proyecto sino que también permiten a los desarrolladores
poder continuar el desarrollo de los núcleos. La aceptación de los
proyectos de código abierto en RTL por parte de las entidades
comerciales de IP no es la misma que para el desarrollo de software de
código abierto.

El problema en la implementación de los proyectos de hardware de
código abierto es que la falta de herramientas de ``Entorno de
Desarrollo Integrado'' (IDE) es inaceptable para desarrollos de ASIC
dado el elevado costo de corregir errores. Para brindar una solución
a este problema se tendría que incluir dicha herramienta junto con el
core. Las herramientas que se encuentran disponibles en la industria
tienen un elevado precio y pueden variar de acuerdo al proveedor. Esto
representa un gran obstáculo para la entrada al mercado de los
desarrolladores de IP.

La falta de alternativas libres en EDA de código abierto limita al
desarrollo de núcleos de código abierto. Esto se debe en parte a que
la tecnología utilizada para el desarrollo de hardware se considera
secreto industrial y esto limita el desarrollo de herramientas
libres. Por tal motivo, actualmente, para minimizar el riesgo las
implementaciones basadas en hardware de código abierto se realizan en
dispositivos como FPGA donde generalmente se pueden hacer cambios del
código RTL a un menor costo.


\subsection{Problema para implementar y desarrollar hardware de código abierto}

La necesidad de una plataforma donde implementar el diseño de hardware
y la falta de herramientas de desarrollo libres son un gran obstáculo
que no enfrenta el software de código abierto. Estas plataformas se
comercializan por empresas de hardware ofreciendo herramientas de
compilación, simulación, síntesis y descarga para sus FPGA. La
complejidad de las herramientas de diseño y la pronunciada curva de
aprendizaje perjudica a los principiantes.

Otra dificultad a la hora de realizar un diseño para FPGA es que el
mismo se debe describir a un nivel de abstracción muy bajo. Para
lograr un resultado ``útil'' se requiere por lo general un gran
trabajo que atraviese varios niveles de abstracción para poder
utilizarlo cómodamente en una computadora.

Como ejemplo de una aplicación en hardware podría ser el desarrollo de
un core que realice transacciones de I/O de un sensor. Dicho core
podría proporcionar información a una aplicación que controle la
temperatura en un espacio determinado. Esto requeriría el desarrollo y
prueba del modelo de hardware y la implementación en FPGA. Por
ejemplo, sea un microprocesador el que se encuentra corriendo sobre la
FPGA, el cual brinda los servicios de red a través de un sistema
operativo de tiempo real (RTOS). Este módulo personalizado debería
requerir el desarrollo de una capa de software, lo que significa la
necesidad de un driver para que permita al sistema operativo en tiempo
real interactuar con el periférico, haciendo una abstracción del
hardware y proporcionando una interfaz para usarlo. El sistema
operativo en tiempo real se conecta con la aplicación que se ejecuta
en el microprocesador de la FPGA para proporcionan lo datos a través
del enlace de red. De esta forma, los datos de este sensor se
encontrarán disponibles para la aplicación de nivel superior. Este es
sólo un ejemplo en el cual el diseñador posiblemente haya seleccionado
una solución que utilice un bus estándar. Sin embargo, existen algunos
casos donde el diseñador desarrolla nuevos cores de interfaces o
controladores en FPGA para proveer acceso a sistemas heredados
(legacy) o nuevos buses estándar, donde vale destacar el trabajo extra
requerido al escribir código RTL que provea la interfaz física. Viendo
la cantidad de desarrollo y pruebas requeridas para poner en práctica
estas soluciones, es fácil sentirse abrumado por la cantidad de
trabajo necesario para completar una tarea tan aparentemente trivial.

Al comparar esto último con la implementación de software de código
abierto. En el segundo, simplemente se descarga el código fuente, se
lo compila, se lo ejecuta y se lo prueba todo usando la misma
plataforma. Hay una gran ventaja en la facilidad con que se accede la
plataforma de desarrollo (el host) y las herramientas de
desarrollo son mucho más simples (gcc, make en el sistema
host). Además, las pruebas son más cortas y más fáciles (se
ejecutan en el equipo host a través de un shell).


A medida que se incremente la cantidad de proyectos de hardware de código
abierto y disminuya la complejidad de las herramientas de desarrollo
es de esperar que se superen los obstáculos mencionados. Las barreras
iniciales a las que se enfrentó el software de código abierto parecían
igual de complicadas de superar. Es de esperar que con el tiempo y el
aumento de participantes, el hardware de código abierto alcance el
mismo éxito. En ese momento, los diseños serán tan grandes que no
cabrán en los dispositivos programables actuales. La reconfiguración
será un elemento imprescindible. Las fronteras entre el hardware y el
software se harán cada vez más difusas.

% También la utilidad de cualquier diseño de hardware que se podría implementar en una FPGA está limitada por el hecho de que se realiza a un nivel muy bajo de abstracción, y para lograr un resultado positivo "útil" para los experimentador o aficionado, por lo general requiere un gran trabajo a través de muchos niveles de abstracción para lograr algo que sea fácilmente utilizable a partir de una interfaz en una PC.
%%%%%
%Un ejemplo podría ser el desarrollo de un core para que realice transacciones de I/O de un sensor, para proporcionar información a una aplicación por ejemplo que controle la temperatura en un espacio determinado. Esto requeriría el desarrollo y prueba del modelo de hardware y la implementación en FPGA . Asumiendo que es un microprocesador el que se encuentra corriendo sobre la FPGA, dando servicios de red a través de un sistema operativo de tiempo real (RTOS), este módulo personalizado debería requerir el desarrollo de una capa de software, lo que significa la necesidad de un driver para que permita al sistema operativo en tiempo real interactuar con el periférico, haciendo una abstracción del hardware y proporcionando una interfaz para usarlo. El sistema operativo en tiempo real se conecta con la aplicación que se ejecuta en el microprocesador de la FPGA para proporcionan lo datos a través del enlace de red, así los datos de este sensor se encontraran disponibles para la aplicación de nivel superior. This is just one example where, quite probably the designer might have chosen a solution that uses a standard bus, however there’s often cases for custom controller or interface cores in FPGAs to provide
%access to legacy, or very-new or esoteric bus standards, and highlights the extra work required beyond writing RTL to provide the physical interface.
%Viendo la cantidad de desarrollo y pruebas requeridas para poner en práctica estas soluciones, es fácil sentirse abrumado por la cantidad de trabajo necesario para completar una tarea tan aparentemente trivial.

%Comparando esto con adoptar un programa de código abierto, que consiste en descargar un código fuente, compilarlo y ejecútalo en su computadora, donde la aplicación puede ser ejecutada en el host para comprobar la funcionalidad y allí finalice la mayor parte del ciclo de desarrollo, . Las diferencias son el acceso inherentes a la plataforma de desarrollo (el host), las herramientas de desarrollo mucho mas simples (gcc, make en el sistema host) el ciclo de desarrollo, pruebas más cortas y más fácil (que se ejecuta en el equipo host a través de un shell.)

%A medida que mas proyectos opnesource son desarrollados y los sistemas de desarrollo sean mas ágiles, se puede esperar que estas barreras para los desarrolladores de diseño de hardware opensource puedan ser superadas. En los principios del del desarrollo de software opensource parecían igual de complicados. Se espera que con el tiempo y el aumento de participantes el hardware de código abierto alcance el mismo exito.En ese momento, los diseños serán tan grandes que no cabrán en los dispositivos programables actuales. La reconfigurabilidad será un elemento imprescindible. Las fronteras entre el hardware y el software se harán cada vez más difusas. El deseo, casi utópico, es lograr correr un kernel Linux hardware basándose en las posibilidades que nos ofrece la reconfigurabilidad.

\section{Licencias de Hardware}

Una cuestión que queda por resolver es la concesión de licencias
para el diseño de hardware de código abierto. Por ejemplo, el proyecto
OpenRISC utiliza licencias públicas del proyecto GNU. Sin embargo,
estas se refieren específicamente a software y no se sabe bien que se
aplica al hardware. El sitio web del proyecto GNU contiene una sección
con preguntas frecuentes (FAQ) que afirma lo siguiente.

\textit{Cualquier material que puede ser licenciado con derechos de
autor puede ser licenciado bajo la GPL}.

%GPLv3 también se puede utilizar para materiales de licencia cubiertos por otras leyes copyrightlike, como máscaras de semiconductores. Así, por ejemplo, puede liberar un dibujo de un diseño de hardware bajo la GPL. Sin embargo, si alguien utilizó esa información para crear hardware físico, que lo harían no tienen obligaciones de la licencia al distribuir o vender el dispositivo: se queda fuera del ámbito del derecho de autor y por lo tanto la propia GPL.}

%Esto no es claro para los diseños específicos para FPGA o código, incluso RTL, ya que puede terminar como un conjunto de máscaras, o puede terminar como un flujo de bits binario para configuración de una FPGA.

%Una indicación de la naciente idea del desarrollo de hardware de
%código abierto proviene de la publicación reciente (febrero de 2011)
%de un conjunto de principios para los participantes de la comunidad de
%hardware de código abierto. El siguiente es el código abierto Hardware
%(OSHW) Declaración de Principios 1.0 de FreedomDefined.org. Estas
%publicaciones proporcionan un punto de referencia para saber si el
%diseño puede estar bajo licencia de ``hardware de código abierto''.

%\textit{El Hardware de código abierto cuyo diseño está a disposición del público
%por lo que cualquier persona puede estudiar, modificar, distribuir, poner, y vender el
%diseño o hardware basado en ese diseño. La fuente de hardware, el diseño
%del que está hecho, está disponible en el formato preferido para realizar
%modificaciones a el mismo. Idealmente, el hardware de código abierto utiliza fácilmente los componentes y materiales disponible, procesos estándares, una infraestructura abierta, sin restricciones
%contenida, herramientas de diseño y de código abierto para maximizar la capacidad
%de las personas para hacer y usar el hardware} \cite{Etiqueta11}

%El FreedomDefined.org específica fuente y documentación, trabajos derivados y las limitaciones de la licencias.Se espera de acuerdo con estos principios, todo el material este disponible como código abierto y la documentación para el diseño. Cualquier trabajos derivados o modificado deba estar disponible.

Cualquier licencia de hardware de código abierto se puede utilizar
para restringir (o en este caso, para no restringir) los planos de un
diseño, pero no el uso del dispositivo fabricado. Estos son conceptos
que ya se encuentran a menudo en las licencias de software de código 
abierto, pero de nuevo, no es tan claro como se aplica para el diseño
del hardware de código abierto.

Una de las primeras licencias de hardware de código abierto es la
Amateur Packet Radio Licencia Open Hardware Tucson (TAPR OHL). Los
autores de TAPR OHL identifican el problema con las licencias de
software existentes. En las mismas, los derechos de autor protegen la
documentación de copias, modificaciones y distribuciones, pero esto
tiene poco que ver con el derecho de hacer, distribuir o usar un
producto basado en la documentación~\cite{Etiqueta12}. En
consecuencia, la TAPR OHL esta siendo adoptada por varios aficionados
y empresas comerciales.


%Su licencia identifica patentes como un problema, pero afirma que quienes se beneficien de la OHL no podran presentar una demanda alegando que el diseño infringen sus patentes u otra propiedad intelectual.
%How open source hardware licenses and patent law will be compatible with regards to handling infringement is yet to be seen

En general, las licencias actuales de hardware de código abierto han
recibido críticas del OSI, entre otras cosas, por la adopción de un
significado diferente de la palabra
``distribución''~\cite{Etiqueta13}. Sin embargo, es de esperar que en
el futuro próximo surjan licencias alternativas de hardware de código
abierto que resuelvan las ambigüedades presentes en las licencias
actuales.

%Para el proyecto OpenRISC hay un equilibrio para afrontar la adopción entre una licencia que es demasiado liberal, y por lo tanto menos probable que resulte en la contribución a la comunidad de desarrollo, y una licencia que fomenta mas el desarrollo de código abierto, pero se considera entonces demasiado restrictiva con respecto a la utilización de codigo abierto IP con una IP patentada.

%Por un lado, hay un deseo de aumentar la participación en el desarrollo de hardware de código abierto en general, y específicamente en el proyecto OpenRISC MinSoc y OrpSoc,para aumentar el conjunto de trabajos disponibles, esto se puede lograr utilizando un licencia viral como lo es la GNU GPL (considerando la síntesis del codigo RTL como un proceso de compilación estatica).



%\section{OpenRisc}

%Como se dijo anteriormente en el debate sobre la tecnología de código abierto por lo general los modelos de desarrollo de código abierto no son imnovadores.
%La mayor parte de los proyectos de software libre tienen como objetivo utilizar recursos ya existentes y bien conocidos de manera que permitan la apertura y eliminación de restricciones que se encuentran en otra implementaciones propietarias. Esta es la duda en el caso de la OpenRISC proyecto. Es en gran medida tomando ideas que ya son bien conocidos y comoditizados y la creación de una versión con más libertad para el usuario final. Había muy poco, si nada, innovador en la especificación arquitectónica OR1K. Esto no quiere decir los resultados no tienen ningún valor. Tampoco necesariamente excluye cualquier OpenRISC futuro5o implementaciones de arquitecturas con el objetivo de innovar.


%Conclusión!!Trabajar con un sistema final bajo licencias de hardware siguiendo el modelo de la Licencia LGPL para el software. Estamos comprometidos con el ideal de libre disposición, de libre uso y hardware de código abierto reutilizable.