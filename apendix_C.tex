
\chapter{CÓDIGOS DE PRUEBAS}

 \section{Introducció}

En el presente anexo se  publicaran los corresppndientes códigos que se realizaron con el fin de probar el correcto funcionamiento de los requerimientos e los diferentes prototipos


 \section{Programa de Prueba para el UART}

\begin{verbatim}
#include <board.h>
#include <support.h>
#include <or1200.h>
#include <int.h>

#include <uart.h>

int main()
{
	uart_init();

	int_init();
	int_add(UART_IRQ, &uart_interrupt, NULL);
	
	/* We can't use printf because in this simple example
	   we don't link C library. */
	uart_print_str("Hello World.\n");
	
	report(0xdeaddead);
	or32_exit(0);
}

\end{verbatim}

 \section{Programa de Prueba para Ethernet}

\begin{verbatim}
#include <board.h>
#include <support.h>
#include <or1200.h>
#include <int.h>

#include <uart.h>
#include <eth.h>


extern int eth_rx_len;
extern int eth_rx_done, eth_tx_done;
extern unsigned char * eth_rx_data;
extern unsigned char * eth_tx_data;

void eth_receive()
{
	int i;
	uart_print_str("Length: \n");
	uart_print_long(eth_rx_len);
	uart_print_str("\n");
	uart_print_str("Data: \n");
	for ( i = 0; i < eth_rx_len; i++ )
	{
		uart_print_short(eth_rx_data[i]);
		uart_print_str("\n");
	}
	eth_recv_ack();
}

int main()
{
	uart_init();

	int_init();
	eth_init();
	int_add(UART_IRQ, &uart_interrupt, NULL);
	int_add(ETH_IRQ, &eth_interrupt, NULL);

	/* We can't use printf because in this simple example
	   we don't link C library. */
	uart_print_str("Hello World.\n");

	eth_tx_data[0] = 0xFF;
	eth_tx_data[1] = 0x2B;
	eth_tx_data[2] = 0x40;
	eth_tx_data[3] = 0x50;

	eth_send(4);

	while(1)
	{
		if (eth_rx_done)
		{
			eth_receive();
		}
	}

	report(0xdeaddead);
	or32_exit(0);
}
\end{verbatim}


 \section{Programa de Prueba para Capacidad de Procesamiento}

\begin{verbatim}
#include <board.h>
#include <support.h>
#include <or1200.h>
#include <int.h>
#include <uart.h>

#define con 15
#define N 16

void genmatriz(unsigned seed1,unsigned *matriz,int indice){
/*
    uart_print_str("seed ="); 
    uart_print_long(seed1);
    uart_print_str("\n");
*/    
    unsigned sr1[N] ;
    unsigned i , j ,q  ;
    unsigned tmp;    
    
    for(i=0; i<N; i++){
  	sr1[i] =  (seed1&1);
  	seed1/=2;
    }

    for (q=0;q<(indice*indice);q++){
		tmp = sr1[15]^sr1[14]^sr1[13]^sr1[11];
		
		for(j=N;j>0;j--) sr1[j]=sr1[j-1];
		sr1[0] = tmp;

		matriz[q] = tmp;
		
                //uart_print_long(tmp);
		//if (((q+1)%con_i)==0) uart_print_str("\n");
    }   

}


int main()
{
 
    // Declaraciones
    int i , j , k ; 
    int indice;
    unsigned matriz  [con*con];
    unsigned matrizA [con*con];
    unsigned matrizB [con*con];
    
    int producto ; 
    int acumulador ;
    unsigned tiempo;
    int useg;

	

////// Setup Inicial
    //tick_init();
    asm("l.mtspr\t\t%0,%1,0": : "r" (SPR_TTMR), "r" (SPR_TTMR_CR));
    asm("l.mtspr\t\t%0,%1,0": : "r" (SPR_TTCR), "r" (0x0));
    //mtspr(SPR_TTCR, SPR_TTMR_CR);
    //mtspr(SPR_TTCR, 0x0);
        
    //tick_start();
    asm("l.mtspr\t\t%0,%1,0": : "r" (SPR_TTCR), "r" (0x0));
    //mtspr(SPR_TTCR, 0x0);
    
    tiempo = mfspr(SPR_TTCR);
    useg = (int) tiempo * 50 ;
    uart_print_str("Tick de ejecucion setup tick :");
    uart_print_long((long)tiempo);
    uart_print_str(" En us = ");
    uart_print_long(useg);
    uart_print_str("\n");   
   
    for (indice=1; indice<=con; indice++){
    //tick_start();
    //asm("l.mtspr\t\t%0,%1,0": : "r" (SPR_TTCR), "r" (0x0));
    //uart_print_str("Matrices de  :");
    uart_print_long(indice);
    uart_print_str(",");
    //uart_print_str("\n");

 
    mtspr(SPR_TTCR, 0x0);
    
    i= 0;
    j= 0;
    k= 0;
    producto = 0;
    acumulador = 0;
/*
    for (i=0;i<indice;i++){
		for (j=0;j<indice;j++){
			matriz  [i][j] = 0;
			matrizA [i][j] = 1;
			matrizB [i][j] = 1;
		}
	}
*/
      tiempo = mfspr(SPR_TTCR);
      genmatriz (tiempo,matriz,indice);
      tiempo = mfspr(SPR_TTCR);
      genmatriz (tiempo,matrizA,indice);
      tiempo = mfspr(SPR_TTCR);
      genmatriz (tiempo,matrizB,indice);
    

//    tiempo = mfspr(SPR_TTCR);
//    useg = (int) tiempo * 50 ; 
//    uart_print_str("Tick de ejecucion seteo de variables :");
//    uart_print_long((long)tiempo);
//    uart_print_str(" En us = ");
//    uart_print_long(useg);
// 	uart_print_str("\n");
 	 
 
	mtspr(SPR_TTCR, 0x0);
    
	for (i=0;i<indice;i++){
		for (j=0;j<indice;j++){
			for (k=0;k<indice;k++){
				producto = matrizA[i*con+k] * matrizB[k*con+j];
				acumulador = acumulador + producto;
			}
			matriz [i*con+j] = acumulador;
			acumulador = 0;
		}	
	}
    
    tiempo = mfspr(SPR_TTCR);
//    useg = (int) tiempo * 50 ;
//    uart_print_str("Tick de ejecucion de la multiplicacion :");
    uart_print_long((long)tiempo);
//    uart_print_str(" En us = ");
//    uart_print_long(useg);
    uart_print_str("\n");   
    

/*
    uart_print_str("Matriz Resultado");   
    uart_print_str("\n");   
    
    for (i=0;i<indice;i++){
		for (j=0;j<indice;j++){
			uart_print_long(matriz [i*con+j]);
			uart_print_str(" ");
		}
		uart_print_str("\n");
	}
  
 */  
    } 
  
    uart_print_str("Fin del programa\n");

    return 0;
}
\end{verbatim}

 \subsection{Generador de secuencia binaria seuda aleatoria}
\begin{verbatim}
#include "stdio.h"
#include "time.h"
#define con_i 10

#define N 16

void genmatriz(unsigned seed1,unsigned *matriz){
    printf ("seed1 ="); 
    printf("%x",seed1);
    printf("\n");   
    
    //unsigned  seed1 ; 
    unsigned sr1[N] ;
    unsigned i , j ,q  ;
    unsigned tmp;    
    
    for(i=0; i<N; i++){
  	sr1[i] =  (seed1&1);
  	seed1/=2;
    }

    for (q=0;q<(con_i*con_i);q++){
		tmp = sr1[15]^sr1[14]^sr1[13]^sr1[11];
		
		for(j=N;j>0;j--) sr1[j]=sr1[j-1];
		sr1[0] = tmp;

		matriz[q] = tmp;
		
                //printf ("%i ",tmp);
		//if (((q+1)%con_i)==0) printf ("\n");
    }   
	printf ("\n\n");
}

int main(void ) {

unsigned matriz[100] ;
unsigned i,j ;
genmatriz (time(NULL),matriz);
for (i=0;i<100;i++){
	printf ("%i ",matriz[i]);
	if (((i+1)%10)==0) printf ("\n");
}

 return 0;
}
\end{verbatim}

