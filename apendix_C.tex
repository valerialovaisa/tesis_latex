
\chapter{Códigos de Prueba} %%%%%%agrego archivo para configurar kernel de linux

 \section{Introducción}

En el presente anexo se  publicaran los corresppndientes códigos que se realizaron con el fin de probar el correcto funcionamiento de los requerimientos e los diferentes prototipos


 \section{Programa de Prueba para el UART}

\begin{lstlisting}[language=C,frame=single]
#include <board.h>
#include <support.h>
#include <or1200.h>
#include <int.h>

#include <uart.h>

int main()
{
	uart_init();
	int_init();
	int_add(UART_IRQ, &uart_interrupt, NULL);
	
	/* We can't use printf because in this simple example
	   we don't link C library. */
	uart_print_str("Hello World.\n");
	
	report(0xdeaddead);
	or32_exit(0);
}

\end{lstlisting}

 \section{Programa de Prueba para Ethernet}

\begin{lstlisting}[language=C,frame=single]
#include <board.h>
#include <support.h>
#include <or1200.h>
#include <int.h>

#include <uart.h>
#include <eth.h>


extern int eth_rx_len;
extern int eth_rx_done, eth_tx_done;
extern unsigned char * eth_rx_data;
extern unsigned char * eth_tx_data;

void eth_receive()
{
	int i;
	uart_print_str("Length: \n");
	uart_print_long(eth_rx_len);
	uart_print_str("\n");
	uart_print_str("Data: \n");
	for ( i = 0; i < eth_rx_len; i++ )
	{
		uart_print_short(eth_rx_data[i]);
		uart_print_str("\n");
	}
	eth_recv_ack();
}

int main()
{
	uart_init();

	int_init();
	eth_init();
	int_add(UART_IRQ, &uart_interrupt, NULL);
	int_add(ETH_IRQ, &eth_interrupt, NULL);

	/* We can't use printf because in this simple example
	   we don't link C library. */
	uart_print_str("Hello World.\n");

	eth_tx_data[0] = 0xFF;
	eth_tx_data[1] = 0x2B;
	eth_tx_data[2] = 0x40;
	eth_tx_data[3] = 0x50;

	eth_send(4);

	while(1)
	{
		if (eth_rx_done)
		{
			eth_receive();
		}
	}

	report(0xdeaddead);
	or32_exit(0);
}
\end{lstlisting}


 \section{Programa de Prueba para Capacidad de Procesamiento}

\begin{lstlisting}[language=C,frame=single]
#include <board.h>
#include <support.h>
#include <or1200.h>
#include <int.h>
#include <uart.h>

#define con 15
#define N 16

void genmatriz(unsigned seed1,unsigned *matriz,int indice){
/*
    uart_print_str("seed ="); 
    uart_print_long(seed1);
    uart_print_str("\n");
*/    
    unsigned sr1[N] ;
    unsigned i , j ,q  ;
    unsigned tmp;    
    
    for(i=0; i<N; i++){
  	sr1[i] =  (seed1&1);
  	seed1/=2;
    }

    for (q=0;q<(indice*indice);q++){
		tmp = sr1[15]^sr1[14]^sr1[13]^sr1[11];
		
		for(j=N;j>0;j--) sr1[j]=sr1[j-1];
		sr1[0] = tmp;

		matriz[q] = tmp;
		
                //uart_print_long(tmp);
		//if (((q+1)%con_i)==0) uart_print_str("\n");
    }   

}


int main()
{
 
    // Declaraciones
    int i , j , k ; 
    int indice;
    unsigned matriz  [con*con];
    unsigned matrizA [con*con];
    unsigned matrizB [con*con];
    
    int producto ; 
    int acumulador ;
    unsigned tiempo;
    int useg;

	

////// Setup Inicial
    //tick_init();
    asm("l.mtspr\t\t%0,%1,0": : "r" (SPR_TTMR), "r" (SPR_TTMR_CR));
    asm("l.mtspr\t\t%0,%1,0": : "r" (SPR_TTCR), "r" (0x0));
    //mtspr(SPR_TTCR, SPR_TTMR_CR);
    //mtspr(SPR_TTCR, 0x0);
        
    //tick_start();
    asm("l.mtspr\t\t%0,%1,0": : "r" (SPR_TTCR), "r" (0x0));
    //mtspr(SPR_TTCR, 0x0);
    
    tiempo = mfspr(SPR_TTCR);
    useg = (int) tiempo * 50 ;
    uart_print_str("Tick de ejecucion setup tick :");
    uart_print_long((long)tiempo);
    uart_print_str(" En us = ");
    uart_print_long(useg);
    uart_print_str("\n");   
   
    for (indice=1; indice<=con; indice++){
    //tick_start();
    //asm("l.mtspr\t\t%0,%1,0": : "r" (SPR_TTCR), "r" (0x0));
    //uart_print_str("Matrices de  :");
    uart_print_long(indice);
    uart_print_str(",");
    //uart_print_str("\n");

 
    mtspr(SPR_TTCR, 0x0);
    
    i= 0;
    j= 0;
    k= 0;
    producto = 0;
    acumulador = 0;
/*
    for (i=0;i<indice;i++){
		for (j=0;j<indice;j++){
			matriz  [i][j] = 0;
			matrizA [i][j] = 1;
			matrizB [i][j] = 1;
		}
	}
*/
      tiempo = mfspr(SPR_TTCR);
      genmatriz (tiempo,matriz,indice);
      tiempo = mfspr(SPR_TTCR);
      genmatriz (tiempo,matrizA,indice);
      tiempo = mfspr(SPR_TTCR);
      genmatriz (tiempo,matrizB,indice);
    

//    tiempo = mfspr(SPR_TTCR);
//    useg = (int) tiempo * 50 ; 
//    uart_print_str("Tick de ejecucion seteo de variables :");
//    uart_print_long((long)tiempo);
//    uart_print_str(" En us = ");
//    uart_print_long(useg);
// 	uart_print_str("\n");
 	 
 
	mtspr(SPR_TTCR, 0x0);
    
	for (i=0;i<indice;i++){
		for (j=0;j<indice;j++){
			for (k=0;k<indice;k++){
				producto = matrizA[i*con+k] * matrizB[k*con+j];
				acumulador = acumulador + producto;
			}
			matriz [i*con+j] = acumulador;
			acumulador = 0;
		}	
	}
    
    tiempo = mfspr(SPR_TTCR);
//    useg = (int) tiempo * 50 ;
//    uart_print_str("Tick de ejecucion de la multiplicacion :");
    uart_print_long((long)tiempo);
//    uart_print_str(" En us = ");
//    uart_print_long(useg);
    uart_print_str("\n");   
    

/*
    uart_print_str("Matriz Resultado");   
    uart_print_str("\n");   
    
    for (i=0;i<indice;i++){
		for (j=0;j<indice;j++){
			uart_print_long(matriz [i*con+j]);
			uart_print_str(" ");
		}
		uart_print_str("\n");
	}
  
 */  
    } 
  
    uart_print_str("Fin del programa\n");

    return 0;
}
\end{lstlisting}

 \subsection{Generador de secuencia binaria seuda aleatoria}
\begin{lstlisting}[language=C,frame=single]
#include "stdio.h"
#include "time.h"
#define con_i 10

#define N 16

void genmatriz(unsigned seed1,unsigned *matriz){
    printf ("seed1 ="); 
    printf("%x",seed1);
    printf("\n");   
    
    //unsigned  seed1 ; 
    unsigned sr1[N] ;
    unsigned i , j ,q  ;
    unsigned tmp;    
    
    for(i=0; i<N; i++){
  	sr1[i] =  (seed1&1);
  	seed1/=2;
    }

    for (q=0;q<(con_i*con_i);q++){
		tmp = sr1[15]^sr1[14]^sr1[13]^sr1[11];
		
		for(j=N;j>0;j--) sr1[j]=sr1[j-1];
		sr1[0] = tmp;

		matriz[q] = tmp;
		
                //printf ("%i ",tmp);
		//if (((q+1)%con_i)==0) printf ("\n");
    }   
	printf ("\n\n");
}

int main(void ) {

unsigned matriz[100] ;
unsigned i,j ;
genmatriz (time(NULL),matriz);
for (i=0;i<100;i++){
	printf ("%i ",matriz[i]);
	if (((i+1)%10)==0) printf ("\n");
}

 return 0;
}
\end{lstlisting}

 \section{Programa de Prueba de SDRAM}
 \begin{lstlisting}[language=C,frame=single]
 #include "cpu-utils.h"
#include "board.h"
#include "sdram.h"
#include "uart.h"
#include "printf.h"

#ifndef _SDRAM_H_
#define _SDRAM_H_

#ifdef  MT48LC32M16A2 // 64MB SDRAM part
#define SDRAM_SIZE 0x04000000
#define SDRAM_ROW_SIZE 2048 // in bytes (10 bits col addr, 2 bytes per)
#define SDRAM_NUM_ROWS_PER_BANK (8192) // 13-bit row address
#define SDRAM_NUM_BANKS 4
#endif

#ifdef  MT48LC16M16A2 // 32MB SDRAM part
#define SDRAM_SIZE 0x02000000
#define SDRAM_ROW_SIZE 1024 // in bytes (9 bits col addr, 2 bytes per)
#define SDRAM_NUM_ROWS_PER_BANK (8192) // 13-bit row address
#define SDRAM_NUM_BANKS 4
#endif

#ifdef MT48LC4M16A2 // 8MB SDRAM part
#define SDRAM_SIZE 0x800000
#define SDRAM_ROW_SIZE 512 // in bytes (8 bits col addr, 2 bytes per)
#define SDRAM_NUM_ROWS_PER_BANK (4096) // 12-bit row address
#define SDRAM_NUM_BANKS 4
#endif


#endif


#define SDRAM_NUM_ROWS (SDRAM_NUM_ROWS_PER_BANK * SDRAM_NUM_BANKS)

#define START_ROW 128

int main()
{

  uart_init(DEFAULT_UART);


  printf("\n\tSDRAM rows test\n");

  printf("\n\tWriting\n");
  
  int i; // Skip first 64KB, code/stack resides there - TODO determine this from
  // stack linker variable!
  for(i=START_ROW;i<(SDRAM_NUM_ROWS);i++)
    {
      REG32((i*(SDRAM_ROW_SIZE))) = i;
      printf("\r\t0x%x", i);
    }

  printf("\n\tReading\n");

  int read_result = 0;

  for(i=START_ROW;i<(SDRAM_NUM_ROWS);i++)
    {
      printf("\r\t0x%x", i);
      read_result = REG32((i*(SDRAM_ROW_SIZE)));
      if (read_result != i)
	{
	  printf("\n\Error at 0x%x, read 0x%x, expected 0x%x\n",
		 (i*SDRAM_ROW_SIZE), read_result, i);
	  report(0xbaaaaaad);
	  report(i);
	  report(read_result);
	  exit(0xbaaaaaad);
	}
    }
  printf("\n\tTest OK.\n");
  exit(0x8000000d);  
}
 
 \end{lstlisting}

 \section{Programa de Prueba Ethernet del Proyecto ORPSoC}
 \begin{lstlisting}[language=C,frame=single]
 
int main ()
{
  
	print_packet_contents = 0; // Default to not printing packet contents.
	packet_inspect_debug = 0;
	print_ethmac_debug_reg = 0;

	/* Initialise vector handler */
	int_init();

	/* Install ethernet interrupt handler, it is enabled here too */
	int_add(ETH0_IRQ, oeth_interrupt, 0);

	/* Enable interrupts */
	cpu_enable_user_interrupts();
    
	last_char=0; /* Variable init for spin_cursor() */
	next_tx_buf_num = 4; /* init for tx buffer counter */

	uart_init(DEFAULT_UART); // init the UART before we can printf
	printf("\n\teth ping program\n\n");
	printf("\n\tboard IP: %d.%d.%d.%d\n",our_ip[0]&0xff,our_ip[1]&0xff,
	       our_ip[2]&0xff,our_ip[3]&0xff);
  
	ethmac_setup(); /* Configure MAC, TX/RX BDs and enable RX and TX in 
			   MODER */
  
	//scan_ethphys(); /* Scan MIIM bus for PHYs */
	//ethphy_init(); /* Attempt reset and configuration of PHY via MIIM */
	//ethmac_scanstatus(); /* Enable scanning of status register via MIIM */

	/* Loop, monitoring user input from TTY */
	while(1)  
	{
		char c;
      
		while(!uart_check_for_char(DEFAULT_UART))
		{
			
			spin_cursor();
			
			if (print_ethmac_debug_reg)
				oeth_print_wbdebug();
		}
      
		c = uart_getc(DEFAULT_UART);

		if (c == 's')
			tx_packet((void*) ping_packet, 98);
		else if (c == 'S')
			tx_packet((void*)big_ping_packet, 1514);
		else if (c == 'h')
			scan_ethphys();
		else if (c == 'i')
			ethphy_init();
		else if (c == 'c')
			oeth_ctrlmode_switch();
		else if (c == 'P')
		{
			print_packet_contents = print_packet_contents ? 0 : 1;
			if (print_packet_contents)
				printf("Enabling packet dumping\n");
			else
				printf("Packet dumping disabled\n");
		}
		else if (c == 'p')
			oeth_printregs();
		else if (c == 'a')
			ethphy_print_status(ethphy_found);
		else if (c >= '0' && c <= '9')
			scan_ethphy(c - 0x30);
		else if (c == 'r')
		{
			ethphy_reset(ethphy_found);
			printf("PHY reset\n");
		}
		else if (c == 'R')
		{
			//oeth_reset_tx_bd_pointer();
			ethmac_setup();
			printf("MAC reset\n");
		}
		else if (c == 'n')
			ethphy_reneg(ethphy_found);
		else if (c == 'N')
			ethphy_toggle_autoneg(ethphy_found);
		else if (c == 'm')
			ethmac_togglehugen();
		else if (c == 't')
			ethphy_set_10mbit(ethphy_found);
		else if (c == 'H')
			ethphy_set_100mbit(ethphy_found);
		else if ( c == 'b' )
		{
			printf("\n\t---\n");
			oeth_dump_bds();
			printf("\t---\n");
		}
      
		else if ( c == 'B' )
		{
			tx_packet((void*) broadcast_ping_packet, 298);
		}
		else if (c == 'u' )
			oeth_transmit_pause();
		else if (c == 'd' )
		{
			oeth_print_wbdebug();
			print_ethmac_debug_reg = !print_ethmac_debug_reg;
		}
		else if (c == 'D')
		{
			marvell_phy_toggle_delay();
		}
		else if (c == 'v' )
		{
			if (packet_inspect_debug)
				printf("Disabling ");
			else 
				printf("Enabling ");
			
			printf("packet type announcments\n");
			
			packet_inspect_debug = !packet_inspect_debug;
		}
		else if ( c == 'o' )
			oeth_toggle_promiscuous();
		else if (c == 'L')
			ethphy_toggle_loopback();
		else if (c == 'g')
			ethphy_toggle_gigadvertise();

	}
	 \end{lstlisting}
	 
	  \section{Programa de Prueba de Compilador y Depurador para ORPSoC - Números Primos}

\begin{lstlisting}[language=C,frame=single]
#include "stdio.h"
#define RANGE 2000

int main(int argc, char *argv[])
{
int i;
int count=0;
int num=1;

while(num<=RANGE){

i=2;
	while(i<=RANGE){
	i=2;
	if(i<=num){
	break;
	i++;}
	
	if(i==num){
	printf("%d is un numero primo\n",i);
	count ++;}
num ++;
}
printf("\nCantidad de numero primos entre 1 y %d : %d\n", RANGE,count);
return 0;
}
 \end{lstlisting}


 \begin{lstlisting}[language=C,frame=single]
Código de prueba de Hilos 
	 \end{lstlisting}
	
	  




