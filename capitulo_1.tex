\chapter{INTRODUCCIÓN}


\section{Descripción General}



\section{Objetivos}
\subsection{Objetivo General}

Implementar un system on chip OpenSource con un microprocesador embebido Soft-core que soporte un sistema operativo libre , con la finalidad de entregar un sitema integral FPGA-SoC-Sistema Operativo completamente funcional y bajo licencia GPL v2.

\subsection{Objetivo Especìfico}
\begin{itemize}
\item Seleccionar, analizar y determinar un microprocesador Sof-Core.
\item Establecer un system on chip Open Source donde poder implementar un Soft-Core.
\item Determinar sistemas operativo con licencia GPL v2 que tengan las prestaciones funcionales adecuadas.
\end{itemize}

\section{Motivación} 

Existe un grupo de cores Sof-Core de código abierto que no están limitados por la tecnología. Los cores destacados de microprocesadores de 32 bits, son los procesadores SPARC LEON OpenRISC 1200 , y el core de LatticeMico32. Usar cores de  codigo abierto,  va unido a una serie de conceptos como:
 \begin {itemize}
\item Flexibilidad. Si el codigo fuente está disponible, los desarrolladores pueden modificar el codigo de acuerdo a sus necesidades.Adémas, se produce un flujo constante de ideas que mejora la calidad del codigo.
\item Fiabilidad y seguridad. Con muchos programadores a la vez escrutando el mismo trabajo, los errores se detectan y corrigen antes, por lo que el producto resultante es mas fiable y eficaz que el comercial.
\item Rapidez de desarrollo. Las actualizaciones y ajustes se realizan a través de una comunicación constante vía Internet.
\item Relación con el usuario. El programador se acerca mucho mas a las necesidades reales de su cliente, y puede crear un producto especifíco para él
 \end {itemize}
 
Obtener un sitema integral de código abierto en donde hay código HDL, assembler y C. Con la principal ventaja del acceso al código pudiendo personalizarlo como por ejemplo en la descripción RTL del SoC para implementar la optimización o funcionalidad deseada y la ausencia de restricciones sobre lo que se puede hacer sobre el sistema final. Ademas de la portabilidad con la que obtengo la capacidad de migrar de una plataforma a otra. Logrando menor dependencia entre el código fuente y la plataforma objetivo. Pudiendo ser usado sobre una ASICs (Application-specific integrated circuit) o con modificaciones menores en cualquier FPGA (Field Programmable Gate Array) de Xilinx, Altera, Lattice, etc. 
Estos tres de los más grandes proveedores de FPGA , Xilinx , Altera y Lattice , ofrecen sus propios micro core RISC de 32bits los dos mayores proveedores de dispositivos FPGA , Altera y Xilinx , proporcionan el micro core Nios y Microblaze, respectivamente. Son micro cores  en donde el codigo fuente RTL no se encuera disponible y solo pueden ser implementados en sus respectivas FPGA.

\section{Metodologìa}



\section{Importancia del Problema}

El softcore OpenRisc  que se encuentra en el SoC OrpSoc y MinSoc se tiene que implementado en una Spartan 3A de Xilinx. Tenemos como fin montar un Linux para validar y verificar el sistema global entregando un sistema funcional bajo licencia libre.
Actualmente las FPGAs nos birndan la posibilidad de implementar estos proyectos, donde el Hardware y el Software son una misma entidad. Este nuevo enfoque nos permite aprovechar la facilidad de implementar soluciones por Hardware.

\section{Alcance de Estudio}
\section{Modelo de Desarollo}
\section{Metodologìa}



 \section{Organización del Proyecto Integrador}

 Una vez detalladas  las motivaciones y expuestas las  ventajas que un
 receptor  coherente puede  aportar a  las comunicaciones  ópticas, el
 presente proyecto tiene como  principal objetivo el estudio, diseño y
 la simulación de los  diferentes métodos de recuperación de portadora
 de un receptor digital coherente para lo cual se organiza
 su contenido de la siguiente manera:\\

%%%%% poner en negrita asi \textit{Python} 


