\documentclass[a4paper,12pt]{article}

\usepackage[spanish]{babel} \usepackage[T1]{fontenc}
\usepackage{times} \usepackage[utf8]{inputenc}

\usepackage{spverbatim} \usepackage{amsmath}
\usepackage[usenames,dvipsnames]{color}
\usepackage{colortbl}

\usepackage{graphicx} 
\usepackage{subfigure}
\usepackage[utf8]{inputenc}

\usepackage[Conny]{fncychap}
\ChTitleVar{\vspace{1.5cm}\centering\Huge\rm\bfseries}
\ChNameVar{\vspace{1.5cm}\centering\Huge\rm\bfseries}

\hyphenation{con-ti-nua-men-te}

%\usepackage[Rejne]{fncychap}
%\usepackage[Bjarne]{fncychap}
%\usepackage{makeidx}
\definecolor{DarkBlue}{rgb}{0.1,0.1,.1}
\usepackage[linkcolor=DarkBlue,urlcolor=DarkBlue,colorlinks,breaklinks]{hyperref}
\usepackage[left=2.5cm,top=3.0cm,right=2.5cm,bottom=3.0cm]{geometry} 
\usepackage{fancyhdr}
\pagestyle{fancy}
\fancyhf{}
\fancyhead[L]{\leftmark} %\fancyhead[R]{\rightmark}
\fancyfoot[L]{Gomez, Roberto Pablo - Lovaisa Michelini, Valeria}
\fancyfoot[R]{P\'agina: \thepage} %\fancyfoot[LE,RO]{P\'agina: \thepage}                               
\renewcommand{\headrulewidth}{1.0pt}
\renewcommand{\footrulewidth}{0.4pt}

\begin{document}


\title{\vspace{2cm} \textbf{Cátedra Proyecto Integrador \\ \line(1,0){350}}\\
  \vspace{0.5cm}\centering{\includegraphics[scale=0.53]{images/logoUNC.png}}\\}
\author{\vspace{-0.25cm} Autor \vspace{0.25cm}\\ Gomez, Roberto Pablo - Lovaisa
Michelini Valeria \bigskip \bigskip \bigskip \\
  Tema \vspace{0.25cm} \\ \textbf{Anteproyecto de Trabajo Final}\\ 
}
\date{}

\maketitle{}
\newpage

\section{Anteproyecto del trabajo final de la asignatura Proyecto Integrador}

\subsection{Título}
Implementación de un Sistema en Chip con Microprocesador Soft-Core y soporte Linux

\subsection{Palabras Clave}
System on Chip, Microprocesador Soft-Core, Sistemas Embebidos, Sistemas Operativos
		
\subsection{Descripción}
Se implementará un sistema en chip (SoC) con microprocesador Soft-Core sobre FPGA. El proyecto estará bajo el paradigma de software libre incluyendo
el código RTL del SoC, las herramientas de desarrollo y los Sistema Operativos. Inicialmente se analizarán las diferentes alternativas de
Microprocesadores Soft-Core que permitan su implentación en proyectos SoC Open-Source tales como MinSoC y ORPSoC. El elemento central de
ambos proyectos es el microprocesador soft OpenRisc1200, de capacidades comparables a los microprocesadores soft MicroBlaze de Xilinx y NiOS II de
Altera. Este procesador está soportado por el compilador GCC y permite correr el sistema operativo Linux. El funcionamiento del sistema se comproborá
implementándose sobre un Kit XtremeDSP Starter Platform disponible en el Centro Universitario de Automatización y Robótica UTN-FRC.


\subsection{Objetivos}

Implementar un system on chip OpenSource con un microprocesador embebido Soft-core que soporte un sistema operativo libre , con la finalidad de
entregar un sitema integral FPGASoC- Sistema Operativo completamente funcional y bajo licencia GPL v2.

\subsection{Antecedentes}

No existen.

\subsection{Datos de los Alumnos}
\subsubsection{Alumno 1}
\begin{itemize}
  \item Apellido : Gomez
  \item Nombre : Roberto Pablo
  \item Matrícula : 200404250
  \item Email : rpg2101@gmail.com
\end{itemize} 
\subsubsection{Alumno 2}
\begin{itemize}
  \item Apellido : Lovaisa Michelini
  \item Nombre : Valeria
  \item Matrícula : 200404512
  \item Email : valerialovaisa@gmail.com
\end{itemize} 

\subsection{Datos de Director y Co-Director}
\subsubsection{Director}
\begin{itemize}
  \item Apellido : Micolini
  \item Nombre : Orlando
  \item Cargo : Director del Lab. de Arquitectura de Computadoras FCEFyN - UNC
  \item Email : omicolini@compuar.com
\end{itemize} 
\subsubsection{Co-Director}
\begin{itemize}
  \item Apellido : Paredes
  \item Nombre : Federico
  \item Cargo : Integrante del Centro Universitario de Automatización y Robótica UTN-FRC
  \item Email : fedesor@gmail.com
\end{itemize} 

\subsection{Metodología}
\begin{itemize}
  	\item Lugar de trabajo : Centro Universitario de Automatización y Robótica (CUDAR) UTN-FRC
	\item Equipamiento Previsto : Placas de desarrollo provistas por el CUDAR :  S3ADSP1800A de Xilinx, Nexys de Digilent.
	\item Costo estimado : No aplicable.
	\item Apoyo económico : El lugar de trabajo cubre el costo total del proyecto.
\end{itemize} 

\begin{thebibliography}{99}	  
\bibitem{Etiqueta00}  Página principal del proyecto MinSOC
\url{http://www.minsoc.com/}

\bibitem{Etiqueta01} Página principal del proyecto microprocesador OpenRISC
\url{http://opencores.org/or1k/OR1200_OpenRISC_Processor}

\bibitem{Etiqueta02} Placa de desarrollo S3ADSP1800A
\url{http://www.xilinx.com/products/boards-and-kits/HW-SD1800A-DSP-SB-UNI-G.htm}

\bibitem{Etiqueta03} Somemerville Ian,2011 .
\textit{Software engineering, 9th edition}

\bibitem{Etiqueta04} D. A. Patterson y J. L. Hennessy ,2012.
 \textit{Computer Architecture A Quantitative Aprroach, 9th edition}

\bibitem{Etiqueta05}Wikipedia, the free encyclopedia, Free and open source software.
\url{http://en.wikipedia.org}

\bibitem{Etiqueta06}Open Source Initiative, The Open Source Definition
\url{http://www.opensource.org/docs/osd}

\bibitem{Etiqueta07}Joseph Feller , 2005 
\textit{Perspectives on Free and Open Source Software.}

\end{thebibliography}

\end{document}
