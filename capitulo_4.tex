\chapter{Software OpenSource y Libre}
la intro al tema contando un poco  las comunidades de codigo abierto y por ulitmo de OpenCore

Ejemplo: 
A medida que la popularidad y la utilidad de Internet ha crecido , también lo han hecho las comunidades de opensource.
La comunicación fue el inicio para las grandes comunidades de opensource. Lo que  dio como resultado un sinnúmero de comunidades y grupos que contribuye a abrir el desarrollo fuente de casi cualquier cosa.

Sitios web de gran tamaño para que las comunidades se centraron en el desarrollo de software de aplicaciones informáticas , como SourceForce , freshmeat , Ohloh y hacia arriba de acogida CPAN de decenas de miles de proyectos . Un grupo llamado Freenode proporciona Internet Relay Chat ( IRC ) servidores donde decenas de miles de desarrolladores de código abierto se reúnen para interactuar .

Hay una serie de pequeños servicios de alojamiento de proyectos libres dirigidos a grupos desarrollo de software como Google Code, Launchpad, GitHub, GNU Savannah,
así como los sitios de la comunidad antes mencionados.

OpenCores es  sitio/ comunidad más grande para el desarrollo de los  IP core de hardware como de código abierto del mundo.
OpenCores.org proporciona el código fuente de los diferentes proyectos HW digitales (IP-cores, SoC, boards, etc) y apoyar a los usuarios con diferentes herramientas, plataformas, foros y otras informaciones útiles. 

		\section{Deferencias}%http://www.slideshare.net/wilberth1594/tesis-alex-8795926,julius
Aca vamos a poner las libertades que permiten uno y otro
		\section{GPL}
		\section{LGPL}
%Revisemos este título -- PABLO JULIUS :P
		\section{OpenSource}
intro a opensource

EJ
La apertura del código de la propiedad intelectual desarrollada para el proyecto OpenRISC, y otros en OpenCores, ha sido a la vez un obstáculo y una ayuda, pone a la vista el estado de el desarrollo, pero es útil, ya que permite que cualquiera pueda participar en el continuo desarrollo de cores. %\Esta sección discutirá los pros y los contras del opensource
		\section{¿Quien tiene el Hardware?} 

la idea es poner pro y contra de trabajar con open source
EJ
El uno de los problemas que enfrentan los desarroladores del opensource que no lo tiene lo desarrolladores  de software es la necesidad de usar plataformas privativas para implementar el diseño.Estas plataformas que como elemento base una FPGA, contienen múltiples periféricos ICs , que tienen que ser comprados , así como la depuración y la programación de hardware.
Se suma a esto que generalmente las complicadas herramientas de programación de los proveedores, 
%así como el desarrollo de hardware curva de aprendizaje relativamente empinada impone a los principiantes , y no es demasiado sorprendente que técnicamente bien las personas que deseen participar en un proyecto de código abierto podrían elegir un software proyectar sobre un proyecto de hardware casi siempre.
 %También es cierto que la utilidad de cualquier diseño de hardware que se podría implementar en FPGA está limitado por el hecho de que se que se hace a un muy bajo nivel de abstracción, y para lograr cualquier resultado " útil" para un experimentador o aficionado, que por lo general requiere una gran cantidad de trabajo a lo largo de muchos niveles de abstracción para lograr algo que es fácilmente utilizable a partir de una interfaz en un modernoPC . Un ejemplo podría ser el desarrollo de un núcleo para llevar a cabo no estándar 
%\I / O con las transacciones de un sensor u otro boutique de IC, para proporcionar información a un programa de asistencia en la automatización del hogar , o el control de un modelo a escala y la similares. Esto requeriría el desarrollo y prueba del modelo de hardware y implementación en FPGA . Suponiendo que no era un microprocesador está ejecutando en ese FPGA proporcionar servicios de red a través de un RTOS , este nuevo módulo personalizado haría luego exigir su capa de software desarrollado , lo que significa que un conductor, y la satisfacción de diversas Ganchos de nivel operativo en la aplicación que se ejecuta en la FPGA , el AM microprocesador, a proporcionar los datos a través del enlace de red . Sólo entonces sería este sensor , los datos del AM y luego estará disponible para la aplicación de nivel superior. Este es sólo un ejemplo en el que , muy probablemente el diseñador podría haber elegido una solución que utiliza un bus estándar, sin embargo no , el AM con frecuencia casos de control personalizado o núcleos de interfaz en FPGAs para proporcionar el acceso a la herencia , o las normas de muy nuevas o esotérica de autobús, y pone de relieve el extra trabajo que se requiere más allá de la escritura RTL para proporcionar la interfaz física. en vista de la cantidad de desarrollo y pruebas requeridas para poner en práctica estas soluciones típicamente , sería fácil sentirse abrumado por la cantidad de trabajo necesario para completar una tarea tan aparentemente trivial.

%Compare esto con el trabajo que participan en empezar a trabajar en una fuente abierta proyecto de software , que normalmente consisten en la descarga de una fuente de desarrollo árbol y el edificio ( en minutos ) que el proyecto con herramientas de desarrollo ya incluidos , o fácilmente obtenido , en el sistema operativo . La aplicación puede entonces ser
%ejecutar en el sistema host para comprobar la funcionalidad y el ciclo de desarrollo en gran medida termina allí. Las diferencias son el acceso inherentes a la plataforma de desarrollo ( el máquina host) , las herramientas de desarrollo más simples ( gcc, make en el sistema host ) y el ciclo de desarrollo y pruebas más corto y más fácil (que se ejecuta en el equipo host a través de un shell. ) 
%A medida que se desarrollan los proyectos de hardware de código abierto más y más ágil sistemas de desarrollo se ponen en su lugar , se puede esperar que estas barreras a entrada se convertirá disminuido . Los primeros días de desarrollo de software de código abierto habría parecido igual de difícil y laborioso . Con el tiempo , sin embargo , mejoras en los proyectos de forma en que se organizan y las herramientas utilizadas para su desarrollo , se han producido. La cantidad de software de código abierto disponible ha crecido y sigue hacerlo a un ritmo creciente . Se puede esperar que con el tiempo y el aumento participación , hardware de código abierto permitirá alcanzar un éxito similar .
		